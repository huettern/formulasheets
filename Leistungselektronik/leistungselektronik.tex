\documentclass[10pt,landscape]{scrartcl}
\usepackage{multicol}
\usepackage{calc}
\usepackage{ifthen}
\usepackage[landscape]{geometry}
\usepackage{amsmath,amsthm,amsfonts,amssymb}
\usepackage{color,graphicx}
\usepackage{hyperref}
\usepackage[font=scriptsize,labelfont=bf]{caption}
\usepackage{wrapfig}
\usepackage{graphicx,calc}
\usepackage{empheq}
\usepackage{xcolor}
%\usepackage{pbox}

\pdfinfo{
  /Title (Leistungelektronik)
  /Creator (TeX)
  /Producer (pdfTeX 1.40.0)
  /Author (Noah Huetter)
  /Subject (Leistungelektronik)
  /Keywords (Leistungelektronik,ETH)}

\definecolor{section}{RGB}{28,82,83}
\definecolor{titletext}{RGB}{0,0,0}
\definecolor{formula}{RGB}{249,220,155}

\setkomafont{section}{\mysection}
\newcommand{\mysection}[1]{%
    \Large\bf%
    \setlength{\fboxsep}{0cm}%already boxed
    \fcolorbox{black}{gray!40}{%
        \begin{minipage}{\linewidth}%
            \vspace*{2pt}%Space before
            #1
            \vspace*{2pt}%Space after
        \end{minipage}%
    }}

% This sets page margins to .5 inch if using letter paper, and to 1cm
% if using A4 paper. (This probably isn't strictly necessary.)
% If using another size paper, use default 1cm margins.
\ifthenelse{\lengthtest { \paperwidth = 11in}}
    { \geometry{top=.5in,left=.5in,right=.5in,bottom=.5in} }
    {\ifthenelse{ \lengthtest{ \paperwidth = 297mm}}
        {\geometry{top=1cm,left=1cm,right=1cm,bottom=1cm} }
        {\geometry{top=1cm,left=1cm,right=1cm,bottom=1cm} }
    }

% Turn on simple page number
\pagestyle{plain}

% Redefine section commands to use less space
\makeatletter
% \renewcommand{\section}{\@startsection{section}{1}{0mm}%
%                                 {-1ex plus -.5ex minus -.2ex}%
%                                 {0.5ex plus .2ex}%x
%                                 {\normalfont\large\bfseries}}
\renewcommand{\subsection}{\@startsection{subsection}{2}{0mm}%
                                {-1explus -.5ex minus -.2ex}%
                                {0.5ex plus .2ex}%
                                {\normalfont\normalsize\bfseries}}
\renewcommand{\subsubsection}{\@startsection{subsubsection}{3}{0mm}%
                                {-1ex plus -.5ex minus -.2ex}%
                                {1ex plus .2ex}%
                                {\normalfont\small\bfseries}}
\makeatother

% bring the old commands back as needed via:
\makeatletter
\DeclareOldFontCommand{\rm}{\normalfont\rmfamily}{\mathrm}
\DeclareOldFontCommand{\sf}{\normalfont\sffamily}{\mathsf}
\DeclareOldFontCommand{\tt}{\normalfont\ttfamily}{\mathtt}
\DeclareOldFontCommand{\bf}{\normalfont\bfseries}{\mathbf}
\DeclareOldFontCommand{\it}{\normalfont\itshape}{\mathit}
\DeclareOldFontCommand{\sl}{\normalfont\slshape}{\@nomath\sl}
\DeclareOldFontCommand{\sc}{\normalfont\scshape}{\@nomath\sc}
\makeatother

% make smaller enumerates
\usepackage{enumitem}
\setlist[enumerate]{itemsep=-1mm,leftmargin=3mm}

% Define BibTeX command
\def\BibTeX{{\rm B\kern-.05em{\sc i\kern-.025em b}\kern-.08em
    T\kern-.1667em\lower.7ex\hbox{E}\kern-.125emX}}

% Don't print section numbers
\setcounter{secnumdepth}{0}


\setlength{\parindent}{0pt}
\setlength{\parskip}{1pt plus 0.5ex}

%My Environments
\newtheorem{example}[section]{Example}
% -----------------------------------------------------------------------
\newcommand{\eqn}[3]
{
  \begin{minipage}[#1]{#2}
    \[ #3 \]
  \end{minipage}
}
\newcommand{\feqn}[3]
{
\fbox{
  \begin{minipage}[#1]{#2}
    \[ #3 \]
  \end{minipage}
}
}

% equation box        
\newcommand{\eqbox}[1]{\fcolorbox{black}{formula}{\hspace{0.5em}$\displaystyle#1$\hspace{0.5em}}}

\newcommand{\Pabl}[2] {\frac{\partial#1}{\partial#2}}
\newcommand{\MATR}[1]{ \displaystyle \left( \begin{matrix} #1 \end{matrix} \right)}
\newcommand{\MATRABS}[1]{ \displaystyle \left| \begin{matrix} #1 \end{matrix} \right|}
\newcommand{\Int}{\int\limits}
\newcommand\warning{%
 \makebox[1.4em][c]{%
 \makebox[0pt][c]{\raisebox{.1em}{\small!}}%
 \makebox[0pt][c]{\color{red}\Large$\bigtriangleup$}}}%

% \DeclareSIUnit[]\\textup{rms}
% {\text{\ensuremath{m_{\textup{p}}}}}

 
\begin{document}
\raggedright
\footnotesize
\begin{multicols}{2}



% multicol parameters
% These lengths are set only within the two main columns
%\setlength{\columnseprule}{0.25pt}
\setlength{\premulticols}{1pt}
\setlength{\postmulticols}{1pt}
\setlength{\multicolsep}{1pt}
\setlength{\columnsep}{2pt}

\begin{center}
     \Large{\underline{Leistungelektronik}} \\
\end{center}
% -----------------------------------------------------------------------
\IfFileExists{../build/revision.tex}{
  \input{../build/revision.tex}
  Author: Noah Huetter \hfill Date: \compiledate \hspace{1em} Commit: \revision
}{Author: Noah Huetter}


% -----------------------------------------------------------------------
 \vspace{2mm} \hrule
\section{Allgemein}
\subsection{Spule und Kondensator}
\begin{align*}
   i_L(t) &= \frac{1}{L}\int u_L(t)\mathrm{d}t &
   u_L(t) &= L\frac{\mathrm{d}}{\mathrm{d}t}i_L(t) &
   i_C(t) &= C\frac{\mathrm{d}}{\mathrm{d}t}u_C(t) &
   u_C(t) &= \frac{1}{C}\int i_C(t)\mathrm{d}t 
\end{align*}


\eqn{h}{0.15\linewidth}{ Q=C\cdot U }


\subsection{Effektiv- und Mittelwerte}
\begin{align*}
  V_{\textup{rms}} &= \sqrt{ \frac{1}{T} \int_T u^2(t)dt} &
  V_{\textup{rms,sin}} &= \hat{V} / \sqrt{2} &
  V_{\textup{rms,pwm}} &= \hat{V} \sqrt{D} \\
  V_{\textup{avg}} &= \hat{V}\frac{1}{2\pi}\Int_{0}^{2\pi}|\sin(x)| \mathrm{d}x =\hat{V}\frac{2}{\pi}
\end{align*}

% -----------------------------------------------------------------------
\vfill\null
\columnbreak
\section{Tiefsetzsteller}

\begin{center}
\includegraphics[width=0.8\linewidth]{img/sch_buck.png}%
% \captionof{figure}{Tiefsetzsteller}%
\end{center}

\subsection{Kontinuierlicher Stromfluss}

\begin{tabular}{c c}
  \includegraphics[width=0.4\linewidth]{img/buck/cont.png}%
  &
  \includegraphics[width=0.4\linewidth]{img/buck/out.png}%
\end{tabular}

\eqn{h}{0.25\linewidth}{ D = \frac{I_1}{I_2} = \frac{U_2}{U_1} }
\eqn{h}{0.25\linewidth}{ \Delta i_{Lpp} = \frac{U_2}{L}(1-D)T_S}
\eqn{h}{0.25\linewidth}{ \Delta u_{2pp}=\Delta U_C = \frac{U_2}{L}(1-D) T_S \frac{T_S}{8 \cdot C}}

\subsection{Diskontinuierlicher Stromfluss}
\begin{tabular}{c c}
  \includegraphics[width=0.4\linewidth]{img/buck/discont.png}%
  &
  \includegraphics[width=0.4\linewidth]{img/buck/discplot.png}%
\end{tabular}

Lückender Betrieb für $I_L < I_{2g}$.

\eqn{h}{0.4\linewidth}{ I_{2g} = \frac{1}{2} \Delta i_{Lpp} = \frac{U_2}{2 \cdot L}(1-D)T_S}
\eqn{h}{0.25\linewidth}{ R_{g} = \frac{2L}{(1-D) \cdot T_S}}
\eqn{h}{0.25\linewidth}{ U_2=U_1 \frac{D_1}{D_1+D_2} }
\eqn{h}{0.3\linewidth}{ I_{2g,max}=I_{2g}|_{D=0.5} = \frac{U_1 T_S}{8L} }
\eqn{h}{0.3\linewidth}{ \frac{U_2}{U_1}=\frac{D^2}{D^2+\frac{1}{4} \frac{I_2}{I_{2g,max}} } }
\eqn{h}{0.25\linewidth}{ D = \frac{1}{2} \sqrt{\frac{I_2/I_{2g,max}}{U_1/U_2-1}} }

% -----------------------------------------------------------------------
\vfill\null
\columnbreak
\section{Hochsetzsteller}

% \begin{center}
% \includegraphics[width=0.8\linewidth]{img/sch_boost.png}%
% \captionof{figure}{Hochsetzsteller}%
% \end{center}

\subsection{Kontinuierlicher Stromfluss}
\begin{tabular}{c c}
  \includegraphics[width=0.4\linewidth]{img/sch_boost.png}%
  &
  \includegraphics[width=0.4\linewidth]{img/boost/cont.png}%
\end{tabular}

\eqn{h}{0.23\linewidth}{ D = 1 - \frac{U_1}{U_2} }
\eqn{h}{0.23\linewidth}{ \frac{U_2}{U_1} = \frac{1}{1-D}}
\eqn{h}{0.23\linewidth}{ \frac{I_2}{I_1}=1-D }
\eqn{h}{0.23\linewidth}{ \Delta i_{Lpp} = \frac{U_1}{L}DT_S}

\eqn{h}{0.23\linewidth}{ \Delta U_{2pp} = \frac{U_2}{R}\frac{DT_s}{C_2} }

\subsection{Diskontinuierlicher Stromfluss}
\begin{tabular}{c c}
  \includegraphics[width=0.4\linewidth]{img/boost/disc1.png}%
  &
  \includegraphics[width=0.4\linewidth]{img/boost/disc2.png}%
\end{tabular}

\eqn{h}{0.3\linewidth}{ I_{Lg} = \frac{1}{2} \frac{(U_2-U_1)}{L}(1-D)T_s }
\eqn{h}{0.3\linewidth}{ I_{2g} = \frac{1}{2} \frac{U_2}{L}D(1-D)^2T_s }
\eqn{h}{0.3\linewidth}{ I_{2g,max} = I_{2g}|_{D=1/3} = \frac{2}{27}\frac{U_2}{L}T_S }

\eqn{h}{0.3\linewidth}{ D= \sqrt{\frac{4}{27}\frac{U_2}{U_1} \left (\frac{U_2}{U_1}-1 \right ) \frac{I_2}{I_{2g,max}} } }
\eqn{h}{0.3\linewidth}{ I_{1g} = \frac{1}{2} \frac{U_2}{L}D(1-D)T_s }

% \eqn{h}{0.25\linewidth}{ }
% -----------------------------------------------------------------------
\vfill\null
\columnbreak
\section{Tief-Hochsetzsteller}

\begin{center}
\includegraphics[width=0.8\linewidth]{img/sch_invers.png}%
% \captionof{figure}{Tief-Hochsetzsteller}%
\end{center}

\subsection{Kontinuierlicher Stromfluss}
\eqn{h}{0.23\linewidth}{ \frac{U_2}{U_1} = -\frac{D}{1-D} }
\eqn{h}{0.23\linewidth}{ D = \frac{U_2}{U_2 - U_1} }
\eqn{h}{0.23\linewidth}{ \frac{I_2}{I_1}=\frac{1-D}{D} }

\subsection{Diskontinuierlicher Stromfluss}
\begin{tabular}{c c}
  \includegraphics[width=0.4\linewidth]{img/inv/disc.png}%
  &
  % \includegraphics[width=0.4\linewidth]{img/boost/disc2.png}%
\end{tabular}

\eqn{h}{0.3\linewidth}{ I_{Lg} = \frac{1}{2} \frac{-U_2}{L}(1-D)T_s }
\eqn{h}{0.3\linewidth}{ I_{2g} = \frac{1}{2} \frac{-U_2}{L}(1-D)^2T_s }
\eqn{h}{0.3\linewidth}{ I_{2g,max} = -\frac{1}{2} \frac{U_2}{L}T_s }

\eqn{h}{0.3\linewidth}{ D = -\frac{U_2}{U_1} \sqrt{\frac{I_2}{I_{2g,max}}} }
\eqn{h}{0.65\linewidth}{ P_2 = E_L f_s = \frac{1}{2}L \cdot i_{Lmax}^2 \cdot f_s = \frac{L}{2} f_s \left( \frac{U_1 D T_s}{L} \right)^2 = \frac{U_1^2D^2}{2L}T_s }


% -----------------------------------------------------------------------
\vfill\null
\columnbreak
\section{Dioden-Gleichrichter}

\begin{tabular}{c c}
  \includegraphics[width=0.49\linewidth]{img/dioderect/sch.png}%
  &
  \includegraphics[width=0.49\linewidth]{img/dioderect/dia.png}%
\end{tabular}

\begin{align*}
  \alpha &= \arcsin(U_d/\hat{U_1}) = \arctan\left(\frac{1-\cos(\beta)}{\beta-\sin(\beta)}\right) & \\
  I_d &= \frac{1}{\pi}\int_{\alpha}^{\alpha+\beta}i_d(\omega t)\mathrm{d}\omega t = \frac{1}{\pi}\frac{\hat{U_1}}{\omega L_1} \left[\frac{\hat{U_1}}{U_d}(1-\cos(\beta))-\frac{U_d}{\hat{U_1}}\frac{\beta^2}{2}\right]
\end{align*}



% -----------------------------------------------------------------------
\vfill\null
\columnbreak
\section{PFC Einphasen Gleichrichter}

\begin{center}
\includegraphics[width=0.8\linewidth]{img/sch_pfc.png}%
% \captionof{figure}{PFC Einphasen Gleichrichter}%
\end{center}

\subsection{Stationärer Betrieb}
L wählen mit $D(\omega t)=1$ and $U_{1,\textup{rms}}=$max. $M$: Minimales Spannungsübersetzungsverhältnis ($[M\dots\infty]$).
\begin{align*}
  D(\omega t) &= \left(1-\frac{U_1(\omega t)}{U_d}\right) & 
  \Delta i_{Lpp} (\omega t) &= D(\omega t) T_s \frac{\hat{U_1}\sin(\omega t)}{L} &
  \overline{i_L}(\omega t) &= \hat{I_1} |\sin(\omega t)| = \sqrt{2} \frac{P_d}{U_{1,\textup{rms}}}|\sin(\omega t)| \\
  \Aboxed {\overline{i_L}(\omega t) &\geq \frac{1}{2}\Delta i_{Lpp}} &
  L &\geq \frac{U_{1,\textup{rms}}^2T_s}{2P_d} &
  M &= \frac{U_d}{\sqrt{2} \cdot U_{1,\textup{rms}} }, D=\left(1-\frac{1}{M}\right) \\
  C_d &> \frac{1}{2\omega\hat{U}_{Cd}/U_d}\frac{P_d}{U_d^2} &
  \hat{U}_{Cd} &= \frac{P_d}{U_d}\frac{1}{2\omega C_d}
\end{align*}

$M$: Minimales Spannungsübersetzungsverhältnis ($[M\dots\infty]$). Das maximum des Eingangsstromrippels $\Delta i'_{pp}$ tritt auf bei $D_{\textup{max}}$. Die Phase berechnet sich durch Umformen von $D(\omega t)$.
\begin{align*}
  D_{\textup{max}} &= \begin{cases} M, & M > 2,\\ 0.5, &M < 2. \end{cases} &
  \Delta i'_{pp} &= DT_s\frac{\sqrt{2}U_{1,\textup{rms}}}{L} &
\end{align*}

\subsection{Bauteilbelastung}
\begin{align*}
  I_{\text{TH,avg}} &= \left(\frac{2}{\pi}-\frac{1}{2M}\right)\hat{I_1} &
  I_{\text{TH,rms}} &= \sqrt{\left(\frac{1}{2}-\frac{4}{3\pi M}\right)\hat{I_1^2}} &
  I_{\text{D,avg}} &= \frac{1}{2M}\hat{I_1} &
  I_{\text{D,rms}} &= \sqrt{\frac{4}{3\pi M}\hat{I_1^2}} \\
  I_{\text{Cd,rms}} &= \sqrt{\frac{1}{M}\left(\frac{4}{3\pi}-\frac{1}{4M}\right)\hat{I}_1^2} 
\end{align*}

\subsection{Toleranzbandregelung}
\eqn{h}{0.23\linewidth}{ i'_{max}=(1+k)i'^* }
\eqn{h}{0.23\linewidth}{ i'_{min}=(1-k)i'^* }
\eqn{h}{0.23\linewidth}{ \Delta i_{pp} = 2k\cdot \hat{i}'^* |sin(\omega t)| }



% -----------------------------------------------------------------------
\vfill\null
\columnbreak
\section{Netzgeführte Stromrichter}

\begin{center}
\includegraphics[width=0.8\linewidth]{img/sch_netz_dreiphase.png}%
% \captionof{figure}{Dreiphasen netzgeführte Stromrichter}%
\end{center}

Text

% -----------------------------------------------------------------------
\vfill\null
\columnbreak
\section{Induktivität Dimensionierung}

\begin{minipage}[h]{0.6\linewidth}
  \begin{tabular}{l l l |}
    $A_n$   & Wicklungsfenster      & $mm^2 = 10^{-6}m^2$ \\
    $A_e$   & Eisenquerschnitt      & $mm^2 = 10^{-6}m^2$ \\
    $B_S$   & Sättigungsinduktion   & $\approx 0.3T$ \\
    $k_f$   & Füllfaktor            & $\approx 0.5$ \\
    $S$     & zulässige Stromdichte & $\approx 5\frac{A}{mm^2}=5\cdot10^6\frac{A}{m^2}$ \\
    $d$     & Luftspalt             & $m$ \\
    $\mu_0$ & Vacuum permeability   & $1.256\cdot10^{-6}Vs/(Am)$
  \end{tabular}
\end{minipage}
\begin{minipage}[h]{0.3\linewidth}
\textbf{Vorgehen}
  \begin{enumerate}
    \item Berechne Flächenprodukt $A_e A_n$
    \item Suche Kern mit grösserem Flächenprodukt
    \item Berechne N
    \item Berechne Luftspalt
  \end{enumerate}
\end{minipage}


\eqn{h}{0.23\linewidth}{ A_e > \frac{L\hat{I}}{B_S N} }
\eqn{h}{0.23\linewidth}{ A_n > \frac{NI_{\textup{rms}}}{k_f S} }
\eqn{h}{0.23\linewidth}{ A_e A_n = \frac{L \hat{I} I_{\textup{rms}}}{B_S k_f S} }
\eqn{h}{0.23\linewidth}{ \frac{L \hat{I} }{A_e S} < N < \frac{A_n k_f S}{I_{\textup{rms}}} }

\eqn{h}{0.23\linewidth}{ A_{cu} S = I_{max} }
\eqn{h}{0.23\linewidth}{ d = \frac{R_{tot}\mu_0 A_e}{2} = \frac{N^2 \mu_0 A_e}{2L} }


% -----------------------------------------------------------------------
\vfill\null
\columnbreak
\section{Trafo Dimensionierung}

\begin{minipage}[h]{0.6\linewidth}
  \begin{tabular}{l l l |}
    $A_n/A_w$   & Wicklungsfenster      & $mm^2 = 10^{-6}m^2$ \\
    $A_e$   & Eisenquerschnitt      & $mm^2 = 10^{-6}m^2$ \\
    $B_S$   & Sättigungsinduktion   & $\approx 0.3T$ \\
    $k_f$   & Füllfaktor            & $\approx 0.5$ \\
    $S$     & zulässige Stromdichte & $\approx 5\frac{A}{mm^2}=5\cdot10^6\frac{A}{m^2}$ \\
    $d$     & Luftspalt             & $m$ \\
    $\mu_0$ & Vacuum permeability   & $1.256\cdot10^{-6}Vs/(Am)$
  \end{tabular}
\end{minipage}
\begin{minipage}[h]{0.3\linewidth}
\textbf{Vorgehen}
  \begin{enumerate}
    \item Berechne Flächenprodukt $A_e A_n$
    \item Suche Kern mit grösserem Flächenprodukt
    \item Berechne N
    \item Berechne Luftspalt
  \end{enumerate}
\end{minipage}

\textbf{Dimensionierung}
\begin{align*}
  A_EA_W &= \frac{U_1}{f_p}\frac{D}{B_S}\frac{2\frac{I_{1,avg}}{\sqrt{D}}}{k_f S_{\textup{rms}}} \approx \frac{U_1I_{1,avg}}{f_p} & 
  \left\lceil \frac{\hat{\Psi}_{phys}}{A_EB_{sat}} \right\rceil &\leq N \leq
  \left\lfloor \frac{1}{2}\frac{S_{\textup{rms}}}{I_{p,\textup{rms}}k_f A_w} \right\rfloor &
  \hat{\Psi}_{phys} &= \max\int_{T}^{}u_p(t)\mathrm{d}t
\end{align*}
Typisch: $N$ das Minimum nehmen. Falls kein $N$ wählbar ist, anderen Kern wählen.

\textbf{Hauptinduktivität}
\begin{align*}
  L_h &= N_p^2\frac{1}{R_m} = \frac{N_p^2A_e\mu_r\mu_0}{l_e} & l_e \quad \text{Mittlere Wdg.länge}
\end{align*}

\textbf{Verluste}
\begin{align*}
  P_{cu} &= R_{cu,p}I_{1,\textup{rms}}^2+R_{cu,s}I_{2,\textup{rms}}^2 &
  R_{cu} &= \rho_{cu} \frac{l_{cu}}{A_{cu}} &
  l_{cu} &= N\cdot l_n &
  A_{cu} &= \frac{I_{\textup{rms}}}{S_{\textup{rms}}}
\end{align*}

\textbf{Streuung}
Problem: Nicht alle Induktionslinien von einer Seite sind mit der anderen Seite verkettet. 

\begin{minipage}[h]{0.6\linewidth}
  \includegraphics[width=\linewidth]{img/trafo/trafo_tech_esb.png}%
\end{minipage}
\begin{minipage}[h]{0.3\linewidth}\begin{align*}
      &\oint_{\partial A}\vec{H}\mathrm{d}\vec{s} = H\cdot h \\
      W_m &= \frac{1}{2}\mu_0 \iiint_V H^2\mathrm{d}V
    \end{align*}
\end{minipage}
% -----------------------------------------------------------------------
\vfill\null
\columnbreak
\section{Sperrwandler}

\begin{center}
\includegraphics[width=0.8\linewidth]{img/sch_sperr.png}%
% \captionof{figure}{Sperrwandler}%
\end{center}

Text

% -----------------------------------------------------------------------
\vfill\null
\columnbreak
\section{Durchflusswandler}

\begin{center}
\includegraphics[width=0.8\linewidth]{img/sch_durchfluss.png}%
% \captionof{figure}{Zwei-Transistor Durchflusswandler}%
\end{center}

Text

% -----------------------------------------------------------------------
\vfill\null
\columnbreak
\section{Einphasen Gleichspannungswechselrichter}

\begin{center}
\includegraphics[width=0.8\linewidth]{img/sch_fullbridge.png}%
% \captionof{figure}{Einphasiger Gleichspannungswechselrichter}%
\end{center}

Text

% -----------------------------------------------------------------------
\vfill\null
\columnbreak
\section{Dreiphasen Gleichspannungswechselrichter}
Text

\clearpage

% You can even have references
\rule{0.3\linewidth}{0.25pt}
%\scriptsize
%\bibliographystyle{abstract}
%\bibliography{refFile}
\end{multicols}
\end{document}
