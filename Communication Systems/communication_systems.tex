

% Communication Systems D-ITET
% ===========================================================================
% @Author: Noah Huetter
% @Date:   2019-09-24 17:26:28
% @Last Modified by:   noah
% @Last Modified time: 2019-09-24 17:19:12
% ---------------------------------------------------------------------------

\documentclass[a4paper, fontsize=8pt, landscape, DIV=1]{scrartcl}
\usepackage{lastpage}
\usepackage{hyperref}
% Include general settings and customized commands
%
% General packages and settings
% ===========================================================================
% Author:			Silvano Cortesi (cortesis@student.ethz.ch)
% Version:			1.2
% Last changed:		03.01.2018
%
% ---------------------------------------------------------------------------




\usepackage[german,british]{babel} %choose your language \usepackage[german]{babel}
%\usepackage[T1]{fontenc}
\usepackage[utf8]{inputenc}
\usepackage{fancyhdr}
%\usepackage{lastpage}
%\usepackage{lmodern}
%\usepackage{enumerate}
%\usepackage{float} % for positioning of figures
\usepackage[landscape, margin=1cm]{geometry}
\usepackage[dvipsnames]{xcolor}
\usepackage{pdfpages}


%% Math %%
\usepackage{amscd}
\usepackage{blindtext}
\usepackage{enumitem}
\usepackage{multicol}
\usepackage{parskip}
\usepackage{empheq}
\usepackage{amsmath}
\usepackage{amsfonts}
\usepackage{amssymb}
\usepackage{amsthm}
%\usepackage{dsfont}
%\usepackage{esint} % provides \oiint
\usepackage{mathrsfs}
%\usepackage{trfsigns}
%\numberwithin{equation}{subsection}
%\usepackage{numprint}

%% Graphics & Charts %%
\usepackage{graphicx}
%\usepackage{pdfpages}
%\usepackage{booktabs}
%\usepackage{array}
%\usepackage{paralist}
%\usepackage{framed}
%\usepackage{trfsigns}
\usepackage{tikz}
%\usepackage[lofdepth,lotdepth]{subfig}
%\usepackage{tikz}  %Graphen zeichnen
%\usetikzlibrary{decorations.pathmorphing}
%\usetikzlibrary{arrows.meta,arrows}
%\usepackage{pgfplots}
%% General Settings %%
%\setlength{\parindent}{0px}
%\setkomafont{captionlabel}{\normalfont\bfseries}

%\pagestyle{fancy}
%\lfoot{\tiny \today}
%\rfoot{\thepage\  / \pageref{LastPage}}
%\cfoot{}
%\renewcommand{\footrulewidth}{0.4pt}

%% provides command \uline{} for underlining words
%\usepackage{ulem}

%% colour headings
%\usepackage{color}
%\definecolor{bluen}{cmyk}{1,0.5,0,0}
%\definecolor{bloodorange}{cmyk}{0,.92,1,.2}
%\addtokomafont{section}{\color{bloodorange}}
%\addtokomafont{subsection}{\color{bloodorange}}
%\addtokomafont{subsubsection}{\color{bloodorange}}
%\addtokomafont{paragraph}{\small\color{bloodorange}}
%\addtokomafont{subparagraph}{\small\color{bloodorange}}

%% Signs & Special Formating %%
%\usepackage{ulem} %normalem: \emph{Text} is italic again.
%\usepackage{multicol,multirow}
%\usepackage{tabularx}
%\usepackage{stackrel}
%\usepackage{makeidx}
%\usepackage{mparhack} % bessere margiale bei seitenumbruch

% make document compact
\usepackage[compact]{titlesec}
\titlespacing{\section}{0pt}{*1}{*1}
\titlespacing{\subsection}{0pt}{*1}{*1}
\titlespacing{\subsubsection}{0pt}{*1}{*1}

\parindent 0pt
\pagestyle{empty}
\setlength{\unitlength}{1cm}
\setlist{leftmargin = *}

%include also newer PDF
\pdfminorversion=6

% Set the color of your style
% Avaiable are: Apricot, Aquamarine, Bittersweet, Black, Blue, blue, BlueGreen, BlueViolet, BrickRed, Brown, BurntOrange, CadetBlue, CarnationPink, Cerulean, CornflowerBlue, Cyan, Dandelion, DarkOrchid, Emerald, ForestGreen, Fuchsia, Goldenrod, Gray, Green, GreenYellow, JungleGreen, Lavender, ... (more at: http://en.wikibooks.org/wiki/LaTeX/Colors)
\def\StyleColor{BrickRed}

%
% General commands
% ===========================================================================
% Author:			Silvano Cortesi (cortesis@student.ethz.ch)
% Version:			1.2
% Last changed:		03.01.2018
%
% ---------------------------------------------------------------------------

%..ROEMISCHE_ZAHLEN
	\newcommand{\Roe}[1]{\uppercase\expandafter{\romannumeral #1 }}

%..ZAHLENMENGEN
	\newcommand{\N}{\mathbb{N}}
	\newcommand{\Z}{\mathbb{Z}}
	\newcommand{\Q}{\mathbb{Q}}
	\newcommand{\R}{\mathbb{R}}
	\newcommand{\real}{\R}
	\newcommand{\C}{\mathbb{C}}
	\newcommand{\complex}{\C}
	\newcommand{\0}{\mathbb{O}}
	\newcommand{\F}{\mathbb{F}}
	\newcommand{\K}{\mathbb{K}}
    \newcommand{\angstrom}{\textup{\AA}}
    
%..PFEILE
	\renewcommand{\leadsto}{\Longrightarrow}
	\newcommand{\leftrightleadsto}{\Longleftrightarrow}

%..VEKTOREN
	\newcommand{\Ul} {\underline}
	\newcommand{\vEx} {\vec{e}_x}
	\newcommand{\vEy} {\vec{e}_y}
	\newcommand{\vEz} {\vec{e}_z}
	\newcommand{\vEq} {\vec{e_1}}
	\newcommand{\vEw} {\vec{e_2}}
	\newcommand{\vEe} {\vec{e_3}}
	\newcommand{\transpose} {^{\text{T}}}
	\newcommand{\vect}[1]{\boldsymbol{#1}}
	
%..MATRIX
    \newcommand{\MATR}[1]{ \displaystyle \left( \begin{matrix} #1 \end{matrix} \right)}
    \newcommand{\MATRABS}[1]{ \displaystyle \left| \begin{matrix} #1 \end{matrix} \right|}


%..KOMPLEXE ZAHLEN
	\renewcommand{\Re}{\text{Re}\,}
	\renewcommand{\Im}{\text{Im}\,}

%..OPERATOREN
	\DeclareMathOperator{\grad}{grad}
	\renewcommand{\div}{\text{div}\,}
    	\DeclareMathOperator{\rot}{rot}
    	\DeclareMathOperator{\divg}{div}
    	\DeclareMathOperator{\Tr}{Tr}
    	\DeclareMathOperator{\const}{const}
	\DeclareMathOperator{\imag}{i}
	\newcommand{\Lapl}{\hbox{\footnotesize{$\Delta$}}}

%..DIFFERENTIALRECHNUNG
	\newcommand{\Dx} {\,\mathrm{d}}
	\newcommand{\abl}[1] {\frac{\mathrm{d}}{\mathrm{d}#1}}
	\newcommand{\Abl}[2] {\frac{\mathrm{d}#1}{\mathrm{d}#2}}
	\newcommand{\ablq}[1] {\frac{\mathrm{d^2}}{\mathrm{d}#1^2}}
	\newcommand{\Ablq}[2] {\frac{\mathrm{d^2}#1}{\mathrm{d}#2^2}}
	\newcommand{\pabl}[1] {\frac{\partial}{\partial#1}}
	\newcommand{\pablq}[1] {\frac{\partial^2}{\partial#1^2}}
	\newcommand{\Pabl}[2] {\frac{\partial#1}{\partial#2}}
	\newcommand{\Pablq}[2] {\frac{\partial^2#1}{\partial#2^2}}

%..INTEGRALRECHNUNG
	\newcommand{\dint}{\displaystyle{\int}}
	\newcommand{\intab}{\int^b_a}
	\newcommand{\intinf}{\int_{-\infty}^\infty}
	\newcommand{\dintab}{\displaystyle{\int^b_a}}
	\newcommand{\dintpi}{\displaystyle{\int^{\pi}_{-\pi}}}
	\newcommand{\dintzpi}{\displaystyle{\int^{2\pi}_{\mbox{-}2\pi}}}
	\newcommand{\dA}{\hspace{4pt}\mathrm{d}A}
	\newcommand{\dx}{\hspace{4pt}\mathrm{d}x}
	\newcommand{\dy}{\hspace{4pt}\mathrm{d}y}
	\newcommand{\dz}{\hspace{4pt}\mathrm{d}z}
	\newcommand{\dr}{\hspace{4pt}\mathrm{d}r}
	\newcommand{\ds}{\hspace{4pt}\mathrm{d}s}
	\newcommand{\dS}{\hspace{4pt}\mathrm{d}S}
	\newcommand{\dt}{\hspace{4pt}\mathrm{d}t}
	\newcommand{\dm}{\hspace{4pt}\mathrm{d}m}
	\newcommand{\dk}{\hspace{4pt}\mathrm{d}k}
	\newcommand{\dl}{\hspace{4pt}\mathrm{d}l}
	\newcommand{\du}{\hspace{4pt}\mathrm{d}u}
	\newcommand{\dv}{\hspace{4pt}\mathrm{d}v}
	\newcommand{\dV}{\hspace{4pt}\mathrm{d}V}
	\newcommand{\dphi}{\hspace{4pt}\mathrm{d}\varphi}
	\newcommand{\domega}{\hspace{4pt}\mathrm{d}\omega}
	\newcommand{\dvarsigma}{\hspace{4pt}\mathrm{d}\varsigma}
	\newcommand{\dtau}{\hspace{4pt}\mathrm{d}\tau}
	\newcommand{\dtheta}{\hspace{4pt}\mathrm{d}\vartheta}
	\newcommand{\dmu}{\hspace{4pt}\mathrm{d}\mu}
	\newcommand{\dxi}{\hspace{4pt}\mathrm{d}\xi}
	\newcommand{\deta}{\hspace{4pt}\mathrm{d}\eta}
	\newcommand{\dvecl}{\hspace{4pt}\mathrm{d}\vec{l}}
	\newcommand{\dvecS}{\hspace{4pt}\mathrm{d}\vec{S}}

%..LIMES
    \DeclareMathOperator{\limni}{\lim\limits_{n\to\infty}}
    \DeclareMathOperator{\limxi}{\lim\limits_{x\to\infty}}
    \DeclareMathOperator{\limho}{\lim\limits_{h\to0}}
    \newcommand{\limxai}[1]{\ensuremath{\lim\limits_{x\to #1}}}

%..SUMMEN
    \DeclareMathOperator{\sumni}{\sum_{n=0}^{\infty}}
    \newcommand{\sumnia}[1]{\ensuremath{\sum_{n=#1}^{\infty}}}


%..PARTIELLE ABLEITUNG
    \DeclareMathOperator{\partf}{\dfrac{\partial f}{\partial x}}
    \newcommand{\partfo}[1]{\ensuremath{\dfrac{\partial f}{\partial #1}}}
    \newcommand{\parto}[1]{\ensuremath{\dfrac{\partial }{\partial #1}}}
    \newcommand{\partt}[2]{\ensuremath{\dfrac{\partial^2 }{\partial #1\partial #2}}}
    \newcommand{\partq}[1]{\ensuremath{\dfrac{\partial^2 }{\partial #1^2}}}


%..ENUMERATION
    \newenvironment{abc}{\begin{enumerate}[(a)]}{\end{enumerate}}
    \newenvironment{cabc}{\begin{compactenum}[(a)]}{\end{compactenum}}
    \newenvironment{romanenum}{\begin{enumerate}[i.]}{\end{enumerate}}
    \newenvironment{cromanenum}{\begin{compactenum}[i.]}{\end{compactenum}}

%..FUNCTIONS
    \DeclareMathOperator{\arsinh}{arsinh}
    \DeclareMathOperator{\arcosh}{arcosh}
    \DeclareMathOperator{\artanh}{artanh}
    \DeclareMathOperator{\arcoth}{arcoth}
    \DeclareMathOperator{\arccot}{arccot}
    \DeclareMathOperator{\Arg}{Arg}
    \DeclareMathOperator{\Log}{Log}
    \newcommand{\dis}[1]{\hspace{#1cm}}
    \newcommand{\abs}[1]{\ensuremath{\left\vert#1\right\vert}}
    \newcommand{\attention}{\raisebox{-1pt}{{\makebox[1.6em][c]{\makebox[0pt][c]{\raisebox{.13em}{\small!}}\makebox[0pt][c]{\color{red}\Large$\bigtriangleup$}}}}}
    \DeclareMathOperator{\meq}{\stackrel{!}{=}}
    
    
% section color box
\setkomafont{section}{\mysection}
\newcommand{\mysection}[1]{%
    \Large\sffamily\bfseries%
    \setlength{\fboxsep}{0cm}%already boxed
    \colorbox{\StyleColor!40}{%
        \begin{minipage}{\linewidth}%
            \vspace*{2pt}%Space before
            #1
            \vspace*{-1pt}%Space after
        \end{minipage}%
    }}

%subsection color box
\setkomafont{subsection}{\mysubsection}
\newcommand{\mysubsection}[1]{%
    \normalsize \sffamily\bfseries%
    \setlength{\fboxsep}{0cm}%already boxed
    \colorbox{\StyleColor!20}{%
        \begin{minipage}{\linewidth}%
            \vspace*{2pt}%Space before
             #1
            \vspace*{-1pt}%Space after
        \end{minipage}%
    }}

%subsubsection color box
\setkomafont{subsubsection}{\mysubsubsection}
\newcommand{\mysubsubsection}[1]{%
	\normalsize \sffamily\bfseries%
	\setlength{\fboxsep}{0cm}%already boxed
	\colorbox{\StyleColor!10}{%
		\begin{minipage}{\linewidth}%
			\vspace*{2pt}%Space before
			#1
			\vspace*{-1pt}%Space after
		\end{minipage}%
}}

% highlighter
\newcommand{\hilight}[1]{\colorbox{\StyleColor}{#1}}
\newcommand{\highlighty}[1]{%
  \setlength{\fboxsep}{0pt}\colorbox{yellow!100}{\ensuremath{#1}}}

\newcommand{\highlightg}[1]{%
  \setlength{\fboxsep}{0pt}\colorbox{green!100}{\ensuremath{#1}}}

\newcommand{\highlightbg}[1]{%
   \colorbox{green!100}{$\displaystyle #1$}}  

% equation box        
\newcommand{\eqbox}[1]{\setlength{\fboxrule}{1mm}\fcolorbox{\StyleColor}{white}{\hspace{0.5em}$\displaystyle#1$\hspace{0.5em}}}

%center equationbox
\newcommand{\ceqbox}[1]{\vspace*{4pt} \begin{center}\eqbox{#1}\end{center}\vspace*{4pt}}


% \bibliography{semiconductordevices}
% \bibliographystyle{ieeetr}

%change page style for header
\pagestyle{fancy}
\footskip 20pt

% Uncomment this line to make formulasheet ultra compact
% This removes
% - list of variables
% \newcommand{\makeultracompact}{irrelevant}
\let\makeultracompact\undefined

% Make stuff ultra compact if so desired
\ifdefined\makeultracompact
  \setlength{\parskip}{0pt}
  \setlength{\abovedisplayskip}{0pt}
  \setlength{\belowdisplayskip}{0pt}
  \setlength{\abovedisplayshortskip}{0pt}
  \setlength{\belowdisplayshortskip}{0pt}
\else
\fi
 
% -----------------------------------------------------------------------
\IfFileExists{../build/revision.tex}{
  \input{../build/revision.tex}
  \rhead{Compiled: \compiledate \hspace{1em} on: \hostname \hspace{1em} git-sha: \revision \hspace{1em} Noah Huetter}
}{\rhead{Noah Huetter}}

\ifdefined\makeultracompact
  \lhead{ETH Communication Systems 2019 \hspace{1em}compact version}
\else
  \lhead{ETH Communication Systems 2019}
\fi
\chead{\thepage}
\cfoot{}
\headheight 17pt \headsep 10pt
\title{ETH Communication Systems 2019}
\author{Noah Huetter}

\date{\today}
\begin{document}

\setcounter{page}{0}
\setcounter{secnumdepth}{2} %no enumeration of sections
\begin{multicols*}{4}
	\section*{Disclaimer}
	This summary is part of the lecture ``ETH Communication Systems'' (227-0121-00) by Prof. Dr. Armin Wittneben (FS19). It is based on the lecture. \\[6pt]
	Please report errors to \href{mailto:huettern@student.ethz.ch}{huettern@student.ethz.ch} such that others can benefit as well.\\[6pt]	
  The upstream repository can be found at \href{https://github.com/noah95/formulasheets}{https://github.com/noah95/formulasheets}
	\vfill\null
  \columnbreak
  %%%%%%%%%%%%%%%%%%%%%%%%%%%%%
  \tableofcontents
  \vfill\null
  %\columnbreak
  %%%%%%%%%%%%%%%%%%%%%%%%%%%%%
	\pagebreak
  \maketitle 
  \setcounter{page}{1}
  \thispagestyle{fancy}

  % ---------------------------------------------------------------------------
  \section{Random Processes}
  % ---------------------------------------------------------------------------
  % What is
  \cgraphic{0.8}{img/rp.png}
  A random process $X(t)$:
  \begin{itemize}
    \item is a sample space composed of (real valued) time functions: 
      $\{x_1(t), x_2(t), \dots, x_n(t)\}$
    \item observed at a fixed $t_k$ is a random variable 
      $X(t_k) = \{x_1(t_k), x_2(t_k), \dots, x_n(t_k)\}$
    \item The time function $x_s(t)$ is a \textbf{realization} (sample function)
    \item $x_s(t_k)$ observed at $t_k$ is a real number
    \item A stochastic process consists of infinitely many random variables, 
      one for each $t_k$, with the CDF $F_{\{X(t_k)\}}(x) = P(X(t_k)\leq x)$
  \end{itemize}


  \subsection{Stationary processes}
  A process is \textbf{Strict Sense Stationary (SSS)} if:
  \begin{itemize}
    \item $X(t)$ and $X(t+\tau)$ have same satistics $\forall \tau$
    \item The joint distribution function of a set of r.v. observed at times
      $t_1,\dots,t_n$ is invariant to a time-shift.
  \end{itemize}
  \begin{empheq}[box=\eqbox]{equation*}
    \begin{gathered}
      \forall n,\tau,t_1,\dots,t_n: \\
      F_{\{X(t_1+\tau),X(t_2+\tau),\dots,X(t_n+\tau)\}}(x_1,x_2,\dots,x_n) = \\
        F_{\{X(t_1),X(t_2),\dots,X(t_n)\}}(x_1,x_2,\dots,x_n)
    \end{gathered}
  \end{empheq}

  Properties:
  \begin{empheq}[box=\eqbox]{equation*}
    \begin{gathered}
      \forall t_k : \mu_X(t_k) = \mu_X \\
      \forall t_1, t_2 : R_X(t_1, t_2) = R_X(t_2-t_1) = R_X(\tau) \\
      C_X(t_1, t_2) = \E[(X(t_1)-\mu_X)(X(t_2)-\mu_X)] \\ = R_X(t_2-t_1) - \mu_X^2 \\
    \end{gathered}
  \end{empheq}
  
  A process is \textbf{Wide Sense Stationary (WSS)} if a r.p. has a \textit{constant} mean and the autocorrelation depends only on the \textit{time difference}.
  \begin{empheq}[box=\eqbox]{equation*}
    \begin{gathered}
      \forall t : \mu_X(t) = \mu_X \\
      \forall t_1, t_2 : R_X(t_1, t_2) = R_X(t_2-t_1) = R_X(\tau)
    \end{gathered}
  \end{empheq}

  Strict sense stationary $\implies$ wide sense stationary.


  \subsection{Mean and correlation}
  Defined as expectation of r.v. $X(t_k)$ by observing process at time $t_k$.
  \begin{empheq}[box=\eqbox]{equation*}
    \begin{gathered}
      \mu_X(t_k) = \E[X(t_k)] = \intinf x f_{\{X(t_k)\}}(x) \dx \\
    \end{gathered}
  \end{empheq}

  Autocorrelation function $R_X$ and autovariance function $C_X$ of a random process:
  \begin{empheq}[box=\eqbox]{equation*}
    \begin{gathered}
      R_{X}(t_{1},t_{2}) = \E[X(t_{1})X(t_{2})] \triangleq \\
      \intinf\intinf x_{1}x_{2}f_{X_{1},X_{2}}(x_{1}, x_{2})\dx_{1}\dx_{2}\\
      R_{XY}(x,y) = \int_{-\infty}^{\infty}\int_{-\infty}^{\infty}xyf_{X,Y}(x, y)dxdy\\
      C_{X}(t_{1},t_{2}) = R_{X}(t_{2}-t_{1}) - m_{X}^{2}
    \end{gathered}
  \end{empheq}

  \begin{itemize}
    \item The mean and autocorrelation function determine the autocovariance function
    \item The mean and autocorrelation function only describe the first two moments of the process
  \end{itemize}

  Properties of the autocorrelation function:
  \begin{empheq}[box=\eqbox]{equation*}
    \begin{align*}
      \E[X^2(t)] &= R_X(0) & R_X(\tau) &= R_X(-\tau) \\
      \abs{R_X(\tau)} &\leq R_X(0)
    \end{align*}
  \end{empheq}
  \cgraphic{0.8}{img/autocorrelation.png}
  
  The Cross-correlation function $R_{XY}(t,u)$ of two random processes:
  \begin{empheq}[box=\eqbox]{equation*}
    \begin{gathered}
      R_{XY}(t,u) = \E[X(t)Y(u)] = \\
      \intinf xy\cdot f_{X,Y}(x, y) \dx \dy\\
    \end{gathered}
  \end{empheq}
  \begin{itemize}
    \item Stationariy menas $R_{XY}(t,u) = R_{XY}(\tau)$ for $\tau=t-u$
    \item Not generally an even function of $t$
    \item Not necessarily a maximum at $\tau=0$
    \item Symmetry: $R_{XY}(\tau) = R_{XY}(-\tau)$
  \end{itemize}

  \subsection{Ergodicity}
  Definition: A random process is \textit{ergodic} in the mean if
  \begin{itemize}
    \item Time average approaches ensemble averages for increasing $T$
    \item The variance of the time average approaches zero for incr. $T$
  \end{itemize}
  \begin{empheq}[box=\eqbox]{equation*}
    \begin{align*}
      \lim_{T\to\infty} \mu_X(T) &= \mu_X & \lim_{T\to\infty} \Var[\mu_X(T)] &= 0
    \end{align*}
  \end{empheq}

  Or in other words: The same behavior averaged over time as averaged over the space of all the system's states.

  \subsection{Filtered processes}
  Stationary random process $X(t)$ is input to a linear timeinvariant (LTI) filter with impulse response $h(t)$.
  \begin{empheq}[box=\eqbox]{equation*}
    \begin{gathered}
      Y(t) = \intinf h(\tau_1)X(t-\tau_1)\dtau_1 \\ S_Y(f) = \abs{H(f)}^2S_X(f)
    \end{gathered}
  \end{empheq}

  Find mean and autocorrelation of $Y(t)$:
  \begin{empheq}{gather*}
      \mu_X = \E[X(t)] \quad R_X(\tau) = \E[X(t)X(t-\tau)] \\
      \mu_Y = \E[Y(t)] = \E\left[\intinf h(\tau_1)X(t-\tau_1)\dtau_1\right]
  \end{empheq}

  Can interchange expectation and integration if stable $\intinf\abs{h(t)}\dt<\infty$ and finite mean $\mu_X<\infty$
  \begin{empheq}[box=\eqbox]{gather*}
      \mu_Y = \intinf h(\tau_1)\E[X(t-\tau_1)]\dtau_1 = \mu_X \intinf h(\tau_1) \dtau_1
  \end{empheq}

  Autocorrelation:
  \begin{empheq}{gather*}
      R_Y(t,u) = \E[Y(t)Y(u)] = \\ \E\left[\intinf h(\tau_1)X(t-\tau_1)\dtau_1
      \intinf h(\tau_2)X(u-\tau_2)\dtau_2\right] \\
  \end{empheq}

  Additional condition for interchange is finite mean-square value: $R_X(0) = \E[X^2(t)]<\infty$
  \begin{empheq}[box=\eqbox]{gather*}
      R_Y(\tau) = \intinf\intinf h(\tau_1)h(\tau_2)R_X(\tau-\tau_1+\tau_2)\dtau_1\dtau_2
  \end{empheq}

  (WS) stationary input process $X(t)$ to a stable LTI filter $\implies$ (WS) stationary output process $Y(t)$.

  \subsection{Power spectral density}
  \begin{empheq}[box=\eqbox]{equation*}
    S_{X}(f) = \mathscr{F}[R_{X}(\tau)](f) = \intinf R_{X}(\tau)e^{-j2\pi f\tau}\dtau 
        \end{empheq}
  \setlength{\leftmargini}{0.5cm} 
  \begin{itemize}
    \item $S_{X}(0) = \int_{-\infty}^{\infty} R_{X}(\tau)d\tau $
    \item $E[X^{2}(t)] = R_{X}(0) =  \int_{-\infty}^{\infty} S_{X}(f)df $
    \item $S_{X}(f) \ge 0$ $\forall f$
    \item $S_{X}(f)  = S_{X}(-f)$ $\forall f$, iff $X(t) \in \mathbb{R}$
  \end{itemize}
  \setlength{\leftmargini}{0pt} 

  \subsection{Gaussian process}
  Consider the r.v. $Y=\int_0^Tg(t)X(t)\dt$ where $g(t)$ is in an arbitraty function. If $Y$ is gaussian distributed, then the process $X(t)$  is a \textit{Gaussian process}
  \begin{itemize}
    \item A filtered Gaussian process remains a Gaussian process
    \item If $X(t)$ is a GP, the arbitrary set of r.v. $\vec{X} = [X(t_1),\dots,
    X(t_n)]^T$ is jointly gaussian distributed for any $n$
    \item The joint cdf is of these r.v. is completely determined by the \textbf{means} $\mu_X(t_i) = \E[X(t_i)]$ and \textbf{covariances} $C_X(t_k,t_i)=\E[(X(t_k)-\mu_X(t_k))(X(t_i)-\mu_X(t_i))]$
  \end{itemize}

  Multivariative Guass distribution:
  \begin{empheq}[box=\eqbox]{equation*}
    \begin{gathered}
      f(x) = \frac{\exp\left(-\frac{1}{2} (\vec{x}-\vec{m}_{x})^{T}\underline{\Sigma}^{-1} (\vec{x}-\vec{m}_{x}) \right)}{(2\pi)^{\frac{n}{2}}\det(\underline{\Sigma})^{\frac{1}{2}}}\\
      \underline{\Sigma} := 
      \begin{bmatrix} 
        \Cov(X_{1},X_{1})&...&\Cov(X_{1},X_{n})\\
        \vdotswithin{\ldots} & \vdotswithin{\ldots} & \vdotswithin{\ldots}\\
        \Cov(X_{n},X_{1})&...&\Cov(X_{n},X_{n})
      \end{bmatrix}
          \end{gathered}
  \end{empheq}

  \subsection{Noise}
  White noise is defined by its autocorrelation.
  \begin{empheq}[box=\eqbox]{align*}
      R_W(\tau) &= \frac{N_0}{2}\delta(t) & S_W(f) =  \frac{N_0}{2}
  \end{empheq}


\end{multicols*}

\setcounter{secnumdepth}{2}
\end{document}
