

% Summary semiconductor devices D-ITET
% ===========================================================================
% @Author: Noah Huetter
% @Date:   2019-02-20 17:26:28
% @Last Modified by:   noah
% @Last Modified time: 2019-02-20 17:30:46
% ---------------------------------------------------------------------------

\documentclass[a4paper, fontsize=8pt, landscape, DIV=1]{scrartcl}
\usepackage{lastpage}
\usepackage{hyperref}
% Include general settings and customized commands
%
% General packages and settings
% ===========================================================================
% Author:			Silvano Cortesi (cortesis@student.ethz.ch)
% Version:			1.2
% Last changed:		03.01.2018
%
% ---------------------------------------------------------------------------




\usepackage[german,british]{babel} %choose your language \usepackage[german]{babel}
%\usepackage[T1]{fontenc}
\usepackage[utf8]{inputenc}
\usepackage{fancyhdr}
%\usepackage{lastpage}
%\usepackage{lmodern}
%\usepackage{enumerate}
%\usepackage{float} % for positioning of figures
\usepackage[landscape, margin=1cm]{geometry}
\usepackage[dvipsnames]{xcolor}
\usepackage{pdfpages}


%% Math %%
\usepackage{amscd}
\usepackage{blindtext}
\usepackage{enumitem}
\usepackage{multicol}
\usepackage{parskip}
\usepackage{empheq}
\usepackage{amsmath}
\usepackage{amsfonts}
\usepackage{amssymb}
\usepackage{amsthm}
%\usepackage{dsfont}
%\usepackage{esint} % provides \oiint
\usepackage{mathrsfs}
%\usepackage{trfsigns}
%\numberwithin{equation}{subsection}
%\usepackage{numprint}

%% Graphics & Charts %%
\usepackage{graphicx}
%\usepackage{pdfpages}
%\usepackage{booktabs}
%\usepackage{array}
%\usepackage{paralist}
%\usepackage{framed}
%\usepackage{trfsigns}
\usepackage{tikz}
%\usepackage[lofdepth,lotdepth]{subfig}
%\usepackage{tikz}  %Graphen zeichnen
%\usetikzlibrary{decorations.pathmorphing}
%\usetikzlibrary{arrows.meta,arrows}
%\usepackage{pgfplots}
%% General Settings %%
%\setlength{\parindent}{0px}
%\setkomafont{captionlabel}{\normalfont\bfseries}

%\pagestyle{fancy}
%\lfoot{\tiny \today}
%\rfoot{\thepage\  / \pageref{LastPage}}
%\cfoot{}
%\renewcommand{\footrulewidth}{0.4pt}

%% provides command \uline{} for underlining words
%\usepackage{ulem}

%% colour headings
%\usepackage{color}
%\definecolor{bluen}{cmyk}{1,0.5,0,0}
%\definecolor{bloodorange}{cmyk}{0,.92,1,.2}
%\addtokomafont{section}{\color{bloodorange}}
%\addtokomafont{subsection}{\color{bloodorange}}
%\addtokomafont{subsubsection}{\color{bloodorange}}
%\addtokomafont{paragraph}{\small\color{bloodorange}}
%\addtokomafont{subparagraph}{\small\color{bloodorange}}

%% Signs & Special Formating %%
%\usepackage{ulem} %normalem: \emph{Text} is italic again.
%\usepackage{multicol,multirow}
%\usepackage{tabularx}
%\usepackage{stackrel}
%\usepackage{makeidx}
%\usepackage{mparhack} % bessere margiale bei seitenumbruch

% make document compact
\usepackage[compact]{titlesec}
\titlespacing{\section}{0pt}{*1}{*1}
\titlespacing{\subsection}{0pt}{*1}{*1}
\titlespacing{\subsubsection}{0pt}{*1}{*1}

\parindent 0pt
\pagestyle{empty}
\setlength{\unitlength}{1cm}
\setlist{leftmargin = *}

%include also newer PDF
\pdfminorversion=6

% Set the color of your style
% Avaiable are: Apricot, Aquamarine, Bittersweet, Black, Blue, blue, BlueGreen, BlueViolet, BrickRed, Brown, BurntOrange, CadetBlue, CarnationPink, Cerulean, CornflowerBlue, Cyan, Dandelion, DarkOrchid, Emerald, ForestGreen, Fuchsia, Goldenrod, Gray, Green, GreenYellow, JungleGreen, Lavender, ... (more at: http://en.wikibooks.org/wiki/LaTeX/Colors)
\def\StyleColor{BrickRed}

%
% General commands
% ===========================================================================
% Author:			Silvano Cortesi (cortesis@student.ethz.ch)
% Version:			1.2
% Last changed:		03.01.2018
%
% ---------------------------------------------------------------------------

%..ROEMISCHE_ZAHLEN
	\newcommand{\Roe}[1]{\uppercase\expandafter{\romannumeral #1 }}

%..ZAHLENMENGEN
	\newcommand{\N}{\mathbb{N}}
	\newcommand{\Z}{\mathbb{Z}}
	\newcommand{\Q}{\mathbb{Q}}
	\newcommand{\R}{\mathbb{R}}
	\newcommand{\real}{\R}
	\newcommand{\C}{\mathbb{C}}
	\newcommand{\complex}{\C}
	\newcommand{\0}{\mathbb{O}}
	\newcommand{\F}{\mathbb{F}}
	\newcommand{\K}{\mathbb{K}}
    \newcommand{\angstrom}{\textup{\AA}}
    
%..PFEILE
	\renewcommand{\leadsto}{\Longrightarrow}
	\newcommand{\leftrightleadsto}{\Longleftrightarrow}

%..VEKTOREN
	\newcommand{\Ul} {\underline}
	\newcommand{\vEx} {\vec{e}_x}
	\newcommand{\vEy} {\vec{e}_y}
	\newcommand{\vEz} {\vec{e}_z}
	\newcommand{\vEq} {\vec{e_1}}
	\newcommand{\vEw} {\vec{e_2}}
	\newcommand{\vEe} {\vec{e_3}}
	\newcommand{\transpose} {^{\text{T}}}
	\newcommand{\vect}[1]{\boldsymbol{#1}}
	
%..MATRIX
    \newcommand{\MATR}[1]{ \displaystyle \left( \begin{matrix} #1 \end{matrix} \right)}
    \newcommand{\MATRABS}[1]{ \displaystyle \left| \begin{matrix} #1 \end{matrix} \right|}


%..KOMPLEXE ZAHLEN
	\renewcommand{\Re}{\text{Re}\,}
	\renewcommand{\Im}{\text{Im}\,}

%..OPERATOREN
	\DeclareMathOperator{\grad}{grad}
	\renewcommand{\div}{\text{div}\,}
    	\DeclareMathOperator{\rot}{rot}
    	\DeclareMathOperator{\divg}{div}
    	\DeclareMathOperator{\Tr}{Tr}
    	\DeclareMathOperator{\const}{const}
	\DeclareMathOperator{\imag}{i}
	\newcommand{\Lapl}{\hbox{\footnotesize{$\Delta$}}}

%..DIFFERENTIALRECHNUNG
	\newcommand{\Dx} {\,\mathrm{d}}
	\newcommand{\abl}[1] {\frac{\mathrm{d}}{\mathrm{d}#1}}
	\newcommand{\Abl}[2] {\frac{\mathrm{d}#1}{\mathrm{d}#2}}
	\newcommand{\ablq}[1] {\frac{\mathrm{d^2}}{\mathrm{d}#1^2}}
	\newcommand{\Ablq}[2] {\frac{\mathrm{d^2}#1}{\mathrm{d}#2^2}}
	\newcommand{\pabl}[1] {\frac{\partial}{\partial#1}}
	\newcommand{\pablq}[1] {\frac{\partial^2}{\partial#1^2}}
	\newcommand{\Pabl}[2] {\frac{\partial#1}{\partial#2}}
	\newcommand{\Pablq}[2] {\frac{\partial^2#1}{\partial#2^2}}

%..INTEGRALRECHNUNG
	\newcommand{\dint}{\displaystyle{\int}}
	\newcommand{\intab}{\int^b_a}
	\newcommand{\intinf}{\int_{-\infty}^\infty}
	\newcommand{\dintab}{\displaystyle{\int^b_a}}
	\newcommand{\dintpi}{\displaystyle{\int^{\pi}_{-\pi}}}
	\newcommand{\dintzpi}{\displaystyle{\int^{2\pi}_{\mbox{-}2\pi}}}
	\newcommand{\dA}{\hspace{4pt}\mathrm{d}A}
	\newcommand{\dx}{\hspace{4pt}\mathrm{d}x}
	\newcommand{\dy}{\hspace{4pt}\mathrm{d}y}
	\newcommand{\dz}{\hspace{4pt}\mathrm{d}z}
	\newcommand{\dr}{\hspace{4pt}\mathrm{d}r}
	\newcommand{\ds}{\hspace{4pt}\mathrm{d}s}
	\newcommand{\dS}{\hspace{4pt}\mathrm{d}S}
	\newcommand{\dt}{\hspace{4pt}\mathrm{d}t}
	\newcommand{\dm}{\hspace{4pt}\mathrm{d}m}
	\newcommand{\dk}{\hspace{4pt}\mathrm{d}k}
	\newcommand{\dl}{\hspace{4pt}\mathrm{d}l}
	\newcommand{\du}{\hspace{4pt}\mathrm{d}u}
	\newcommand{\dv}{\hspace{4pt}\mathrm{d}v}
	\newcommand{\dV}{\hspace{4pt}\mathrm{d}V}
	\newcommand{\dphi}{\hspace{4pt}\mathrm{d}\varphi}
	\newcommand{\domega}{\hspace{4pt}\mathrm{d}\omega}
	\newcommand{\dvarsigma}{\hspace{4pt}\mathrm{d}\varsigma}
	\newcommand{\dtau}{\hspace{4pt}\mathrm{d}\tau}
	\newcommand{\dtheta}{\hspace{4pt}\mathrm{d}\vartheta}
	\newcommand{\dmu}{\hspace{4pt}\mathrm{d}\mu}
	\newcommand{\dxi}{\hspace{4pt}\mathrm{d}\xi}
	\newcommand{\deta}{\hspace{4pt}\mathrm{d}\eta}
	\newcommand{\dvecl}{\hspace{4pt}\mathrm{d}\vec{l}}
	\newcommand{\dvecS}{\hspace{4pt}\mathrm{d}\vec{S}}

%..LIMES
    \DeclareMathOperator{\limni}{\lim\limits_{n\to\infty}}
    \DeclareMathOperator{\limxi}{\lim\limits_{x\to\infty}}
    \DeclareMathOperator{\limho}{\lim\limits_{h\to0}}
    \newcommand{\limxai}[1]{\ensuremath{\lim\limits_{x\to #1}}}

%..SUMMEN
    \DeclareMathOperator{\sumni}{\sum_{n=0}^{\infty}}
    \newcommand{\sumnia}[1]{\ensuremath{\sum_{n=#1}^{\infty}}}


%..PARTIELLE ABLEITUNG
    \DeclareMathOperator{\partf}{\dfrac{\partial f}{\partial x}}
    \newcommand{\partfo}[1]{\ensuremath{\dfrac{\partial f}{\partial #1}}}
    \newcommand{\parto}[1]{\ensuremath{\dfrac{\partial }{\partial #1}}}
    \newcommand{\partt}[2]{\ensuremath{\dfrac{\partial^2 }{\partial #1\partial #2}}}
    \newcommand{\partq}[1]{\ensuremath{\dfrac{\partial^2 }{\partial #1^2}}}


%..ENUMERATION
    \newenvironment{abc}{\begin{enumerate}[(a)]}{\end{enumerate}}
    \newenvironment{cabc}{\begin{compactenum}[(a)]}{\end{compactenum}}
    \newenvironment{romanenum}{\begin{enumerate}[i.]}{\end{enumerate}}
    \newenvironment{cromanenum}{\begin{compactenum}[i.]}{\end{compactenum}}

%..FUNCTIONS
    \DeclareMathOperator{\arsinh}{arsinh}
    \DeclareMathOperator{\arcosh}{arcosh}
    \DeclareMathOperator{\artanh}{artanh}
    \DeclareMathOperator{\arcoth}{arcoth}
    \DeclareMathOperator{\arccot}{arccot}
    \DeclareMathOperator{\Arg}{Arg}
    \DeclareMathOperator{\Log}{Log}
    \newcommand{\dis}[1]{\hspace{#1cm}}
    \newcommand{\abs}[1]{\ensuremath{\left\vert#1\right\vert}}
    \newcommand{\attention}{\raisebox{-1pt}{{\makebox[1.6em][c]{\makebox[0pt][c]{\raisebox{.13em}{\small!}}\makebox[0pt][c]{\color{red}\Large$\bigtriangleup$}}}}}
    \DeclareMathOperator{\meq}{\stackrel{!}{=}}
    
    
% section color box
\setkomafont{section}{\mysection}
\newcommand{\mysection}[1]{%
    \Large\sffamily\bfseries%
    \setlength{\fboxsep}{0cm}%already boxed
    \colorbox{\StyleColor!40}{%
        \begin{minipage}{\linewidth}%
            \vspace*{2pt}%Space before
            #1
            \vspace*{-1pt}%Space after
        \end{minipage}%
    }}

%subsection color box
\setkomafont{subsection}{\mysubsection}
\newcommand{\mysubsection}[1]{%
    \normalsize \sffamily\bfseries%
    \setlength{\fboxsep}{0cm}%already boxed
    \colorbox{\StyleColor!20}{%
        \begin{minipage}{\linewidth}%
            \vspace*{2pt}%Space before
             #1
            \vspace*{-1pt}%Space after
        \end{minipage}%
    }}

%subsubsection color box
\setkomafont{subsubsection}{\mysubsubsection}
\newcommand{\mysubsubsection}[1]{%
	\normalsize \sffamily\bfseries%
	\setlength{\fboxsep}{0cm}%already boxed
	\colorbox{\StyleColor!10}{%
		\begin{minipage}{\linewidth}%
			\vspace*{2pt}%Space before
			#1
			\vspace*{-1pt}%Space after
		\end{minipage}%
}}

% highlighter
\newcommand{\hilight}[1]{\colorbox{\StyleColor}{#1}}
\newcommand{\highlighty}[1]{%
  \setlength{\fboxsep}{0pt}\colorbox{yellow!100}{\ensuremath{#1}}}

\newcommand{\highlightg}[1]{%
  \setlength{\fboxsep}{0pt}\colorbox{green!100}{\ensuremath{#1}}}

\newcommand{\highlightbg}[1]{%
   \colorbox{green!100}{$\displaystyle #1$}}  

% equation box        
\newcommand{\eqbox}[1]{\setlength{\fboxrule}{1mm}\fcolorbox{\StyleColor}{white}{\hspace{0.5em}$\displaystyle#1$\hspace{0.5em}}}

%center equationbox
\newcommand{\ceqbox}[1]{\vspace*{4pt} \begin{center}\eqbox{#1}\end{center}\vspace*{4pt}}

%change page style for header
\pagestyle{fancy}
\footskip 20pt

\rhead{Noah Huetter}
\lhead{ETH Semiconductor Devices}
\chead{\thepage}
\cfoot{}
\headheight 17pt \headsep 10pt
\title{ETH Semiconductor Devices}
\author{Noah Huetter}

\date{\today}
\begin{document}

\setcounter{secnumdepth}{2} %no enumeration of sections
\begin{multicols*}{4}
	\section*{Disclaimer}
	This summary is part of the lecture ``ETH Semiconductor Devices'' by Prof. Dr. Colombo Bolognesi (FS19). It is based on the lecture. \\[6pt]
	Please report errors to \href{mailto:huettern@student.ethz.ch}{huettern@student.ethz.ch} such that others can benefit as well.\\			
	\vfill\null
	\pagebreak
  \maketitle 
  \thispagestyle{fancy}

  % ---------------------------------------------------------------------------
  \section{Allgemein}        
  % ---------------------------------------------------------------------------


\end{multicols*}

%     \maketitle 
%       \thispagestyle{fancy}
%       % ---------------------------------------------------------------------------
%       % ---------------------------------------------------------------------------
%       \section{Allgemein}        
%       % ---------------------------------------------------------------------------
% 	\subsection{Definition}
%       		Allgemeine Form einer partiellen Differentialgleichung (PDG) für eine Funktion 
%       		$u(x_1, x_2, .., x_n)$ ist: 
%       		\[F(x_1, x_2, ..,x_n, u, u_{x1}, u_{x2}, .. , u_{xn}, u_{x1x1}, ..) = 0\] 
%       		wobei $u$ die gesuchte Funktion ist und $u_{x1}=\frac{\partial u}{\partial x},\dots$ dessen partielle Ableitungen sind. 
%       	\subsection{Wohlgestelle PDG}
% 			Eine PDG heisst wohl gestellt (well posed) falls,
% 			\begin{itemize}
% 				\item \textbf{Existence and Uniqueness}: es genau eine Funktion gibt, welche die PDG löst und auch die Randbedingungen erfüllt.\\
% 				\textbf{Beweisen:} Einfach PDG lösen!
% 				\item \textbf{Stability}: eine kleine Änderung an der PDG oder an den Randbedingungen nur eine kleine Änderung der Lösung bewirkt.
% 			\end{itemize}
%       	\subsection{Klassifizierung}
%       		\subsubsection{Ordnung einer PDG:}
%       			Die Ordnung einer PDG entspricht der Ordnung der höchsten vorkommenden Ableitung.
%       		\subsubsection{Linearität:}
%       		    \attention Dürfen von $x,y,x^2,\dots$ abhängen!\\
%       		    Eine PDG heisst linear, falls sie linear in $u$ und deren partiellen Ableitungen ist.\\
%       			Es gibt Subgruppen der nichtlinearen PDG:
% 				\begin{itemize}
% 					\item \textbf{Semilinear:} In der PDG enthaltene Ableitungen sind linear.
% 					    \begin{multline*}
% 					    a(x,y,u)\frac{\partial u}{\partial x} + b(x,y,u)\frac{\partial u}{\partial y} \\ + c(x,y)\frac{\partial^3 u}{\partial x^2 \partial y}= d(x,y,u)
%                       	\end{multline*}
% 					\item \textbf{Quasilinear:} In der PDG enthaltene Ableitungen höchster Ordnung sind linear.
% 					    \begin{multline*}
% 						a(x,y,u,u_{x})\frac{\partial u}{\partial x} + b(x,y,u,u_{y})\frac{\partial u}{\partial y} \\ + c(x,y,u,u_x,u_y)\frac{\partial^2 u}{\partial x^2} = d(x,y,u)
% 				        \end{multline*}

% 				\end{itemize}
%       		\subsubsection{Homogenität:}
%       		    Eine PDG heisst homogen, falls in der Gleichung neben der Funktion $u$ keine anderen Funktionen abhängig von den Variablen $x_1,\dots,x_n$ (auch keine Konstanten) vorkommen.
      			
      	
%       	\subsection{Linearer Differentialoperator}
%       		Eine lineare PDG definiert in natürlicher Weise einen linearen Differentialoperator
%       		\[L[u] = f\]
%       	    welcher folgende Relation erfüllt:
%       	    \[L[a_1u_1+a_2u_2]=a_1L[u_1]+a_2L[u_2]\]
%       		Eine PDG $L[u] = 0$ nennt man homogene, lineare PDG, während dessen $L[u]=x^2$ eine nicht-homogene, lineare PDG ist. 
      	
      	
%       \section{Superpositionsprinzip}
%       	Ist die Funktion $u_i$ eine Lösung der lin. PDG \\
%       	$L[u_i] = f_i $, \\
%       	dann ist die Linearkombination \\
%       	$v := \sum_{i=1}^n \alpha_i u_i $ \\
%       	ebenfalls eine Lösung der PDG.
              
      
%       \section{PDG erster Ordnung}
%       	PDG hat die Form: $F(x,y,u_x,u_y) = 0$ \\
%       	Idee: Graph von u ist eine Fläche in $\mathbb{R}^3$ mit Normale $(u_x,u_y,-1)$ \\[2pt]
%       	\textbf{Beweis}: Konstruiere allgemeine Kurve $\gamma(s)$ und betrachte deren Ableitung.
      	
      	
%       	\subsection{Methode der Charakteristiken}
%       		Cauchyproblem (quasilinear): \\[5pt]
%       		\textbf{PDG:} $a(x,y,u) u_x + b(x,y,u) u_y = \textcolor{red}{c(x,y,u)}$ \\[2pt]
%       		\textbf{AB:} $u\big(x(s), y(s)\big)=F\big(x(s), y(s)\big),\\[2pt] \hspace*{20pt} s\in(\alpha,\beta)$ (Startbedingung)
%       		\\[6pt]
%       		Die PDG kann auch als Skalarprodukt geschrieben werden: \\
%       		$\underbrace{(a,b,c)}_{\text{Tangentialebene}} \cdot \underbrace{(u_x,u_y,-1)}_{\text{Normale}} = 0$ \\
%       		\attention\textbf{Definitionsraum angeben!}
%       		\subsubsection{Kochrezept:}
%       		\begin{enumerate}[start=0, label=\textbf{\arabic*.}]
%       		    \item Finde $a(x,y,u),b(x,y,u),c(x,y,u)$
%       		    \item Schreibe Anfangsbedingung als Kurve abhängig von Parameter $s$:
%       		    \[\hspace*{-0.05\columnwidth}\Gamma = \Gamma(s)=\big(x_0(s),y_0(s),u_0(s)\big),\ \in I=(\alpha,\beta)\]
%       		    \item Jeder Punkt auf $\Gamma(s)$ ist Anfangspt. einer Charakteristik. Parametrisiere jede Charakteristik $t\to\big(x(t),y(t),u(t)\big)$ und löse (mit AB):
%       		    \begin{align*}
%       		        \frac{\partial x}{\partial t}(t,s)&=a(x,y,u) && x(0,s)=x_0(s)\\
%       		        \frac{\partial y}{\partial t}(t,s)&=b(x,y,u) && y(0,s)=y_0(s)\\
%       		        \frac{\partial u}{\partial t}(t,s)&=c(x,y,u) && u(0,s)=u_0(s)\\
%       		    \end{align*}
%       		    \textcolor{red}{Wenn verknüpft, parametrisiere:}\\$x=s\cdot\cos(t),\ y=s\cdot\sin(t)$
%       		   \item Finde $s(x,y),\ t(x,y)$ und setze in Lösung $u(t,s)$ ein. Löse nach $u(x,y)$ auf.\\[4pt]
%       		    \textbf{\attention NUR FALLS JACOBI $\neq0$\ (TB-Cd.)!}
%       		    \item Zeige Eindeutigkeit (an Stelle $(0,s)$ auswerten!)\\
%   		        $\det\Big(J\big(x(t,s),y(t,s)\big)\Big) \neq 0$
%       		\end{enumerate}
	   				  			  
%       	\subsection{Jacobi-Determinante}
%       		\[
%       			\det\Big(J\big(x(t,s),y(t,s)\big)\Big) = \MATRABS{\frac{d(x,y)}{d(t,s)}} = \MATRABS{\Pabl{x}{t} & \Pabl{x}{s} \\[4pt] \Pabl{y}{t} & \Pabl{y}{s} }
%       		\]
%       	Die Jacobi-Determinante muss ungleich null sein, so dass die Funktionen $x$ und $y$ invertierbar sind und 
%       	die Methode der Charakteristiken funktioniert. Folgt aus dem Satz der Umkehrabbildung.
      	
%       \subsection{Existenz- und Eindeutigkeitssatz}
%       	\textbf{Def:} Die Projektionen der charakteristischen Kurven in x,y-Ebene heissen Charakteristiken. \\[5pt]
%       	\textbf{Transversalitätsbedingung (TB)} in $\Gamma(s)$: \\
%       	Projektion $\Gamma'$ von $\Gamma$ und der charakt. Kurve (Charakteristik) schneiden sich transversal 
%       	(Tangenten verschieden in diesem Punkt). \\
%       	\\
%       	\textbf{Satz:} (erfüllt für $\det\Big(J\big(x(t,s),y(t,s)\big)\Big)|_{t=0}\neq0$) \\
%       	\[\MATRABS{a & b \\[4pt] (x_0)_s & (y_0)_s }\neq0\]
% 		\begin{enumerate}
% 			\item Ist die TB für $s \in [s_0, s_1] $ erfüllt, d.h. $J(s) \neq 0$, dann gibt es ein 
% 			$\varepsilon > 0$ und Funktionen $x(t,s), y(t,s), u(t,s)$ definiert für $(t,s) \in (-\varepsilon,
% 			 \varepsilon) \times (s_0, s_1)$ welche die eindeutige Lösung des Cauchy-Problems darstellen.
% 			\item Gilt $J(s) \equiv 0$ auf einem Intervall $[s_0,s_1]$, dann hat das Cauchy-Problem entweder
% 			 unendlich viele oder gar keine Lösungen. 
% 		\end{enumerate}
% 		\textbf{$\rightarrow$ eindeutige Lösung (lokal)}
% 		\\[6pt]
% 		\textbf{GP:} Globale Probleme sind Konflikte, welche auftreten können \textbf{selbst wenn} die TB erfüllt ist.
% 		\begin{itemize}
% 		    \item Charakteristiken schneiden die Anfangskurve mehr als einmal $\rightarrow$ keine Lsg.    % Stimmt das wirklich?
% 		    \item Charakteristiken schneiden einander in der $(x,y)$ Ebene.
% 		    \item Wenn eine der char. Gleich. nicht eindeutig lösbar ist, z.B. wenn $a$ nicht Lipschitz-stetig in $x$ ist.
% 		\end{itemize}
	
% 			\subsection{Erhaltungssatz, Stoss- und Schockwellen}
% 	    Eine PDG wird \textbf{Erhaltungssatz} (conservation law) genannt, wenn sie in folgender Form vorliegt:
% 	    \begin{center}
% 		    \eqbox{u_y + \frac{\partial}{\partial x}F(u) = 0}
% 	    \end{center}
%           \[\begin{cases}
%           u_y+\frac{\partial}{\partial x} F(u)=0\\
%           u(x,0)=u_0(s)=\begin{cases}
%           u^-, x\leqslant \alpha\\
%           u^+, x> \alpha
%           \end{cases}
%           \end{cases}\]
%           Mit der Kettenregel folgt: 
%           \[\begin{cases} 
%           u_y+F'(u)u_x = 0\\
%           u(x,0)=u_0
%           \end{cases}\]
%           \textbf{Entropy Bedingung}\\
%               f"ur $u_y + \frac{\partial}{\partial x}F(u) = 0$:\\
%               $F_u(u^-) > \gamma_y > F_u(u^+)$\\
%               f"ur $F(u) = 1/2 u^2:$ \\$u^->u^+$\\
%               Charakteristiken dürfen Stosswellen nicht verlassen, sondern nur auf diese auftreffen.
%           \subsubsection{1. Fall: Schockwelle: $u^- > u^+$}
%               Die Charakteristiken schneiden sich und es resultiert eine Schockwelle. Die Schockwelle $\gamma (t)$ bezeichnet die Kurve, entlang der $u(x,t)$ verschiedene Werte annimmt. Die Schockwelle ist bestimmt durch: 
%               \[u(x,y)=\begin{cases}
%               u^-, x < \gamma (y)\\
%               u^+, x>\gamma (y)
%               \end{cases}\] 
%               \[\gamma '(y) = \frac{F(u^+)-F(u^-)}{u^+-u^-}=\frac{[F]}{[u]}=\frac{\text{Sprung von F}}{\text{Sprung von U}}\]
%               \[\begin{cases}
%               \gamma(y) = \int \gamma '(t) dt \\
%               \gamma(0) = \alpha
%               \end{cases} \]
%               $\alpha$ ist die Projektion der Unstetigkeitsstelle der Stosswelle auf $y=0$.
%           \subsubsection{2. Fall: Verdünnungswelle: $u^- < u^+$}
%               Die Charakteristiken laufen auseinander. Die Funktion $u(x,t)$ nimmt im Bereich ohne Charakteristiken z.B. die Lösung der Verdünnungswelle an. Diese ist:
%               \begin{align*}
%               u(x,y)=\begin{cases}
%               u^-,\qquad &x\leqslant F'(u^-)t\\
%               (F')^{-1}(\frac{x}{t}),\quad &F'(u^-)t<x\leqslant F'(u^+)t\\
%               u^+,\qquad &x>F'(u^+)t
%               \end{cases}
%               \end{align*}
%           \subsubsection{Rankine-Hugoniot Bedingung}
%               Auf beiden Seiten der Schockkurve gilt die Lösung der jeweiligen Charakteristiken, die sich ausser in der Kurve nicht mehr schneiden werden. Formel für den Verlauf der Schockkurve für Erhaltungssätze:
%               \begin{align}\gamma_y(y)&=\gamma '(y)=\Pabl{\gamma}{y}= \frac{F(u^+)-F(u^-)}{u^+-u^-}\nonumber \\&=\frac{[F]}{[u]}=\frac{\text{Sprung von F}}{\text{Sprung von U}}\nonumber \end{align}
%               wobei: \\
%               $u^+(y) = \lim_{x \to \gamma_y(y)_+} u(x,y)$ \\ $u^-(y) = \lim_{x \to \gamma_y(y)_-} u(x,y)$
%           \subsubsection{Speziallfall: Burgersgleichung/Allgemeines zu Schocks}
%               Schocks treten auf, wenn sich die Charakteristieken schneiden. Das Problem der Lösungen ist, dass die Lösungen der Charakteristischen Gleichungen nicht immer stetig/stetig diffbare DGLs sind. Beispiel: Burgersgleichung
%               \ceqbox{F(u)=\frac{1}{2}u^2}
%               \ceqbox{\begin{cases} 
%               u_y+u u_x=0\\
%               u(x,0)=h(x)
%               \end{cases}}
%                \attention\textcolor{red}{$u$ ist unabhängig von $t$ $\Rightarrow u(x,t)=h(x)$}\\
%               Es entstehen sich schneidene Charakteristiken. Das zugehörige physikalische Problem lautet (in kanonischer Form): 
%               \[\Pabl {u}{y}+\frac{1}{2}\pabl{x} u^2=\Pabl {u}{y}+\pabl{x}F(u)=0\]
%               \textcolor{red}{Wird dies nach $x$ integriert findet man das schwache Problem (schwach=hat unstetige Lösungen).\\Es lautet}
%               \ceqbox{\pabl{y}\int_a^b u(\xi,y)d\xi+[F\big(u(b,y)\big)-F\big((a,y)\big)]=0} 
%               \textcolor{red}{(Für Beweis, Integral splitten mit neuer Grenze $\gamma$)}\\Durch lösen dieser Gleichung kommen wir auf die Resultate der Stoss- bzw. Verdünnungswelle. Wir lösen das Problem mit der Methode der Charakteristiken:\\
%               $\rightarrow$ Erhaltungssatz anwenden\\
%               	$\begin{array}{l}
% 	     	x_t = u\\
% 	     	y_t = 1\\
% 	     	u_t = 0
% 	    \end{array}
% 	   \Rightarrow 
% 	   \begin{array}{l}
% 	     	x(t,s) = h(s) t + s\\
% 	     	y(t,s) = t\\
% 	     	u(t,s) = h(s)
% 	     	u(x,y) = h(x-uy)
% 	    \end{array}$\\[4pt]
% 	    $\Rightarrow$
% 	    $u_x = \frac{h'}{1+yh'} \Rightarrow y_c = - \frac{1}{h'(s)}$\\[6pt]
% 	    Meist setzen wir:
% 	    \ceqbox{y_c=-\frac{1}{\inf(h'(x))},\quad\text{falls}\ h'(x)<0}
%               Für alle $y>y_c$ existiert keine starke Lösung.
%               Die Charakteristiken können sich also schneiden, \textcolor{red}{\textbf{wenn $h(x_0)$ abnimmt, d.h. $\frac{d}{dx}h(x)=h'(x)<0$}}\\
%               Ausdruck $y(x)$ finden, Charakteristiken sind Geraden. Entlang dieser Charakteristiken ist $u = h(s)$ konstant! \\
% 		Charakteristiken könne sich schneiden, an solchen Punkten ist $u$ nicht wohldefiniert
%               Die Kurve bewegt sich mit der mittleren Geschwindigkeit der Geschwindigkeit links und rechts der Kurve: \\[6pt]
%               \resizebox{\linewidth}{!}{\eqbox{\gamma '(y) = \frac{F(u^+)-F(u^-)}{u^+-u^-}=\frac{[F]}{[u]}=\frac{\text{Sprung von F}}{\text{Sprung von U}}}}
              
%               Die schwache Lösung bildet sich aus:
%                \[\begin{cases}
%               \gamma(y) = \int \gamma '(y) dy + C \\
% 	    C=\text{Alte Ungleichungsgrenze}
%               \end{cases} \]
%               \ceqbox{u(x,y)=\begin{cases}
%               u^-\quad &x<\gamma(y)\\
%               u^+\quad &x>\gamma(y)
%               \end{cases}}
% 	\subsection{Charakteristiken Graphisch zeichnen}
% 	\begin{enumerate}
% 	\item Löse die Charakteristischen Gleichungen: $x(t,s),\ y(t,s),\ u(t,s)$
% 	\item Löse $x(t,s)$ nach t auf, uns setze es in $y(t,s)$ ein
% 	\item Erhalte $y(x,s)$
% 	\item Der Punkt, gegeben als $\big(x(0,s),\ y(0,s),\ u(0,s)\big)$ mit Hilfe von 1. nach $s$ auflösen
% 	\item Zeichne $y(x,s)$ für jedes gefundene $s$
% 	\end{enumerate}	
% \section{Lin. PDG 2. Ordnung}
% 	\subsection{Klassifikation}
% 		Allgemeine Form:\\
% 		\resizebox{\linewidth}{!}{
% 		    \eqbox{L[u] = \underbrace{au_{xx} + 2bu_{xy} + c u_{yy}}_{\text{Hauptteil, nur T. 2. Ord.}} + d u_x + e u_y + f u = g}
% 		}
% 		\\
% 		\\
% 		linear: nur Terme wie $u, u_x, u_{xx}$\\
% 		semilinear: nur Terme wie $u \cdot u_x, u \cdot u_y, u \cdot u_{xx}$\\
% 		\\
% 		\textbf{Def:} Diskriminante\\
% 		\\
% 		\eqbox{\delta(L)(x,y) \coloneqq b^2(x,y) - a(x,y)\cdot c(x,y)}
% 		\\
% 		\\
% 		Die PDG heisst im Punkt $(x,y)$ \\
% 		$\left \{
% 		\begin{array}{l}
% 	     	\text{hyperbolisch} \\
% 	     	\text{parabolisch} \\
% 	     	\text{elliptisch}
% 	    \end{array}
% 	    \right . \text{, falls} \quad \delta(L)(x,y)
% 	    \left \{
% 	    \begin{array}{l}
% 	     	> 0\\
% 	     	= 0\\
% 	     	< 0
% 	    \end{array}
% 		\right .$
% 		\\
% 		\\
% 		\textbf{Satz:} Der Typ einer lin. PDG 2. Ordnung ist invariant unter Koordinatentransformation.
		
% 		\subsubsection{Hauptsymbol A}
% 		Die Gleichung $au_{xx} + 2bu_{xy} + cu_{yy} = 0$ hat als Hauptsymbol:
% 		\[A = \left| \begin{array}{ll}
% 				     	a & b\\
% 				     	b & c\\
% 				    \end{array}\right|\]
				    
% 				    Die PDG heisst im Punkt $x_0$
% 		\begin{center}$\left \{
% 		\begin{array}{l}
% 	     	\text{elliptisch} \\
% 	     	\text{parabolisch} \\
% 	     	\text{hyperbolisch}
% 	    \end{array}
% 	    \right . \text{, falls} \quad det(A)
% 	    \left \{
% 	    \begin{array}{l}
% 	     	> 0\\
% 	     	= 0\\
% 	     	< 0
% 	    \end{array}
% 		\right .$\end{center}
		
% 		\subsubsection{Variablenwechsel}
% 		\[\underbrace{\Omega'}_{\in\, \mathbb{R}^n, \quad t_1,...,t_n} \mathrel{\mathop{\rightleftarrows}^{\mathrm{\Phi}}_{\mathrm{\Psi}}} \underbrace{\Omega}_{\in\, \mathbb{R}^n, \quad x_1,...,x_n}\]
		
% 		Es gilt:
% 		\[J_{\Phi}(t)=\left(\frac{\partial}{\partial t_i}\Phi_j(t)\right)_{ij} \qquad J_{\Psi}(t)=\left(\frac{\partial}{\partial t_i}\Psi_j(t)\right)_{ij}\]
% 		\[J_{\Phi}(t)\cdot J_{\Psi}(t)=J_{\Psi}(t)\cdot J_{\Phi}(t)=I\quad (Einheitsmatrix)\]
		
% 		Das neue Haupstymbol $A_0$ ist dann:
% 		\begin{center}
% 			\eqbox{A_0 = J_\Psi^T A J_\Psi}
% 		\end{center}			
% 		\textbf{Korollar:} Die Vorzeichen der EW des Hauptsymboles bleiben unter Variablenwechsel erhalten
% 		\subsection{Die kanonische Form}
% 			Die kanonische Form hat zum Ziel durch Variablen-Transformation, eine vordefinierte Form des Hauptsymbols zu erreichen.
% 			\begin{enumerate}
% 			\item $s=s(x,y)\quad t=t(x,y)$
% 			\item Jacobi-Matrix: $\Psi=\begin{pmatrix} s_x & t_x \\ s_y & t_y \end{pmatrix}$\\
% 			$\Psi^TA\Psi=\begin{pmatrix} s_x & s_y \\ t_x & t_y \end{pmatrix}\begin{pmatrix} a & b \\ b & c \end{pmatrix}\begin{pmatrix} s_x & t_x \\ s_y & t_y \end{pmatrix}=\begin{pmatrix} a_0 & b_0 \\ b_0 & c_0 \end{pmatrix}$\\
% 			$a_0=a\cdot s_x^2+2b\cdot s_x\cdot s_y+c\cdot s_y^2$\\
% 			$b_0=a\cdot s_x\cdot t_x+b\cdot (s_x\cdot t_y+s_y\cdot t_x)+c\cdot s_y\cdot t_y$\\
% 			$c_0=a\cdot t_x^2+2b\cdot t_x\cdot t_y+c\cdot t_y^2$
% 			\begin{itemize}
% 				\item $det(A)>0$: elliptisch $\Rightarrow \Psi^T A\Psi=\begin{pmatrix}  1 & 0 \\ 0 & 1 \end{pmatrix}$
% 				\item $det(A)=0$: parabolisch $\Rightarrow \Psi^T A\Psi=\begin{pmatrix}  1 & 0 \\ 0 & 0 \end{pmatrix}$
% 				\item $det(A)<0$: hyperbol. $\Rightarrow \Psi^T A\Psi=\begin{pmatrix}  0 & \frac{1}{2} \\ \frac{1}{2} & 0 \end{pmatrix}$
% 			\end{itemize}
% 			\item Anhand der Komponenten können PDG's aufgestellt werden.\\
% 			Bsp.: parabolisch $\Rightarrow a_0=0\quad c_0=0$\\
% 			ACHTUNG: Es dürfen für $s$ und $t$ nicht die gleichen Cauchy-Probleme gelöst werden.\\
% 			Als \textbf{Anfangsbedingung}, kann immer $s(0,y)=y$ verwendet werden.
% 			\item Hat man $s(x,y)$ und $t(x,y)$ bestimmt, führt man die Funktion $v(s,t)=u(x,y)$ ein.
% 			\item Nun kann man unter Verwendung der Kettenregel $v(s,t)$ in die Gleichung substituieren und ist fertig.
% 			\end{enumerate}
       				
% 	\subsection{Normalformen}
% 		\textbf{Satz:} \\
% 		Sei eine lin. PDG 2. Ord. in einem Gebiet D
% 		\begin{enumerate}
% 			\item hyperbolisch
% 			\item parabolisch
% 			\item elliptisch, Koeff. a,b,c analytisch
% 		\end{enumerate}
% 		, dann gibt es eine Koordinatentransformation \\
% 		$x = x(r,s)$\\
% 		$y = y(r,s)$\\
% 		\\
% 		so, dass die Transformierte PDG die Form
% 		\begin{enumerate}
% 			\item $w_{rs} + l_1(w) = g(r,s)$
% 			\item $w_{rr} + l_1(w) = g(r,s)$
% 			\item $w_{rr} + w_{ss} + l_1(w) = g(r,s)$
% 		\end{enumerate}
% 		hat, mit \\
% 		\eqbox{w(r,s) = u(x(r,s), y(r,s))}\\
% 		\\
% 		$\rightarrow l_1$ ist ein Differentialoperator 1. Ordnung
	
% 	\subsection{Ableitungen}
% 		$u_x = \omega_r r_x + \omega_s s_x$ \\
% 		$u_y = \omega_r r_y + \omega_s s_y$\\
% 		$u_{xx} = \omega_{rr} r_x^2 + 2\omega_{rs} r_x s_x + \omega_{ss} s_x^2  + \omega_r r_{xx}+ \omega_s s_{xx}$\\
% 		$u_{xy} = \omega_{rr} r_x r_y + \omega_{rs}(r_x s_y + r_y s_x) + \omega_{ss} s_x s_y + \omega_r r_{xy} + \omega_s s_{xy}$\\
% 		$u_{yy} = \omega_{rr} r_y^2 + 2\omega_{rs} r_y s_y + \omega_{ss} s_y^2 + \omega_r r_{yy} + \omega_s s_{yy}$
		
% 	\subsection{Lösung der PDG}
% 		\begin{enumerate}
% 			\item $r$ und $s$ sind gegeben durch charakteristische Gleichung: \\
% 				\eqbox{\frac{dy}{dx} = \frac{b \pm \sqrt{b^2 - ac}}{a} = \frac{b \pm \sqrt{\delta}}{a}}
% 			\item Gleichung auf beiden Seiten integrieren und nach Int.-Konstante auflösen ergibt zwei Lösungen: $\rightarrow r(x,y) \; \text{und} \; s(x,y)$
% 			\item es gilt: $u(x,y) = \omega(r,s)$, partielle Ableitungen von $u$ berechnen (Kettenregel: innere Ableitungen beachten!)
% 			\item partielle Ableitungen in original PDG einsetzen
% 			\item Gleichung lösen (evt. Subsitution)
% 			\item Rücktransformation, evt. Anfangsbedingungen einsetzen
% 		\end{enumerate}

% \section{Die 1D Wärmeleitungsgleichug}
% \attention\textcolor{red}{Maximum auf dem Rand!}\\
% 				$\begin{cases}
% 					u_t = \alpha u_{xx}\\
% 					u(0,x)=f(x)\qquad\text{Anfangsbed.}\\
% 					u(t,0)=u(t,L)=0\qquad \text{Randbed.}
% 				\end{cases}$
				
% 	Mit Hilfe von Variablenseparation und Superpositionsprinzip folgt:\\
% 	\[u(t,x)=T(t)X(x)\]
% 	\[\text{mit}\;u_t=T'(t)X(x)\;\text{und}\;u_{xx}=T(t)X''(x)\]
% 	\[\Rightarrow T'(t)X(x)=\alpha T(t)X''(x)\quad\]\[\Rightarrow\frac{T'(t)}{T(t)}=\alpha\frac{X''(x)}{X(x)}=-\lambda\]
% 	somit ist die Linke seite von $x$ unabhängig und die rechte von $t$ und somit gleich eine Konstante $\lambda$. Es folgen hierraus die ODE's
% 	\[T'(t)+\lambda T(t)=0\]
% 	\[X''(x)+\frac{\lambda}{\alpha}X(x)=0\]
% 	für positive $\lambda$ ist die Lsg dann:
% 	\[T(t)=Ce^{-\lambda t}\]
% 	\[X(x)=A\sin(\omega x)+B\cos(\omega x)\qquad\omega:=\sqrt{\frac{\lambda}{\alpha}}\]
% 	mit Konstanten $A,B,C\;\in\mathbb{R}$.\\Die folgende Funktion ist eine Lösung der Wärmeleitungsgleichung:
% 	\[u(t,x)=e^{-\lambda t}(A\sin (\omega x)+B\cos(\omega x))\]
% 	Mit den Randbedingungen folgt nun für die Konstanten:
% 	\[0=u(t,0)=T(t)X(0)=Be^{-\lambda t}\quad \Rightarrow \;B=0\]
% 	\[0=u(t,L)=T(t)X(L)=Ae^{-\lambda t}\sin(\omega L)\quad \Rightarrow \;A=1\]
% 	Mit $\omega =\frac{\pi n}{L}\quad,n\in\mathbb{N}^+$ erhält man die Lsg:
% 	\[u_n(t,x)=e^{-\frac{\pi^2n^2\alpha}{L^2}t}\cdot\sin\left(\frac{\pi n}{L}x\right)\]
% 	ODER mit der zweiten Variante des Superposprinz.:
% 	\begin{center}
% 		\eqbox{u(t,x)=\sum_{n=1}^{\infty}c_ne^{-\frac{\pi^2n^2\alpha}{L^2}t}\cdot\sin\left(\frac{\pi n}{L}x\right)}
% 	\end{center}
% 	mit den Fourierkoeffizienten:
% 	\[c_n = \frac{2}{L}\int_{0}^{L}f(x)\sin\left(\frac{\pi n}{L}x\right)dx\]
	
	
% \section{Die 1D Wellengleichung}
% 	\subsection{Allg. Lösung}
% 		Wir suchen Lösungen der Gleichung: \\
% 		\eqbox{u_{tt} - c^2 \cdot u_{xx} = 0} , $t > 0, r \in \mathbb{R}$ \\
% 		\\
% 		Transformation auf Normalform: \\
% 		$r = x + ct, \quad s = x - ct$ \\
% 		$\omega(r(x,t), s(x,t)) = u(x,t)$ \\ \\
% 		$\Rightarrow -4c^2 \cdot \omega_{rs} = 0$ \\
% 		$\Rightarrow \omega = F(r) + G(s)$ \\
% 		\\
% 		allg. Lsg: \eqbox{u(x,t) = F(x + ct) + G(x - ct)}\\
% 		wobei $F(x-ct)$ sich nach rechts (forwards) bewegt $(+x\ Richtung)$ und $F(x+ct)$ nach links (backwards).
	
% 	\subsection{Formel d'Alembert}
% 	\subsubsection{Homogene Wellengleichung}
% 		\[\left \{
% 		\begin{array}{l}
% 	     	u_{tt} - c^2 \cdot u_{xx} = 0 \qquad x \in \mathbb R,\ t \ge 0\\
% 	     	u(x,0) = f(x) \\
% 	     	u_t(x,0) = g(x) \\
% 		\end{array}
% 		\right .\]
% 		Mit Ansatz mit zu bestimmenden Funktionen F, G:
% 		$$u\left(x,t\right) = F\left(x+ct\right)+G\left(x-ct\right)$$
% 		\textbf{Allgemeine Lösung des Anfangswertproblem}
% 		\eqbox{u = \frac 1 2 (f(x+ct) + f(x - ct)) + \frac{1}{2c} \int \limits_{x - ct}^{x + ct} g(\xi)d\xi}
		
% 	\subsubsection{Inhomogene Wellengleichung}
% 	\textbf{Anfangswertproblem}
% 	\begin{align}
% 	\begin{cases}
% 	u_{tt}-c^2u_{xx}=F\left(x,t\right), & x \in \mathbb R,\ t \ge 0\\
% 	u\left(x,0\right)=f\left(x\right), \\
% 	u_{t}\left(x,0\right)=g\left(x\right).\\
% 	\end{cases}
% 	\end{align}
% 	Mittels Green'scher Formel wird das Anfangswertproblem gelöst.\\
% 	\textbf{Allgemeine Lösung des Anfangswertproblem}
%       		\resizebox{\linewidth}{!}{
%       		    \eqbox{u = \frac{1}{2}(f(x+ct) + f(x - ct)) + \frac{1}{2c} \int \limits_{x - ct}^{x + ct} g(s)ds + \frac{1}{2c} \int\int_{\Lapl}F\left(\xi,\tau\right)d\xi d\tau}
%   		    }
%   		    \resizebox{\linewidth}{!}{\eqbox{\int\int_{\Lapl}F\left(\xi,\tau\right)d\xi d\tau = \int_0^t\int_{x-c(t-\tau)}^{x+c(t-\tau)}F\left(\xi,\tau\right)d\xi d\tau}}	
  		    
% 	\subsubsection{Beispiel}
% 	\begin{align}
% 	\begin{cases}
% 	u_{tt}-9u_{xx}=e^x+e^{-x}, & x \in \mathbb R,\ t \ge 0\\
% 	u\left(x,0\right)=x, \\
% 	u_{t}\left(x,0\right)=\sin\left(x\right).\\
% 	\end{cases}
% 	\end{align}
% 	\begin{center}
% 	 \includegraphics[width=\columnwidth,keepaspectratio]{images/InHomWellenGL_2.png}
% 	\end{center}
%     \subsection{Domain of Dependence}
%     Wenn man die Lösung nach der d'Alembert Formel an einem bestimmten Punkt $(x_0,t_0)$ betrachtet:
%     \[\hspace*{-10pt}u(x_0,t_0)=\frac{f(x_0+ct_o)+f(x_0-ct_0)}{2}+\frac{1}{2c}\int\limits_{x_0-ct_0}^{x_0+ct_0}g(s)ds\]
%     erkennt man, dass die Lösung nur von Punkten auf und am Rand eines bestimmten Intervalls der Anfangskurve $u(x,0)$ und $u_t(x,0)$ abhängt.\\\textbf{Intervall:} $[x_0-ct_0, x_0+ct_0]$
% \begin{center}
% 	 \includegraphics[width=\columnwidth,keepaspectratio]{images/domofdep.png}
% 	\end{center}	    \subsection{Region of Influence}
%     Umgekehrt kann man sich auch fragen, welche Punkte in der $(x,t)$-Ebene für $t>0$ von einem bestimmten Intervall $[a,b]$ auf der Anfangskurve beeinflusst werden. Aus der vorherigen Diskussion muss für einen Punkt $(x_0,t_0)$ das folgende gelten:
%     \[[x_0-ct_0,x_0+ct_0]\cap[a,b]\neq\emptyset\]
%     Anders ausgedrückt, alle Punkte $(x,t)$ werden vom Intervall $[a,b]$ beeinflusst für die gilt:
%     \[\{(x,t)|(x-ct\leq b)\wedge(x+ct\geq a)\}\]
%     ($f$ beeinflusst in unteren Ecken des Dreieck, $g$ an unterer Kante)
%     \subsection{Graphische Lösung}
% \begin{enumerate}
% \item Betrachte die kritischen Stellen (z.B. an Intervall übergängen), z.B. $-1,0,1,2$ bei $[-1,0]\cup[1,2]$
% \item Setze in D'Alembert ein (Wellenfunktion)
% \vspace*{-0.2cm}
% \[ \hspace*{-0.3cm} \text{(im Bsp.: 8 Gl lösen)}
% \begin{cases}
% x+ct=\ \text{Kritischer Punkt}\\
% x-ct=\ \text{Kritischer Punkt}
% \end{cases}\]
% \vspace*{-0.4cm}
% \item Zeichne alle Gleichungen
% \item Wert in Schnittfläche: $\frac{1}{2}$ der Summe der Werte in \underline{nicht} Schnittflächen
% \end{enumerate}
% \subsection{Gerade, ungerade und Periodische L"osungen}
%         \begin{align}
%           		\begin{cases}
%               		u_{tt}-c^2u_{xx}=F\left(x,t\right), & x \in \mathbb R,\ t \ge 0\\
%               		u\left(x,0\right)=f\left(x\right), \\
%               		u_{t}\left(x,0\right)=g\left(x\right).\\
%           		\end{cases}
%       		\end{align}
%   		Angenommen $f$ und $g$ seien gerade Funktionen, und f"ur alle $t\geq0$ sei $F(\cdot,t)$ auch gerade. Dann ist jede L"osung $u(\cdot, t)$ des Cauchy Problems auch gerade.\\ "Ahnlich gilt, dass jede L"osung ungerade oder periodisch mit der Periode $L$(als Funktion von $x$) ist, wenn the die Angaben ungerade bzw.periodisch mit Periode $L$ sind.
% \section{Separation der Variablen}
	
% 	\subsection{Separationsansatz}
% 			Kommen in der PDG keine gemischten partiellen Ableitungen vor, kann der Separationsansatz / Bernoullische Produktansatz verwendet werden.
% 			\begin{enumerate}
% 				\item Ist der Ansatz möglich? Gleichung der Form:\\
% 				$u_{xx}+\frac{2}{x}u_x+\frac{1}{x^2}u_{y}+\frac{1}{x^2}u_{yy}+2zu_z-u_{zz}=0$
% 				\item Substitution: $u(x,y,z):=X(x)\cdot Y(y)\cdot Z(z)$:\\
% 				\hphantom{1}\dis{-1}\mbox{$X''YZ+2\frac{X'YZ}{x}+\frac{XY'Z}{x^2}+\frac{XY''Z}{x^2}+2zXYZ'-XYZ''=0$}
% 				\item Division der PDG durch $u(x,y,z)$:\\
% 				$\frac{X''}{X}+2\frac{X'}{xX}+\frac{Y'}{x^2Y}+\frac{Y''}{x^2Y}+2z\frac{Z'}{Z}-\frac{Z''}{Z}=0$
% 				\item Terme mit gleichen Variablen auf eine Seite bringen:\\
% 				$\frac{X''}{X}+2\frac{X'}{xX}+\frac{Y'}{x^2Y}+\frac{Y''}{x^2Y}=-2z\frac{Z'}{Z}+\frac{Z''}{Z}$
% 				\item \textbf{Standard-Argumentation:} Beide Seiten müssen Konstant sein, da sie von jeweils unterschiedlichen Variablen abhängig sind. Die Seite mit einer Variablen, kann zu einer ODG vereinfacht werden:\\
% 				$-2z\frac{Z'}{Z}+\frac{Z''}{Z}=\lambda \Leftrightarrow Z''-2zZ'-\lambda Z=0$
% 				\item Schritte 4 und 5 wiederholen, bis nur noch ODG vorhanden sind:\\
% 				$x^2\frac{X''}{X}+2x\frac{X'}{X}-x^2\lambda=-\frac{Y''}{Y}-\frac{Y'}{Y}$\\
% 				$\Rightarrow Y''+Y'+\mu Y=0$\\
% 				$\Rightarrow x^2X''+2xX'-(\mu+\lambda x^2)=0$
% 				\item DGL oder DGL-System mit bekannten Methoden lösen.
% 				\item Löungen:\\
% 				$X(x)Y(y)Z(z)=\sum\limits_{\lambda,\mu}X_{\lambda,\mu}(x)Y_\mu(y)Z_\lambda(z)$
% 			\end{enumerate}
% 			\subsection{Lösen einer homogenen Gleichung}
% 			Gegeben sei ein Cauchy-Problem:
% 			$$\begin{cases} u_{tt}-c^2u_{xx}=0 \\ t\geq0, x\in[0,\pi] \end{cases}$$
% 			Die ist ein Spezialfall einer PDE, die Technik kann aber analog auf andere Probleme übertragen werden. Für eine eindeutige Lösung brauchen wir Anfangs- und Randbedingungen:\\[6pt]
% 			\textbf{Dirichlet-Bedingung:} (Randbed.)
% 			\[u(0,t)=u(\pi,t)=0\]
% 			\textbf{Von Neumann-Bedingung:} (Randbed. Steigu.)
% 			\[u_x(0,t)=u_x(\pi,t)=0\]
% 			\textbf{Anfangsbedingung:} (Zeit $t=0$)
% 			\[u(x,0)=f(x),\ x\in[0,\pi]\]
% 			Durch Substitution kann das Problem auf ein System von ODG's zurückgeführt werden:\\
% 			\begin{center}\eqbox{u(x,t)=X(x)T(t)}\end{center}
% 			\begin{enumerate}[label=\textbf{\arabic*.}]
% 				\item Einsetzten in Gleichung
% 				\item Dividiere die PDG durch Ansatz $\rightarrow\cdot\frac{1}{X(x)T(t)}$
% 				\item Umformen zu: $\frac{X(x)''}{X(x)}=\frac{T(t)''}{T(t)}\meq-\lambda,\ \lambda\in\mathbb{R}$
% 				\item Schreibe Gleichung\underline{en} $X(x)''=-\lambda X(x)$ usw.
% 				\item Löse die ODG's für \underline{alle} Vorzeichen $(\lambda\ge0,\ \lambda=0,\ \lambda\le0)$
% 				\item Eine Lösung ist dann jeweils das Produkt $X(x)T(t)\ \rightarrow$ Superpositionsprinzip erlaubt Addition all dieser Lösungen
% 				\item Als Regel: Immer zuerst für die Konstanten der Funktion $X(x)$ für ein bestimmtes $\lambda$ lösen und herausfinden ob Koeffizienten durch homogene Randb. $=0$ sind.\\ Finde ausserdem weitere Bedingungen für $\lambda$, z.B. $\lambda=n^2,\ n=1,2,3,\dots$
% 				\item $\lambda=0$ kann helfen nicht-homogene Randb. auf $0$ zu setzen, weil die Lsg. für $X(x)$ Konstanten oder Polynome sind und $T(t)$ konstant ist. (Wärmeleitungsgl.)
% 				\item Setze $t=0$ und finde exakte Konstanten mithilfe der AB
% 				\item \textbf{\attention ÜBERPRÜFE DEINE LÖSUNG}
% 			\end{enumerate}
% 			\subsection{Lösen einer inhomogenen Gleichung}
% 			    Betrachte z.B. folgendes inhomogenes Problem mit Dirichlet Bedingungen:
% 			    \[\begin{cases}u_{tt}-u_{xx}=F(x,t),\ t\geq0,x\in[0,L] \\ u(0,t)=u(L,t)=0 \\ u(x,0)=f(x)\end{cases}\]
% 			    Wir lösen dies mit folgender Methode:
% 			\subsubsection{Method of Eigenfunction Expansion}
% 			\textbf{Vorgehensweise:}
% 			\begin{enumerate}[label=\textbf{\arabic*.}]
% 			    \item \attention Stelle sicher, dass die Randbedingungen homogen sind, ansonsten siehe nächstes Kapitel
% 			    \item Setze einen Separationsansatz $u^*(x,t)=v(x)T(t)$ in die \underline{zugehörige homogene} PDG ein.
% 			    \item Dividiere die PDG durch Ansatz $\rightarrow\cdot\frac{1}{v(x)T(t)}$
% 				\item Umformen zu: $\frac{v(x)''}{v(x)}=\frac{T(t)''}{T(t)}\meq-\lambda,\ \lambda\in\mathbb{R}$
% 				\item Schreibe \underline{nur die} Gleichung für Ortsvariable x auf. Die Lsgn. dieser Gl. (Sturm-Liouville problem, SLP) abhängig von den Randb. nennen sich Eigenfct. mit Eigenvalue $\lambda$
% 				\item \attention\textcolor{red}{SCHREIBE EW und EF auf}
% 				\item Für Bsp wäre das SLP: $v''(x)+\lambda v(x)=0$ mit entsprechenden Randb.
% 				\item Die Lsg. sind dann je nach Randb wie folgt (Eigenfct.) (\attention für dieses Bsp!):
%   					\[\begin{cases}\textbf{Dirichlet:}\\v_n(x)=\sin(\frac{n\pi}{L}x),\ n=1,2,\dots \\ \textbf{Von Neumann:}\\v_n(x)=\cos(\frac{n\pi}{L}x),\ n=0,1,2,\dots \end{cases}\]
% 			    \item Die Basis wird als Reihe mit (noch unbekanntent) zeitabh. Fourierkoeff. expandiert. Daraus folgt (\attention in diesem Bsp:)
% 		    		\[\begin{cases}\textbf{Dirichlet:}\\u(x,t)=\sumnia{1}T_n(t)\sin(\frac{n\pi}{L}x),\ n=1,2,\dots \\ \textbf{Von Neumann:}\\u(x,t)=\sumni T_n(t)\cos(\frac{n\pi}{L}x),\ n=0,1,2,\dots \end{cases}\]
%                   \item Expandiere $F(x,t)$ in derselben Fourier-Basis in eine Fourierreihe.
%                   \item Setze den Ansatz in die PDE ein und führe Koeffizientenvgl. durch und erhalte ODG für jedes $T_n$
%                   \item Löse die ODG's \textcolor{red}{zusammen mit den Anfb.} für $t=0$
% 				\item \textbf{\attention ÜBERPRÜFE DEINE LÖSUNG}
% 			\end{enumerate}
			
% 			\subsection{Nicht homogene Randbedingungen}
% 			Führe eine Hilfsfunktion $w(x,t)$ ein, welche die inhomogenen Randb. erfüllt. Sind die Randb. in folgender Form, können die Funktionen der Tabelle entnommen werden:
% 			\[\begin{cases}B_a[u]=\alpha u(a,t)+\beta u_x(a,t)=a(t),\ t\geq0 \\ B_b[u]=\gamma u(b,t)+\delta u_x(b,t)=b(t),\ t\geq0\end{cases}\]
% 			\includegraphics[width=\columnwidth]{images/table_inho.png}\\[6pt]
% 			Mithilfe des Superpositionsprinzip können wir $v(x,t)=u(x,t)-w(x,t)$ lösen:
% 			\[\begin{cases}v_t-v_{xx}=\tilde{F}(x,t),&t\geq0,x\in[a,b] \\ B_a[v]=0,&t\geq 0 \\ B_b[v]=0,&t\geq 0 \\ v(x,0)= \tilde{f}(x) \end{cases}\]
% 			Wobei $\tilde{F}(x,t)=F(x,t)+w_{xx}-w_t$ und $\tilde{f}(x)=f(x)-w(x,0)$\\
% 			\textbf{$\Rightarrow$ Nach dem lösen ist $u(x,t)=w(x,t)+v(x,t)$}

% \section{Laplace-Gleichung}
% 	\subsection{Harmonische Funktionen}
% 		\subsubsection{Definition}
% 		Sei $D\subset\mathbb{R}^n$. Eine Funktion $f:D\to\mathbb{R}^n$ heisst harmonisch auf D falls sie zweimal stetig diffbar ist und für alle $x\in D$ gilt:
% 		\[\Delta f(x)=0\]
% 	\subsection{Definition}
% 		\begin{center} \eqbox{\Delta u = u_{xx} + u_{yy} + u_{zz} = 0} \end{center}
	
% 		in Polarkoordinaten: 
% 		\begin{center}\eqbox{\Delta u = \omega_{rr} + \frac 1 r \omega_{r} + \frac 1 {r^2} \omega_{\varphi \varphi}}\end{center}
		
% 		in Kugelkoordinaten: \\
% 		$\Delta u = \omega_{rr} + \frac 2 r  \omega_r + \frac 1 {r^2} 
% 		\omega_{\vartheta\vartheta} + \frac 1 {r^2}\frac{\cos \vartheta}{\sin \vartheta} 
% 		\omega_{\vartheta} + \frac 1 {r^2 \sin^2 \vartheta} \omega_{\varphi \varphi}$
% 		\\ \\
% 		Lösungen dieser Gleichung heissen \textbf{harmonische Funktionen}.
		
% 	\subsubsection{Definitionen des Laplace-Operators}
% 	\begin{itemize}
% 		\item[] \textbf{Allgemein gilt}
% 		$$\Delta = \dfrac{\partial^2}{\partial x_{1}^2} + \dfrac{\partial^2}{\partial x_{2}^2} + \cdots +\dfrac{\partial^2}{\partial x_{n}^2}$$
% 		\item[] \textbf{Kartesische Koordinaten}
% 		$$\Delta = \dfrac{\partial^2}{\partial x^2} + \dfrac{\partial^2}{\partial y^2} + \dfrac{\partial^2}{\partial z^2}$$
% 		\item[] \textbf{Polar Koordinaten}
% 		$$\Delta = \dfrac{\partial^2}{\partial r^2} + \dfrac{1}{r}\dfrac{\partial}{\partial r} + \dfrac{1}{r^2}\dfrac{\partial^2}{\partial \varphi^2}$$
% 		\item[] \textbf{Zylinder Koordinaten}
% 		$$\Delta={1 \over r} {\partial \over \partial r} \left( r {\partial \over \partial r} \right) + {1 \over r^2} {\partial^2 \over \partial \theta^2} + {\partial^2 \over \partial z^2 }$$
% 		\item[] \textbf{Kugelkoordinaten}
% 		\begin{center}
% 			 \includegraphics[width=\linewidth/1,keepaspectratio]{images/Laplace_KugelK.png}
% 		\end{center}
% 	\end{itemize}
	
% 	\subsection{Poissongleichung}
% 		PDG: $\Lapl u = F(x,y), \quad x \in \Omega$ \\[6pt]
% 		Sei D ein zshgd Gebiet und f, g Funktionen.\\[6pt]
% 		\textbf{Dirichlet Problem:} \\
% 		RB: $u(x,y) = g(x,y), \quad (x,y), \in \partial D$\\[6pt]
% 		\textbf{Neumann Problem:} \\
% 		RB: $\frac{\partial }{\partial n}u(x,y) = g(x,y), \quad (x,y)\in \partial D$\\
% 		Eine Voraussetzung für die Existenz einer Lösung des NP ist: \\
% 		$$\int_{\partial D}g(x(s),y(s))ds =\int_D F(x,y)dxdy$$ 
% 		wo $(x(s),y(s))$ eine parametrisierung von $\partial D$ ist \\[6pt]
% 		\textbf{Robin Problem:}\\
% 		RB: $u(x,y)+\alpha\frac{\partial }{\partial n}u(x,y) = g(x,y), \quad (x,y) \in \partial D$
% 	\subsection{Green's identities}
% 		\begin{enumerate}[label=\textbf{\arabic*.}]
% 			\item \[\int_D\Delta u\dx\dy=\int_{\partial D}\partial_nu\ds\]
% 			\item \[\int_D(v\Delta u-u\Delta v)\dx\dy=\int_{\partial D}(v\partial_n u-u\partial_n v)\ds\]
% 			\item \[\int_D\vec\nabla u\cdot\vec\nabla v \dx\dy = \int_{\partial D}v\partial_nu\ds-\int_Dv\Delta u\dx\dy\]
% 		\end{enumerate}
% 	\subsection{General uniqueness theorem for Pisson's eq.}
% 		\begin{itemize}
% 			\item Das Dirichlet-Problem hat höchstens eine Lösung
% 			\item Das ``problem of the third kind''/Robin problem:
% 			\[u(x,y)+\alpha(x,y)\partial_nu(x,y)=g(x,y),\quad(x,y)\in\partial D\]
% 			hat höchstens eine Lösung für alle $\alpha\geq0$
% 			\item Falls eine Funktion $u$ das Neumann-Problem löst, dann hat jede andere Lösung die Form $v=u+c,\ c\in\mathbb{R}$
% 		\end{itemize}
% 	\subsection{Maximumprinzip}
% 		\textbf{Satz: (Schwaches Maximumprinzip)} \\
% 		Sei $D$ ein beschränktes Gebiet, $u$ harmonisch ($\Delta u = 0$) in $D$ und stetig in 
% 		$\overline{D}$. Dann wird das Maximum von $u$ auf dem Rand $\partial D$ angenommen.	
% 		\\
% 		\textbf{Satz: (Starkes Maximumprinzip)} \\
% 		Nimmt eine harmonische Funktion $u$ im Innern ihres zusammenhängenden Definitionsbereichs $D$
% 		ein Maximum an, so ist $u$ eine Konstante.
% 	\subsection{\attention\textcolor{red}{ Bedingung für Maximum von Funktionen}}
% 		Um zu zeigen, dass im Inneren kein Maximum existiert, müssen die \underline{notwendigen} Bedingungen für ein Maximum zusammen mit der Annahme einen Widerspruch ergeben. In 2-Dimensionen sehen diese wie folgt aus:
% 		\begin{itemize}
% 			\item \textbf{Gradient}
% 			\[\vec\nabla u(x,y)=\MATR{u_x \\ u_y}=0\]
% 			\item \textbf{Hesse-Matrix} muss neg.-definit sein \textcolor{red}{$(\Lapl u<0)$}
% 			\[H=\MATR{\Pablq{u}{x} & \frac{\partial^2u}{\partial x\partial y} \\ \frac{\partial^2u}{\partial x\partial y} & \Pablq{u}{y}}\]
% 		\end{itemize}
% 		\subsubsection{Negative Definitheit 2x2 Matrix}
% 		Eine 2x2 Matrix heisst negativ definit
% 		\[\Leftrightarrow \Pablq{u}{x}<0\quad\text{und}\quad\Pablq{u}{x}\cdot\Pablq{u}{y}>\bigg(\frac{\partial^2u}{\partial x\partial y}\bigg)^2\]
% 		Daraus folgt: $\Lapl u < 0$
% 	\subsection{Mittelwerteigenschaft}
% 		\textbf{Satz: (Mittelwerteigenschaft/mean value principle)} \\
% 		Sei $u$ harmonisch in $D$ und einer Kreisscheibe $B_R \subset D$ mit Radius $R$, Mittelpunkt 
% 		$(x_0, y_0) \in D$ und Rand $\partial B_R$. Dann ist $u(x_0, y_0)$ gleich dem Mittelwert
% 		von $u$ auf dem Kreis $\partial B_R$. 
% 		\begin{center}
% 		    \eqbox{u(x_0 , y_0) = \frac{1}{2 \pi R}\oint_{\partial B_R} u(x(s), y(s))\ds}
% 		    \resizebox{\linewidth}{!}{
% 		        \eqbox{u(x_0 , y_0) = \frac{1}{2 \pi}\int_{0}^{2 \pi} u(x_0+R cos(\theta), y_0+R sin(\theta)) \,d\theta}
% 		    }
% 		\end{center}
% 		Es gilt auch die Umkehrung:
% 		Erfüllt $u$ die Mittelwerteigenschaft, so ist $u$ harmonisch.
%       \subsection{Separation der Vairablen auf einem Rechteck}
%     		Betrachte das Dirichletproblem auf einem rechteckigen Bereich $D$:
%     		\[\begin{cases}
%     		\Lapl u(x,y)=0\quad a<x<b,\quad c<y<d\\
%     		u(a,y)=f(y),\ u(b,y)=g(y),\\u(x,c)=h(x),\ u(x,d)=k(x)
%     		\end{cases}\]
%     			Um dies zu lösen brauchen wir homogene RB in min. einer Koordinate.
%     		\begin{enumerate}[label=\textbf{\arabic*.}]
%     			\item Setze $u=u_1+u_2$
%     			\[\begin{cases}
%     				u_1(a,y)=f(y),\ u_1(b,y)=g(y),\\u_1(x,c)=0,\ u_1(x,d)=0
%     			\end{cases}\]
%     			\[\begin{cases}
%     				u_2(a,y)=0,\ u_2(b,y)=0,\\u_2(x,c)=h(x),\ u_2(x,d)=k(x)
%     			\end{cases}\]
%     			\item ``combatibility condition'' muss erfüllt sein:
%     			\begin{align*}f(c)&=f(d)=g(c)=g(d)\\&=h(a)=h(b)=k(a)=k(b)=0\end{align*}
%     			Ist dies nicht der Fall, nimm ein allg. harm. Polynom als Hilfsfunktion:
%     			\[P_2(x,y)=a_1(x^2-y^2)+a_2xy+a_3x+a_4y+a_5\]
%     			\item Definiere $v(x,y)=u(x,y)-P_2(x,y)$
%     			\item $v(x,y)$ erfüllt die condition genau dann, wenn:
%     			\begin{align*}\begin{cases}
%     				G(a,c)+a_1(a^2-c^2)+a_2(a\cdot c)+a_3a+a_4c+a_5=0\\
%     				G(b,c)+a_1(b^2-c^2)+a_2(b\cdot c)+a_3b+a_4c+a_5=0\\
%     				G(a,d)+a_1(a^2-d^2)+a_2(a\cdot d)+a_3a+a_4d+a_5=0\\
%     				G(b,d)+a_1(b^2-d^2)+a_2(b\cdot d)+a_3b+a_4d+a_5=0
%     			\end{cases}\end{align*}
%     			Wobei $G(x,y)$ die angegebene Funktion auf dem Rand ist.  Danach kann obiges Problem für $v(x,y)$ mit dem illustrierten Schema gelöst werden.\\
% 			\hspace*{-10pt}\includegraphics[width=\columnwidth,keepaspectratio]{images/sep_on_rect.png}
%     		\end{enumerate}
%     		\textbf{Kommentar:}\\
%     		Um RB einfacher in Fourierreihe zu entwickeln, sollte folgendes Problem (für $X(x),Y(y)$)
%     		\[X''(x)-\lambda^2X(x)=0,\quad a<x<b,\quad \lambda\in\mathbb{R}\]
%     		mit dem Ansatz:\vspace*{-10pt}
%     		\begin{center}
%     		\eqbox{X_n(x)=A_n\sinh\big(\lambda(x-a)\big)+B_n\sinh\big(\lambda(x-b)\big)}
%     		\end{center}
%     		gelöst werden, damit eine der beiden basis-function am jeweiligen Rand verschwindet.
%       \subsection{Von Neumann Problem for the Laplace Eq.}
%         	Man betrachte:
%         	\[\begin{cases}
%         		\Lapl u(x,y)=0,\ x\in D\\
%         		\partial_nu(x,y)=g(x,y),\ x\in\partial D
%         	\end{cases}\]
%         	Da \underline{alle} Funktionen, welche eine Lapl.gl. erfüllen \textit{harmonisch} sind, muss folgende Bedingung gelten:
%         	\begin{center}
%         	\eqbox{\int_{\partial D}\partial_nu\ds=\int_{\partial D}g\ds=0\label{eq:NB}} \eqref{eq:NB}\\[4pt]
%         	\textbf{\attention CHECK THE SIGNS! } \\
%         	\textcolor{red}{Oberfl.-Normale: + in pos. Achse, - in neg. Achse}
%         	\end{center}
% 		$\rightarrow$ Rechteck kann in vier Ränder geteilt werden, mit jeweils einer anderen Normale!\\
%         	\textbf{Ist die Gleichung \underline{nicht} erfüllt, existiert \underline{keine Lösung}!}\\
%         	\textbf{Vereinfachung des Problems}
%         	\begin{enumerate}[label=\textbf{\arabic*.}]
%         		\item Definiere $u=u_1+u_2$ und $\partial D=\partial_1D + \partial_2D$
%         		\item $u_1$ und $u_2$ erfüllen jetzt:
%         		\[\partial_nu_1=\begin{cases}
%         			g(x,y),\quad &x\in\partial_1D\\
%         			0,\quad &x\in\partial_2D
%         		\end{cases}\]
        		
%         		\[\partial_nu_2=\begin{cases}
%         			0,\quad &x\in\partial_1D\\
%         			g(x,y),\quad &x\in\partial_2D
%         		\end{cases}\]
%         		\begin{center}\attention i.A. ist nun \eqref{eq:NB} nicht erfüllt.\end{center} 
%         		\item Führe Hilfspolynom ein:
%         		\[P(x,y)=a(x^2-y^2)\]
%         		für welches explizit gilt:
%         		\[\int_{\partial_nD}P\big(x(s),y(s)\big)\ds\neq0\]
%         		\item Suche harmonische Funktionen $v_1$ und $v_2$, so dass:
%         		\[\partial_nv_1=\begin{cases}g(x,y)+a\partial_n(x^2-y^2),\quad&x\in\partial_1D\\0,\quad&x\in\partial_2D\end{cases}\]
%         		\[\partial_nv_2=\begin{cases}0,\quad&x\in\partial_1D\\g(x,y)+a\partial_n(x^2-y^2),\quad&x\in\partial_2D\end{cases}\]
%         		\item Die Lösung ist gegeben durch:
%         		\[u=v_1+v_2-a(x^2-y^2)\]
%         	\end{enumerate}
%         	\hspace*{13pt}\textbf{Kommentar:}
%         	\[\partial_nu(x,y)=\vec\nabla u(x,y)\cdot\vec n_{\partial D}(x,y)\]
		
% 	\subsection{Laplace Eq. on Circular Domain}
% 		Sei $B_a\subset\mathbb{R}^2$ die Kreisscheibe mit Radius $a$ um den Nullpunkt. Darauf stellen wir das \textit{Dirichlet-Problem} wie folgt:
% 		\[\begin{cases}
% 		\Lapl u(x,y)=0,\quad&(x,y)\in B_a\\
% 		u(x,y)=g(x,y),\quad&(x,y)\in\partial B_a\end{cases}\]
% 		\vspace*{-7pt}
% 		\subsubsection{Lösung mit Poissonformel}
% 		Wir transformieren in Polarkoordinaten:
% 		\[w(r,\theta)=u\big(x(r,\theta),y(r,\theta)\big)\]
% 		\[r=\sqrt{x^2+y^2},\quad \theta=\arctan\left(\frac{y}{x}\right)\]
% 		mit den RB ($a$: Stelle am Rand, z.B. max. Radius):
% 		\[w(a,\theta)=g\big(x(a,\theta),y(a,\theta)\big)=:h(\theta)\]
% 		Ergibt die separierbare PDG mit:
% 		\[w(r,\theta)=R(r)\Theta(\theta)\]
% 		\[r^2R''(r)+rR'(r)-\lambda R(r)=0\]
% 		\[\Theta''(\theta)+\lambda\Theta(\theta)=0\]
% 		\[\Theta(0)=\Theta(2\pi),\ \Theta'(0)=\Theta'(2\pi)\]
% 		\vspace*{-0.6cm}
% 		\ceqbox{\Theta_n(\theta)=A_n\cos(n\theta)+B_n\sin(n\theta),\ \lambda_n=n^2}
% 		\[r^2R_n''(r)+rR_n'(r)-n^2R_n(r)=0\]
% 		\vspace*{-0.6cm}
% 		\ceqbox{R_n(r)=C_nr^n+D_nr^-n, n=1,dots}
% 		\[R_0(r)=C_0+D_0\log(r)\]
% 		\attention\textbf{\textcolor{red}{log($r$) und $r^{-n}$ nicht stetig bei $r=0$, deshalb keine Lösung, falls $r$ im Intervall mit $0$ liegt!!!}}\\
% 		Wir verlangen, dass $h(\varphi)$ und ihre Ableitungen auf dem Kreis stetig sind. Das Problem wird mit der Separation der Variablen gelöst und hat folgende Lösung:\vspace*{-10pt}
% 		\[w(r,\theta)=\frac{1}{2\pi}\int\limits_0^{2\pi}K(r,\theta;a,\varphi)h(\varphi)\text{d}\varphi\]
% 		\vspace*{-0.5cm}
% 		\ceqbox{\begin{minipage}{0.75\columnwidth}
% 		\begin{align*}
% 		K(r,\theta;a,\varphi)&:=\frac{a^2-r^2}{a^2-2ar\cdot\cos(\theta-\varphi)+r^2}
% 		\\ \textcolor{red}{K(r,(\theta-\varphi))}&\textcolor{red}{:=\frac{1}{2}+\sumnia{1}r^n\cos(n(\theta-\varphi))}
% 		\end{align*}
% 		\end{minipage}
% 		}
% 		Dabei ist $K$ der Poissonkern. Die allgemeine Lösung für eine harmonische Funktion in Polarkoordinaten ist gegeben als:\vspace*{-11pt}
% 		\ceqbox{w(r,\theta)=\frac{\alpha_0}{2}+\sumnia{1}r^n\big(\alpha_n\cos(n\theta)+\beta_n\sin(n\theta)\big)}
% 	\subsection{Harmonic polynomial}
% 		\begin{itemize}
% 		\item\emph{Harmonic polynomial of degree n}
% 		\[P_n(x,y)=\sum\limits_{0\leq i+j\leq n}a_{i,j}x^iy^j\]
% 		\item\emph{Homogenous harmonic polynomial of degree n}
% 		\[P_n^H(x,y)=\sum\limits_{i+j=n}a_{i,j}x^iy^j\]
% 		\end{itemize}
% 		Dimension usw.: Serie 9
% 		\begin{itemize}
% 		\item Wandle in Polarkoordinaten um
% 		\item Löse $\Lapl P=0$
% 		\item Zeige mit Induktion, dass es eine lin. Komb. der oberen Summe ist
% 		\item Dimension: Anzahl cos und sin = 2
% 		\end{itemize}
% 	\subsection{Fundamentale Harmonische Funktion auf Kreisscheibe}
% 		Sei $B_a\subset\mathbb{R}^2$ eine Kreisscheibe mit Radius $a$ um den Nullpunkt. Folgendes Problem definiert eine harm. Fkt. auf dem Gebiet \underline{ohne} den Ursprung:
% 		\[\begin{cases}
% 		\Lapl u(x,y)= 0,\quad&(x,y)\in B_a\backslash(0,0)\\
% 		u(x,y)=0,\quad&(x,y)\in\partial B_a
% 		\end{cases}\] \\
% 		Die nichttriviale Lösung ist gegeben durch:
% 		\[u(x,y)=-\frac{1}{2\pi}\log\Bigg(\sqrt{\frac{1}{a^2}(x^2+y^2)}\Bigg)\]
		
% 		\[\begin{cases}
% 		\Lapl u(x,y)= 0,\quad&(x,y)\in B_a\backslash(0,0)\\
% 		u(x,y)=g(x,y),\quad&(x,y)\in\partial B_a
% 		\end{cases}\] \\
% 		\begin{enumerate}
% 		    \item Transformation zu Polarkoordinaten\\
% 		        $\omega(r,\theta)=u(x(r,\theta), y(r,\theta))$\\
% 		        $\Lapl = \dfrac{\partial^2}{\partial r^2} + \dfrac{1}{r}\dfrac{\partial}{\partial r} + \dfrac{1}{r^2}\dfrac{\partial^2}{\partial \varphi^2}$\\
% 		        $\omega(a,\theta)=h(\theta)=g(x(a,\theta),y(a,\theta))$
% 		    \item Wir seperieren die Variablen\\
% 		        $\omega(r,\theta) = R(r)\Theta(\theta)$\\
% 		        $r^2R''(r)+rR'(r)-\lambda R(r)=0 \quad 0<r<a$\\
% 		        $\Theta''(\theta)-\lambda\Theta(\theta)=0$\\
% 		        $\Theta$ muss periodisch sein \\$\rightarrow \Theta(0)=\Theta(2\pi), \Theta'(0)=\Theta'(2\pi)$
% 	        \item L"osen der Differentialgleichung\\
% 	            \resizebox{\linewidth}{!}{$\Theta_n(\theta)=A_n cos(n\theta)+B_n sin(n\theta), \lambda_n = n^2$}\\
% 	            $r^2R''_n+rR'_n-n^2R_n=0$\\
% 	            $n=0$\\
% 	            $\quad R_0(r)=C_0+D_0 ln(r)$\\
% 	            $n\neq0$\\
% 	            $\quad R_n(r)=C_nr^n+D_nr^{-n}, n=1,2,...$
%             \item Singularitäten\\
%                 Da nur glatte Lösungen zugelassen sind und $ln(r)$ sowie $r^{-n}$ eine Singularitäten im Ursprung $(r=0)$ haben, folgt:\\
%                 $D_n = 0 \quad n= 0,1,2,...$
%             \item Superposition\\[6pt]
%             \hspace*{-0.4cm}\resizebox{\columnwidth}{!}{\eqbox{\begin{minipage}{\columnwidth}
%                 $\omega(r,\theta)=\frac{\alpha_0}{2}+\sumnia{1}r^n\big(\alpha_n\cos(n\theta)+\beta_n\sin(n\theta)\big)$
%                 $\alpha_0 =\frac{1}{\pi}\int_0^{2\pi}{h(\phi)d\phi}$\\
%                 $\alpha_n =\frac{1}{\pi a^n}\int_0^{2\pi}{h(\phi)cos(n\phi)d\phi}$\\
%                 $\beta_n  =\frac{1}{\pi a^n}\int_0^{2\pi}{h(\phi)sin(n\phi)d\phi}$, n$\geq$ 1
%                 \end{minipage}}}
% 		\end{enumerate}
% 	\section{Lösung häufiger DGL's}
% 		\[\begin{cases}
% 		\frac{X''(x)}{X(x)}=-\lambda,\quad\lambda\in\mathbb{R}
% 		\end{cases}\]
% 		\begin{enumerate}[label=\textbf{\arabic*.}]
% 		\item $\lambda<0$:
% 		\[ X_\lambda(x)=A_\lambda\sinh(\sqrt{\abs{\lambda}}x)+B_\lambda\cosh(\sqrt{\abs{\lambda}}x)\]
% 		\item $\lambda=0$:
% 		\[ X_0(x)=A_0x+B_0\]
% 		\item $\lambda>0$:
% 		\[ X_\lambda(x)=A_\lambda\sin(\sqrt{\abs{\lambda}}x)+B_\lambda\cos(\sqrt{\abs{\lambda}}x)\]
% 		\end{enumerate}
% 		\vspace{-10pt}
% 		\rule{\columnwidth}{0.7pt}
% 		\[\begin{cases}
% 		\frac{T'(t)}{T(t)}=-\lambda,\quad\lambda\in\mathbb{R}
% 		\end{cases}\]
% 		\begin{enumerate}[label=\textbf{\arabic*.}]
% 		\item $\lambda<0$:
% 		\[ T_\lambda(t)=C_\lambda\cdot e^{-\lambda t}\]
% 		\item $\lambda=0$:
% 		\[ T_0(t)=C_0\]
% 		\item $\lambda>0$:
% 		\[ T_\lambda(t)=C_\lambda\cdot e^{-\lambda t}\]
% 		\end{enumerate}
% 		\vspace{-10pt}
% 		\rule{\columnwidth}{0.7pt}
% 		\[\begin{cases}
% 		\frac{X''(x)}{X(x)}=-\lambda^2,\quad\lambda\in\mathbb{R},\ a<x<b
% 		\end{cases}\]
% 		\begin{enumerate}[label=\textbf{\arabic*.}]
% 		\item $\lambda^2<0$
% 		\[ X_\lambda(x)=A_\lambda\sinh\big(\lambda(x-a)\big)+B_\lambda\sinh\big(\lambda(x-b)\big)\]
% 		\end{enumerate}
% 		\null\vfill
% \end{multicols*}		

% %%% Aus der Zusammenfassung `Analysis 1 + 2' von Silvano Cortesi kopiert %%%

% 	    		\begin{multicols*}{3}	
% 		\section{Vektoranalysis}
% 			\attention\textcolor{red}{Bei +,-,/,* auf Gebieten immer beide Seiten verändern!}\\
% 			Wird jedem Punkt ein Skalar zugeordnet, spricht man von einem \textbf{Skalarfeld}.\\
% 			Wird jedem Punkt ein Vektor zugeordnet, spricht man von einem \textbf{Vektorfeld}.
% 			\subsection{Gradient}
% 				Der Gradient einer Funktion $f(x_1, x_2, \dots, x_n)$ ist der "`Vektor mit allen partiellen Ableitungen \underline{einer} Funktion:\\
% 				$f:U\subset\mathbb{R}^n\to\mathbb{R}$, \underline{NICHT $\mathbb{R}^m$}\\[6pt]
% 				\(\grad f(x)=\nabla f(x_1, x_2, \dots, x_n) = \MATR{
% 					\frac{\partial }{\partial x_1} f(x_1, x_2, \dots, x_n)\\
% 					\frac{\partial }{\partial x_2} f(x_1, x_2, \dots, x_n)\\
% 					\vdots \hspace*{2.2cm}\\
% 					\frac{\partial }{\partial x_n} f(x_1, x_2, \dots, x_n)},\vspace*{-0.6cm} \\ mit\ Nabla\ \nabla\coloneqq\MATR{\frac{\partial }{\partial x_1}\\
% 					\frac{\partial }{\partial x_2}\\
% 					\vdots\\
% 					\frac{\partial }{\partial x_n}} \) als Differentialoperator\\
% 				Der Gradient gibt die gr"osste Steigung im entsprechenden Punkt an.\\
% 				\emph{Spezialfall:} 2 Dimensionen:\\
% 				$\nabla f(x,y) = \MATR{\frac{\partial }{\partial x} f(x,y)	\\	\frac{\partial }{\partial y} f(x,y) } $
% 			\subsection{Divergenz / Quellst"arke}
% 				Die Divergenz eines Vektorfeldes \( \vec v(x) = \MATR{v_1(x_1,x_2,\dots,x_n) \\ v_2(x_1,x_2,\dots,x_n) \\\dots\\ v_m(x_1,x_2,\dots,x_n} \) ist:
% 				$ \text{div}(\vec v) = \nabla \cdot \vec{v} = \dfrac{\partial v_1}{\partial x_1} + \dfrac{\partial v_2}{\partial x_2}+\dots+\dfrac{\partial v_n}{\partial x_n}$

% 				Die Divergenz eines Vektorfeldes kann als Quellst"arke intepretiert werden. Ein Feld mit $div(\vec v)$ kann als \textbf{quellenfrei} betrachtet werden.					
% 			\subsection{Rotation}
% 				Gegeben sei ein Vektorfeld \( \vec v:\Omega\subset\mathbb{R}^3\to\mathbb{R}^3\) von der Klasse $C^1$ mit Komponenten $v_1,v_2,v_3$. Die \textbf{Rotation} von $\vec v$ ist definiert durch:
% 				\[ \text{rot}(\vec v)=\nabla\times\vec v= \MATR{\frac{\partial v_3}{\partial y}-\frac{\partial v_2}{\partial z}\\ \frac{\partial v_1}{\partial z}-\frac{\partial v_3}{\partial x}\\\frac{\partial v_2}{\partial x}-\frac{\partial v_1}{\partial y}} \]
				
% 			\subsection{Laplace-Operator}
% 				Der Laplace Operator f"ur die $C^2$ Funktion von $n$ Raumvariablen ist:\\
% 				\[ \Delta u = \frac{\partial^2 u}{\partial {x_1}^2}+ \dots +\frac{\partial^2 u}{\partial {x_n}^2} = \text{div}( \nabla u ) \]
% 			\subsection{Identit"aten}
% 				\begin{itemize}
% 					\item	\(\text{div}\big(f\cdot K\big) = \nabla f \cdot K + f \cdot \text{div}\big(K\big)\)
% 					\item	\(\text{div}\big(K \times L\big) = L\cdot\text{rot}\big(K\big) - K\cdot \text{rot}\big(L\big)\)
% 					\item	\(\text{rot}(\text{grad}(f))=\text{rot}\big(\nabla f\big) = 0\)
% 					\item	\(\text{div}(\text{grad}(f))= \Delta f \)
% 					\item	\(\text{div}\Big(\text{rot}\big(K \big)\Big) = 0\)
% 					\item	\(\text{div}\Big(f\cdot \text{rot}\big(K \big)\Big) = \nabla f \cdot \text{rot}\big(K\big)\)
% 					\item \( \nabla \times ( \nabla \times \vec{K} ) = \nabla \left( \nabla \cdot \vec{K} \right) - \MATR{\Delta K_1 \\ \Delta K_2 \\ \Delta K_3} \)
% 				\end{itemize}
% 			\subsection{Transformationen}
% 				Kettenregel!
% 				$$\frac{\partial\hat f}{\partial x_i}=\frac{\partial\xi_1}{\partial x_i}\frac{\partial\hat f}{\partial\xi_1}+\dotsm+\frac{\partial\xi_n}{\partial x_i}\frac{\partial\hat f}{\partial\xi_n}$$
% 				\begin{enumerate}
% 					\item Leite die neue Koordinate $\xi_1$, als Ausdruck der alten Koordinaten, nach$x_1$ ab. Setze den neuen Operator $\frac{\partial\hat f}{\partial\xi_1}$ als Faktor dazu.
% 					\item mache das mit allen neuen Koordinaten $\xi$ und addiere sie zusammen
% 					\item ersetze $\xi$ durch die neuen Koordinaten
% 				\end{enumerate}
% 			\subsection{Mehrfachdifferentiation}
% 				$\dfrac{\partial ^2}{\partial x\partial x} f(x,y)=f_{xx}$,\ $\dfrac{\partial ^2}{\partial y\partial y} f(x,y)=f_{yy}$,\ $\underbrace{\dfrac{\partial ^2}{\partial x\partial y}}_{1. y,\ 2. x} f(x,y)=f_{yx}$,\ \dots
% 				\subsubsection{Satz von Schwarz}
% 					Falls $\dfrac{\partial ^2}{\partial x\partial y} f(x,y)$ existiert und stetig, dann gilt:
% 					$$\dfrac{\partial ^2}{\partial x\partial y} f(x,y)=\dfrac{\partial ^2}{\partial y\partial x} f(x,y)$$
					
% \subsection{Potential}
% 				\emph{Definition}:\\
% 				Ein Vektorfeld $\vec v:\Omega\subset\mathbb{R}^n\to\mathbb{R}^n$ heisst \textbf{Potenzialfeld}, falls es eine stetig differenzierbare Abbildung $\Phi:\Omega\subset\mathbb{R}^n\to\mathbb{R}$ gibt (ein Skalarfeld), für welche gilt:
% 				$$\textcolor{red}{\vec v=\nabla\Phi}$$
% 				$\Phi$ wird \textbf{Potenzial} von $\vec v$ genannt.\\
% 				Das Kurvenintegral über $\gamma$ vereinfacht sich dann zu:\\
% 				\[\int_\gamma\vec vd\vec s=\Phi(\vec\gamma(b))-\Phi(\vec\gamma(a)),\ oder\ Kurvenint.\ mit\ einf.\ Param.\] und ist unabhängig vom Wege.
% 				Das Feld wird auch Gradientenfeld oder konservativ genannt. In solch einem ist das Kurvenintegral von $a$ nach $b$ unabh"angig vom Weg. Das geschlossenen Kurvenintegral ist also 0. F"ur ein beliebiges Kurvenintegral von $\gamma_0$ nach $ \gamma_1 $ gilt:
% 				$ \int_\gamma \vec{K}(\vec{x}) d\vec{x} = \Phi(\gamma_1) - \Phi(\gamma_0)  $\\
% 				\textbf{Existenz eines Potentials:}\\
% 				\begin{itemize}
% 					\item in $\mathbb{R}^2$: lösen und sehen ob Existent
% 					\item in $\mathbb{R}^3$: $\text{rot}(\vec v)\meq0$
% 				\end{itemize}
% 				\textbf{Berechnung:}
% 				\begin{enumerate}
% 					\item $\MATR{x_1\\ \vdots\\x_n}\meq\MATR{\frac{\partial\Phi}{\partial x_1}\\\vdots\\\frac{\partial\Phi}{\partial x_1}}$
% 					\item erste Variable integrieren, Integrationskonstante: $C(x_2,\dots,x_n)$
% 					\item $\Phi$ ableiten nach $x_2$, gleichsetzen mit $x_2$, auflösen nach $C'$
% 					\item $C'$ nach $x_2$ integrieren, Integrationskonstante $C(x_3,\dots,x_n)$
% 				\end{enumerate}
% 				\subsubsection{Zusammenfassung Potenzialfelder}
% 					\textit{Sei $\vec v:\Omega\subset\mathbb{R}^n\to\mathbb{R}^n$ ein stetig diff'bares Vektorfeld mit \textbf{einfach zusammenhängendem} Definitionsbereich $\Omega$. Dann sind äquivalent:} 
% 					\begin{enumerate}
% 						\item \textcolor{red}{$\vec v$ erfüllt die Integrabilitätsbedingungen auf $\Omega$ ($\frac{\partial v_i}{\partial x_j}=\frac{\partial v_j}{\partial x_i},\ \forall i\neq j$)}
% 						\item \textcolor{red}{rot$(\vec v)=0$}
% 						\item $\vec v$ ist ein Potenzialfeld
% 						\item $\vec v$ ist konservativ
% 						\item $\oint_\gamma\vec v\cdot d\vec s=0$
% 						\item Das Integral $\int_\gamma\vec v\cdot d\vec s$ ist wegunabhängig
% 					\end{enumerate}
				
% 			\subsection{Flussintegrale}
% 				Wir sagen ein Flächenstück $S$ ist \textbf{orientierbar}, falls man klar sagen kann was innen/aussen oder oben/unten ist.Das Möbiusband ist z.B. nicht orientierbar. Mathematisch:
% 				$$Fuer\ \gamma\ geschlossen:\ \gamma:[a,b]\to S\ muss\ gelten:\ \vec n(\gamma(a))=\vec n(\gamma(b))$$
% 				\textbf{Der Flussbegriff}\\[4pt]
% 				Es seien $\vec v$ ein Vektorfeld und $S\subset\mathbb{R}^3$ ein mittels $\vec n$ orientiertes Flächenstück. Die Grösse
% 				\ceqbox{\int_S\vec v\cdot\vec ndo,\hspace{10pt}\highlightbg{\vec n=\pm\frac{\Phi_u\times\Phi_v}{\abs{\Phi_u\times\Phi_v}}}}
% 				wird \textbf{Fluss} von $\vec v$ durch $S$ genannt. Der Ausdruck $\vec ndo$ wird \textbf{orientiertes Flächenelement} genannt.
% 				\subsubsection{Kochrezept für Flussintegrale}
% 					\textit{Gegeben: Vektorfeld $\vec v$, Fläche S}\\
% 					\textit{Gesucht: Flussintegral $\iint_S\vec v\cdot\vec ndo$}
% 					\begin{enumerate}
% 						\item Parametrisiere die Fläche $S$, d.h. finde:
% 						$$\Phi:[a,b]\times[c,d]\to\mathbb{R}^3,\ (u,v)\to\Phi(u,v)=(\Phi_1(u,v),\Phi_2(u,v),\Phi_3(u,v))$$
% 						\item Berechne $\Phi_u=\frac{\partial\Phi}{\partial u}$ und $\Phi_v=\frac{\partial\Phi}{\partial v}$ indem du jede Komponente von $\Phi$ nach $u$ respektive $v$ partiell ableitest.\\ Berechne ferner das Kreuzprodukt: $\Phi_u\times\Phi_v$
% 						\item Benutze die Formel
% 						$$\int_S\vec v\cdot\vec ndo=\pm\int_a^b\int_c^d\vec v(\Phi(u,v))\cdot(\Phi_u\times\Phi_v)dudv$$ Entscheide das Vorzeichen je nach Situation (Richtung des $\times$)
% 						\item Integriere das ganze, je nachdem mit Hilfe von Substitution.
% 					\end{enumerate}
				
% 			\subsection{Satz von Gauss/Divergenzsatz $(Flussint \rightarrow Volumenint)$}
% 				\textit{Es sei ein beschränkter räumlicher Bereich V mit Rand $\partial V\in C^1_{pw}$ gegeben, es sei das Vektorfeld $\vec v$ auf ganz V definiert und stetig diff'bar. Dann gilt:}	\\[6pt]
% 				\eqbox{\int_{\partial V}\vec v\cdot\vec ndo=\int_Vdiv(\vec v)d\mu}\\[6pt]
% 				\attention\textit{Wobei $\vec n$ die nach \textcolor{red}{\textbf{aussen}} gerichtete Normale längs $\partial V$ ist.}
% 				\subsubsection{Berechnen von Volumina mit Gauss}
% 					Wir betrachten das geschlossen Volumen $\Omega\subset\mathbb{R}^3$ mit Rand $\partial\Omega$ und nach aussen gerichteter Normale $\vec n$. Wir betrachten das Vektorfeld $\vec v=(x,y,z)=\vec x$\\
% 					Natürlich gilt: $\text{div}(\vec v)=\frac{\partial x}{\partial x}+\frac{\partial y}{\partial y}+\frac{\partial z}{\partial z}=1+1+1=3$\\
% 					Nach Satz von Gauss folgt:
% 					$$\int_{\partial V}\vec x\cdot\vec ndo=\int_\Omega \text{div}(\vec x)d\mu=\int_\Omega 3d\mu=3\mu(\Omega)$$
% 					\vspace*{-0.2cm}$\rightarrow\vec v$ kann auch $(x,0,0)\ ,\ \dots$ sein. Dann nur $\mu(\Omega)$\vspace*{0.1cm}
% 				\subsubsection{Satz von Gauss in der Ebene}\vspace*{-0.2cm}
% 					Es seien $\gamma\in C^1_{pw}$ eine stückweise stetg diff'bare Kurve in $\mathbb{R}^2$ und $\vec v=(v_1,v_2)$ ein stetig diff'bares Vektorfeld. $\dot{\vec \gamma}$ ist tangential zur Kurve, wir brauchen $\vec n$ so, dass $(\vec n,\dot{\vec\gamma})$ positiv orientiert ist.
% 					\begin{center}\includegraphics[width=0.5\columnwidth]{images/kurve_neg_pos.png}\end{center}
% 					Für eine Parametrisierung $\gamma:[a,b]\to\mathbb{R}^2$ der Kurve gilt folgende Formel:
% 					$$\int_\gamma\vec v\cdot \vec nds=\int_a^b\vec v(\vec\gamma(t))\cdot\vec n(t)dt$$
% 					$\vec n\coloneqq$\textit{Zeilen von $\gamma$ vertauschen, obere $\bullet-1$, Richtung prüfen}\\[6pt]
% 					\textbf{Gauss:}\\[4pt]
% 					\textit{Für einen regulären, beschränkten ebenen Bereich C mit orientiertem Rand $\gamma$ und ein stetig differenzierbares Vektorfeld $\vec v$ gilt:}\\[4pt]
% 					\eqbox{\int_\gamma\vec v\cdot \vec nds=\int_C\text{div}(\vec v)do}
% 			\subsection{Integralsatz: Satz von Green in \(\mathbb{R}^2\)}
% 				\textit{Es seien $\vec v=(v_1,v_2)$ ein stetig diff'bares Vektorfeld auf einem Gebiet $\Omega\subset\mathbb{R}^2$ und $C\subset\Omega$ ein beschränkter Bereich mit $C^1_{pw}\ Rand\ \partial C$, dann gilt:}\\[4pt]
% 				\eqbox{\int_{\gamma=\partial C}\vec v\cdot d\vec s=\int_C\Big(\frac{\partial v_2}{\partial x}-\frac{\partial v_1}{\partial y}\Big)dxdy=\int_C\text{rot}(\vec v)dxdy}
% 			\subsection{Integralsatz: Satz von Stokes in \(\mathbb{R}^3\)}
% 				\textit{Es seien $\vec v=(v_1,v_2,v_3)$ ein stetig diff'bares Vektorfeld auf einem Gebiet $\Omega\subset\mathbb{R}^3$ und $C\subset\Omega$ eine offene Fläche durch die geschlossene $C^1_{pw}$ Kurve $\gamma=\partial C$ berandet (in positiver Richtung*), dann gilt:\\ *positive Richtung: Wenn man entlang $\vec n$ den \textbf{Rand} durchläuft, sieht C links von sich}\\[4pt]
% 				\vspace*{-10pt}\begin{center}\eqbox{\int_{\gamma=\partial C}\vec v\cdot d\vec s=\int_C\text{rot}(\vec v)\cdot\vec ndo}\end{center}
				
				
				
% 		\section{Fourier-Reihen}
% 			Eine periodische Funktion $f$ lässt sich als \textit{Fourier-Reihe} darstellen:\\
% 			\eqbox{f(t)=\frac{a_0}{2}+\sum\limits_{k=1}^{\infty}{a_k\cos{\left(k\frac{2\pi}{T}t\right)}+b_k\sin{\left(k\frac{2\pi}{T}t\right)}}}\\
% 			\textit{Komplexe Fourier-Reihe:} \eqbox{f(t)=\sum\limits_{k=-\infty}^{\infty}{c_ke^{ik\frac{2\pi}{T}t}}}\\
% 			$T=$ Fundamentalperiode von $f(t)$
% 			\begin{itemize}
% 				\item $a_k=\frac{2}{T}\int_{t_0}^{t_0+T}{f(t)\cdot\cos{\left(k\frac{2\pi}{T}t\right)}}dt$
% 				\item $b_k=\frac{2}{T}\int_{t_0}^{t_0+T}{f(t)\cdot\sin{\left(k\frac{2\pi}{T}t\right)}}dt$
% 				\item $c_k=\frac{1}{T}\int_{t_0}^{t_0+T}{f(t)\cdot e^{-ik\frac{2\pi}{T}t}}dt$
% 			\end{itemize}
% 			\subsection{Gerade Fortsetzung}
% 				\begin{itemize}
% 					\item $b_k=0$
% 					\item $a_k=\frac{4}{T}\int_{0}^{T/2}{f(t)\cdot\cos{\left(k\frac{2\pi}{T}t\right)}}dx$
% 					\item Kosinus-Reihe
% 				\end{itemize}
% 			\subsection{Ungerade Fortsetzung}
% 				\begin{itemize}
% 					\item $a_0=0; a_k=0$
% 					\item $b_k=\frac{4}{T}\int_{0}^{T/2}{f(t)\cdot\sin{\left(k\frac{2\pi}{T}t\right)}}dt$
% 					\item Sinus-Reihe
% 				\end{itemize}
% 			\subsection{Koeffizienten-Umrechnung}
% 				\begin{itemize}\vspace{-3pt}
% 					\item $c_k=\begin{cases} \frac{1}{2}(a_k-ib_k) & k>0 \\ \frac{1}{2}(a_k+ib_k) & k<0 \\ \frac{a_0}{2} & k=0 \end{cases}$
% 					\item $a_k=c_k+c_{-k}$
% 					\vspace{-5pt}	\item $a_0=2c_0,\hspace{10pt} b_k=i(c_k-c_{-k})$
% 				\end{itemize}
% 			    		\end{multicols*}
% 			    		\begin{multicols*}{3}	
% 		\section{Differentialrechnung einer Variablen}
% 			\subsection{Ableitung}
% 				Eine Funktion $ f $ heiss \emph{differenzierbar in $ x_0 \in X $}, falls gilt\\
% 				$ \exists a \in \mathbb{R}: f(x)=f(x_0)+a \cdot (x-x_0) + o(x-x_0) $.\\
% 				Die Zahl $ a $ ist dann eindeutig bestimmt und es gilt\\
% 				$ a = f'(x_0) = \lim \limits_{x \to x_0} \frac{f(x)-f(x_0)}{x-x_0}= \lim \limits_{h \to 0} \frac{f(x_0 + h)-f(x_0)}{h} $ ($ h=x-x_0$)\\
% 				Die Funktion $ f $ heisst \emph{differenzierbar}, falls sie in jedem Punkt von $ X $ differenzierbar ist. Dann gilt\\
% 				$ f'(x)= \frac{dy}{dx} = \lim \limits_{x \to x_0} \frac{\Delta y}{\Delta x} $.
				
% 				\textbf{Rechenregeln}\\
% 				\( (f + g)' = f' + g' \)\\
% 				\( (cf)' = cf' \ , c\in\mathbb{R}\)\\
% 				\( (fg)' = f'g+fg' \)\\
% 				$ \left( \frac{f}{g} \right) ' = \frac{f'g-fg'}{g^2} $\\
% 				\( (g \circ f)'(x) = (g(f(x)))' = g'(f(x)) \cdot f'(x) \) 					
% 				\subsection{Wichtige Ableitungen}	
% 					\begin{tabular}{|c|c|}
% 					\hline
% 					$f(x)$ & $f'(x)$ \\
% 					\hline
% 					$x^\alpha,\ \alpha\in\mathbb{R}$ & $\alpha\cdot x^{\alpha-1}$\\
% 					\hline
% 					$e^x$ & $e^x$\\
% 					\hline
% 					$\sin{(x)}$ & $\cos{(x)}$\\
% 					\hline
% 					$\cos{(x)}$ & $-\sin{(x)}$\\
% 					\hline
% 					$\tan{(x)}$ & $\frac{\cos^2(x)+\sin^2(x)}{\cos^2(x)}=\frac{1}{\cos^2(x)}=1+\tan^2(x)$\\
% 					\hline
% 					$\ln(x)$ & $\frac{1}{x}$\\
% 					\hline
% 					$Konstante$ & $0$\\
% 					\hline
% 					\end{tabular}\\[10pt]
% 				\textbf{Weitere:}\\
% 					\begin{multicols*}{2}
% 				$\frac{d}{dx}(c)=0$\\
% 				$\frac{d}{dx}(x^\alpha) = \alpha x^{\alpha-1}, \alpha\in\mathbb{R}$\\
% 				$\frac{d}{dx}(\sqrt{x})=\frac{1}{2\sqrt{x}}$\\
% 				$\frac{d}{dx}(a^x)=\ln{(a)}a^x$\\
% 				$\frac{d}{dx}(\sin{x})=\cos{x}$\\
% 				$\frac{d}{dx}(\cos{x})=-\sin{x}$\\
% 				$\frac{d}{dx}(\ln{(x)})=\frac{1}{x}$\\
% 				$\frac{d}{dx}\left(\log{\left( (f(x)\right)}\right)=\frac{f'(x)}{f(x)}$\\
% 				$\frac{d}{dx}(e^z)=e^z$\\
% 				$\frac{d}{dx}(\cot{(x)})=-\frac{1}{\sin^2{(x)}}$\\
% 				$\frac{d}{dx}(\arccos{(x)})=\frac{-1}{1-x^2}$\\
% 				$\frac{d}{dx}(\arctan{(x)})=\frac{1}{1+x^2}$\\
% 				$\frac{d}{dx}(\arccot{(x)})=\frac{-1}{1+x^2}$\\
% 				$\frac{d}{dx}(x^x)=x^x(\ln{(x)}+1)$\\
% 				$\frac{d}{dx}(\log_a{(x)})=\frac{1}{\ln{(a)}\cdot x}$\\
% 				$\frac{d}{dx}(\cosh{(x)})=\sinh{(x)}$\\
% 				$\frac{d}{dx}(\sinh{(x)})=\cosh{(x)}$\\
% 				$\frac{d}{dx}(\arsinh{(x)})=\frac{1}{x^2+1}$\\
% 				$\frac{d}{dx}(\artanh{(x)})=\frac{1}{1-x^2}$
% 			\end{multicols*}
% 			$\frac{d}{dx}(\tan{x})=1+\tan^2{(x)}=\frac{1}{\cos^2{(x)}}=\frac{1}{1-\sin^2{(x)}}$\\
% 			$\frac{d}{dx}(\arcsin{(x)})=\frac{1}{\cos{(\arcsin{(x)})}}=\frac{1}{\sqrt{1-x^2}}$\\
% 			$\frac{d}{dx}(\arcosh{(x)})=\frac{1}{\sqrt{x^2-1}}$\\
% 			$\frac{d}{dx}(\tanh{(x)})=\frac{1}{\cosh^2{(x)}}=1-tanh^2{(x)}$
% 			\subsection{Kurvendiskussion}
% 				ACHTUNG: Immer Definitionsbereich bzw. Randpunkte beachten!
				
% 				\emph{Begriffe}
				
% 				\begin{tabular}{l}
% Eine Funktion ist konkav (konvex), wenn die Sekante durch je zwei\\Punkte $P_1$ und $P_2$ des Graphen von $f$ unterhalb (oberhalb) des\\Graphen liegt.\\
% 				\end{tabular}
% 				\begin{tabular}{ll}
% 					\textbf{Konvex}				&	\(f''\geqslant 0\)								\\
% 					\textbf{Konkav}				&	\(f''\leqslant 0\)								\\
% 					\\
		
% 					Lokale Extrema:\\
% 					\textbf{lokales Maximum}		&	\(f'=0\) und \(f''<0\)							\\
% 					\textbf{lokales Minimum}		&	\(f'=0\) und \(f''>0\)							
% 						\end{tabular}\\
% 			\begin{tabular}{ll}
% 					\textbf{Wendepunkte}		&	\(f''=0\)									\\
% 					\textbf{Sattelpunkte}			&	\(f'=0\) und \(f''=0\)							\\
% 					\\
					
% 					Globale Extrema:\\
% 					\textbf{Globales Maximum}	&	\(\max(\text{lokale Maxima}, \text{Randpunkte})\)	\\
% 					\textbf{Globales Minimum}		&	\(\min(\text{lokale Minima}, \text{Randpunkte})\)		\\
% 				\end{tabular}
% 			\subsection{Umkehrsatz}	
% 			\eqbox{\left(f^{-1}\right)'\left(y\right)= \frac{1}{f'\left(f^{-1}(y)\right)}= \left(f^{-1}\right)'\left( f(x) \right)= \frac{1}{f'\left( x \right)}}\\[4pt]
% 			\textbf{Beispiel 1}\\
% 				$ f(x) = y = e^x$, $ f^{-1}(y) = x = \log(y)$\\
% 				$y' = e^x$\\
% 				Gesucht: $x'=\left(f^{-1}(y)\right)'$\\
% 					$ x'=\left( f^{-1} \right)'(y) = \left( \log(y) \right)'=\frac{1}{y'}=\frac{1}{e^x} = \frac{1}{e^{\log(y)}}=\frac{1}{y} $
% 				\subsection{Zwischenwertsatz}
% 				Ist $f$ auf $[a,b]$ stetig, so nimmt $f$ jeden Wert zwischen $f(a)$ und $f(b)$ und sein Maximum
% 				und Minimum an einer Stelle in $[a,b]$ an. Das Bild von $[a,b]$ ist abgeschlossen und beschränkt.			
% 			\subsection{Mittelwertsatz}
% 				Eine Funktion $ f: [a,b] \to \mathbb{R} $ ist stetig und auf $ ]a,b[ $ differenzierbar. Dann gilt:\\
% 				$ \exists t \in (a,b): f'(t) = \frac{f(b)-f(a)}{b-a} $
	
			
% 			\subsection{Tangente an einen Punkt \(a\) legen}
% 				\(\to\) \emph{Taylorpolynom 1. Ordnung} mit Entwicklungspunkt \(a\) ausrechnen.
				

% 		\section{Integralrechnung einer Variablen}
% 			{\Huge\textbf{\textcolor{red}{+C nicht vergessen!!!}}}
% 			\subsection{Rechenregeln}
% 			\includegraphics[width=\columnwidth]{images/Image7-2.pdf}
% 			$\int\limits_a^bf(x)dx:= F(b)-F(a)=[F(x)]_a^b=-\int\limits_b^af(x)dx$
% 			\subsection{Elementare Integrale}
% 			\begin{tabular}{|r|l|}
% 			\hline
% 			$f(x)\ (Integrand)$ & $F(x)\ (Stammfunktion)$\\
% 			\hline
% 			$x^\alpha\ (\alpha\neq-1)$ & $\frac{x^{\alpha+1}}{\alpha+1}+C$\\
% 			\hline
% 			$\frac{1}{x}$ & $\log(x)+C$\\
% 			\hline
% 			$e^x$ & $e^x + C$\\
% 			\hline
% 			$\alpha^x$ & $\frac{\alpha^x}{\log(\alpha)}$\\
% 			\hline
% 			$\sin(x)$ & $-\cos(x)+C$\\
% 			\hline
% 			$\cos(x)$ & $\sin(x)+C$\\
% 			\hline
% 			$\sinh(x)$ & $\cosh(x)+C$\\
% 			\hline
% 			$\cosh(x)$ & $\sinh(x)+C$\\
% 			\hline
% 			$\frac{1}{\sqrt{1-x^2}}$ & $\arcsin(x)+C$\\
% 			\hline
% 			$\frac{-1}{\sqrt{1-x^2}}$ & $\arccos(x)+C$\\
% 			\hline
% 			$\frac{1}{{1+x^2}}$ & $\arctan(x)+C$\\
% 			\hline
% 			$\frac{1}{\sqrt{1+x^2}}$ & $\arsinh(x)+C$\\
% 			\hline
% 			$\frac{1}{\sqrt{x^2-1}}$ & $\arcosh(x)+C$\\
% 			\hline
% 			$\frac{1}{{1-x^2}}$ & $\artanh(x)+C$\\
% 			\hline
% 			\end{tabular}\\
% 			\textbf{Weitere:}\vspace*{-0.3cm}
% 			\begin{multicols*}{2}
% 				$\int{x^z}dx=\frac{1}{z+1}\cdot x^{z+1}$\\
% 				$\int{\frac{1}{x-a}}dx=\ln{(|x-a|)}$\\
% 				$\int{e^{\lambda x}}dx=\frac{1}{\lambda}\cdot e^{\lambda x}$\\
% 				$\int{\ln{(x)}}dx=x(\ln{(x)}-1)$\\
% 				$\int{\sin{x}}dx=-\cos{x}$\\
% 				$\int{\cos{x}}dx=\sin{x}$\\
% 				$\int{\frac{1}{\sqrt{1-x^2}}}dx=\arcsin{(x)}$\\
% 				$\int{\sinh{(x)}}dx=\cosh{(x)}$\\
% 				$\int{\cosh{(x)}}dx=\sinh{(x)}$\\
% 				$\int{\frac{1}{\cosh^2{(x)}}}dx=\tanh{(x)}$\\
% 				\mbox{$\int{\frac{1}{\sqrt{1+x^2}}}dx=\sinh^{-1}{(x)}$}\\
% 				\mbox{$\int{\frac{1}{\sqrt{x^2-1}}}dx=\cosh^{-1}{(x)}$}\\
% 				\mbox{$\int{\frac{1}{1-x^2}}dx=\frac{1}{2}\ln{\left(\left|\frac{x+1}{x-1}\right|\right)}$}\\
% 				$\int{\frac{1}{\cos^2{(x)}}}dx=\tan{(x)}$\\
% 				\mbox{$\int{\tan{(x)}}dx=-\ln{|\cos{(x)}|}$}\\
% 				\mbox{$\int{\frac{1}{\tan{(x)}}}dx=\ln{(\sin{(x)})}$}\\
% 				$\int{\sqrt{x}}dx=\frac{2}{3}x^{\frac{3}{2}}$\\
% 				$\int{\frac{1}{x^2}}dx=\frac{-1}{x}$\\
% 				$\int{\frac{1}{x+a}}dx=\ln{|x+a|}$\\
% 				$\int{\frac{1}{(x+a)^n}}dx=\frac{-1}{(n-1)(x+a)^{n-1}}$\\
% 				$\int{\frac{1}{x^2+a^2}}dx=\frac{1}{a}\tan^{-1}{\left(\frac{x}{a}\right)}$\\
% 				$\int{\frac{1}{1+x^2}}dx=\tan^{-1}{(x)}$
% 				\end{multicols*}\vspace{-0.5cm}
% 				$\int{\sin^2{(x)}}dx)=\frac{x}{2}-\frac{\sin{(x)}\cos{(x)}}{2}$\\
% 				$\int{\cos^2{(x)}}dx)=\frac{x}{2}+\frac{\sin{(x)}\cos{(x)}}{2}$\\
% 				$\int{\sin^{-1}{(x)}}dx=\sqrt{1-x^2}+x\sin^{-1}{(x)}$\\
% 				$\int{\sinh^{-1}{(x)}}dx=x\sinh^{-1}{(x)}-\sqrt{1+x^2}$\\
% 				$\int{\cos^{-1}{(x)}}dx=x\cos^{-1}{(x)}-\sqrt{1-x^2}$\\
% 				$\int{\cosh^{-1}{(x)}}dx=x\cosh^{-1}{(x)}-\sqrt{x^2-1}$\\
% 				$\int{\tan^{-1}{(x)}}dx=x\tan^{-1}{(x)}-\frac{1}{2}\ln{(x^2+1)}$\\
% 				$\int{\tanh^{-1}{(x)}}dx=x\tanh^{-1}{(x)}+\frac{1}{2}\ln{(1-x^2)}$\\
% 				$\int{\frac{1}{\sqrt{a-x^2}}}dx=\tan^{-1}{\left(\frac{x}{\sqrt{a-x^2}}\right)}$\\
% 				$\int{\frac{1}{\sinh{(x)}}}dx=\ln{\left(e^x-1\right)}-\ln{\left(e^x+1\right)}$\\
% 				$\int{\frac{1}{\sin{(x)}}}dx=\ln{\left(\frac{\sin{(x)}}{\cos{(x)}+1}\right)}$\\
% 				$\int{\frac{1}{\sqrt{1-x}}}dx=-2\sqrt{1-x}$\\
% 				$\int{\frac{1}{\cos{(x)}}}dx=\ln{\left(\frac{-\cos{(x)}}{\sin{(x)}-1}\right)}$\\
% 				$\int{2\sin{(x)}\cos{(x)}}dx=-\cos^2{(x)}$\\
% 				$\int{\frac{1}{\sin^2{(x)}}}dx=-\frac{1}{\tan{(x)}}$\\
% 				$\int{x^s\ln{(x)}}dx=\frac{x^{s+1}}{s+1}\left( \ln{(x)}-\frac{1}{s+1} \right)$\\
% 				\mbox{$\int{\tanh{(x)}}dx=\ln{(e^{2x}+1)-x}$}\\
% 				$\int{\frac{\ln{(x)}}{x}}dx=\frac{1}{2}(\ln{(x)})^2$\\
% 				$\int_0^{2\pi}\sin(t)\cos(t)dt=\int_0^{2\pi}\sin(t)dt=\int_0^{2\pi}\cos(t)dt=0$\\	
% 				\textbf{Gauss-Integral:} $\int_{-\infty}^{\infty}{e^{-x^2}}dx=\sqrt{\pi}$\\
% 				\textbf{Elliptisches Integral:} $\int_{0}^{2\pi}{\sqrt{\alpha^2\sin^2t+\beta^2\cos^2t}}dt$\\
% 				$\rightarrow$ nur numerisch bestimmbar oder $=2\alpha\pi$ falls $\alpha=\beta$
% 			\subsection{Riemannsches Summe}
% 			$\overbrace{\eqbox{\int \limits_a^b f(x)dx:= \lim \limits_{n\to\infty} \sum \limits_{k=1}^n f(\xi_k)\Delta_k}}^{\text{zum Ber. von Integralen via Summe, $\Delta_k$ ist die Feinheit(Länge des max. Interv.)}}$\\[4pt]
% 				$\underbrace{\eqbox{\limni\frac{1}{a}\smashoperator{\sum \limits_{k=1}^{n}}\frac{1}{n}\cdot f\Big(\frac{a\cdot k}{n}\Big)=\frac{1}{a}\int \limits_0^a f(x) dx}}_{\text{zum Berechnen von Summen via Integral}}$\\
% 				\textbf{Beispiel:}\\
% 				$\int\limits_0^1(x^3-2x)dx$\\
% 				\textit{Wir unterteilen [0,1] in n gleich grosse Teilintervalle der Länge $\Delta x=\frac{1}{n}$ mit $x_k=\frac{k}{n}, k=1,\dots,n$, die Feinheit ist $\frac{1}{n}$ und es gilt $\limni\Delta_n=0$\\
% 				$\Rightarrow \int\limits_0^1(x^3-2x)dx=\limni\sum\limits_{k=1}^nf(\frac{k}{n})\frac{1}{n}=\limni\sum\limits_{k=1}^n\Big(\frac{k^3}{n^3}-2\frac{k}{n}\Big)$}
			
% 			\subsection{Einfache Integrale}
% 				\textbf{Allgemein}
% 				\begin{itemize}
% 				\item$\int f(a+x)dx=F(a+x)+c$
% 				\item$\int f(-x)dx=-F(-x)+c$
% 				\item$\int f(ax+b) \text{ d}x = \dfrac{1}{a} F(ax+b) + c $
% 				\item$\int f(a-x)dx=-F(a-x)+c$
% 				\item$\int \dfrac{f'(x)}{f(x)} \text{ d}x = \log(\vert f(x) \vert ) + c$
% 				\item$\int f(ax)dx=\frac{1}{a}F(ax)+c$
% 				\item$\int g(x)\cdot g'(x)dx=\frac{1}{2}g(x)^2$
% 				\item$\int\limits_a^b f(x)dx=-\int\limits_b^af(x)dx$
% 				\end{itemize}
% 			\subsection{Direkte Integrale}
% 			\eqbox{F(g(x))=\int f(g(x))g'(x)dx=\int(F\circ g)'(x) dx}\\[4pt]
% 			\textbf{Bsp.:}\\
% 			$\int\frac{dx}{x\log(x)}=\int\overbrace{\frac{1}{\log x}}^{\textit{f(g(x))}}\cdot\overbrace{\frac{1}{x}}^{\textit{g'(x)}}dx=\log(\log(x))+C$
% 			\subsection{Partielle Integration}

% 				\eqbox{ \int \underbrace{f(x)}_{\downarrow} \underbrace{g'(x)}_{\uparrow} dx = f(x) g(x) - \int f'(x) g(x) dx }
				
% 				\emph{Typische Anwendung}: Integrale, die immer wieder sich selbst ergeben!\\
% 				Bsp.: \(\displaystyle \int e^{-2x} \cdot \sin(x) \text{ d}x\)\\
% 				\(\implies\) So lange partielle Integration anwenden (hier zwei Mal) bis man wieder das Integral selbst erhält (plus die
% 				ganzen Überreste der partiellen Integration). Umformen, dass das Integral nur noch auf einer Seite steht. Fertig!
% 				\begin{tabular}{|r|l|}
% 				\hline
% 				$\uparrow$ & 1 (falls arc-Funktion oder Logarithmus vorkommt), \\ & $x^n,\frac{1}{1-x^2}, \frac{1}{1+x^2},\dots$\\
% 				\hline
% 				$\downarrow$ & $x^n, \log(x), \arcsin x, \arccos x, \arctan x, \arsinh x, \arcosh x,\dots$\\
% 				\hline
% 				``egal'' & $e^x, \sin x, \cos x, \sinh x, \cosh x,\dots$\\
% 				\hline
% 				\end{tabular}
% 			 \subsection{Integrale rationaler Funktionen}
% 			Funktionen in Form: $\int \frac{p(x)}{q(x)}dx$
% 			\begin{itemize}
% 				\item Falls Grad$(p)\ge$ Grad$(q)\Rightarrow$ Polynomdivision
% 				\item Falls Grad$(p)<$ Grad$(q)\Rightarrow$ Partialbruchzerlegung					
% 			\end{itemize}			
% 			\subsection{Substitution}
% 				\eqbox{ \int \limits_a^b f(\varphi (x)) \cdot \varphi' (x)dx = \int \limits_{\varphi(a)}^{\varphi(b)} f(t)dt }
% 				\begin{enumerate}
% 					\item $ \varphi (x) \coloneqq t$, $ \varphi '(x)dx \coloneqq dt$ substituieren. (Siehe Anhang)
% 					\item Grenzen $a$, $b$ durch $\varphi(a)$, $\varphi(b)$ ersetzen.
% 					\item Integrieren.
% 				\end{enumerate}
% 				\subsubsection{Integrale mit $ e^x,\sinh x, \cosh x, \dots$}
% 				Substitution: $e^x=t,\ (dx=\frac{1}{t}dt)$
% 				\subsubsection{Integrale mit $\log x$}
% 				Substitution: $\log x=t,\ (dx=e^t dt)$
% 				\subsubsection{Integrale mit $\sqrt[\alpha]{Ax+B}$}
% 				Substitution: $\sqrt[\alpha]{Ax+B}=t$
% 				\subsubsection{Integrale mit $\cos x,\sin x$ in geraden Potenzen oder $\tan x$}
% 				Substitution: $\tan x=t, (dx=\frac{1}{1+t^2}dt)$\\
% 				$\Rightarrow \sin^2x=\frac{t^2}{1+t^2},\ \cos^2x=\frac{1}{1+t^2}$
% 				\subsubsection{Integrale mit $\cos x,\sin x$ in ungeraden Potenzen}
% 				Substitution: $\tan\Big(\frac{x}{2}\Big)=t, (dx=\frac{2}{1+t^2}dt)$\\
% 				$\Rightarrow \sin x=\frac{2t}{1+t^2},\ \cos x=\frac{1-t^2}{1+t^2}$
% 				\subsubsection{Integrale mit $\sqrt{Ax^2+Bx+C}$ im \underline{Nenner}}
% 				Mithilfe quadratischer Ergänzung auf:\\
% 				$\int\frac{1}{\sqrt{1-x^2}}dx=\arcsin x + C$\\
% 				$\int\frac{1}{\sqrt{x^2-1}}dx=\arcosh x + C$\\
% 				$\int\frac{1}{\sqrt{1+x^2}}dx=\arsinh x + C$
% 				\subsubsection{Integrale mit $\sqrt{Ax^2+Bx+C}$ im \underline{Zähler}}
% 				Mithilfe quadratischer Ergänzung auf:\\
% 				$\int\sqrt{1-x^2}dx\ \Rightarrow\ Substitution\ x=\sin t$\\
% 				$\int\sqrt{x^2-1}dx\ \Rightarrow\ Substitution\ x=\cosh t$\\
% 				$\int\sqrt{1+x^2}dx\ \Rightarrow\ Substitution\ x=\sinh t$
% 			\subsection{Grad $\rightarrow$ Radian}
% 			$Radian=Grad\cdot\frac{\pi}{180^\circ}$,\hspace{10pt}$Grad=Radian\cdot\frac{180^\circ}{\pi}$
% 			\subsection{Trigonometrische Grössen}
% 			\renewcommand{\arraystretch}{1.5}
% 			\begin{tabular}{|c|c|c|c|c|c|c|c|c|}
% 				\hline
% 					 \(\varphi \) &$0$ & \(\frac{\pi}{6}\) & \(\frac{\pi}{4}\) & \(\frac{\pi}{3}\) &  \(\frac{\pi}{2}\) &  \(\frac{2\pi}{3}\) &  \(\frac{3\pi}{4}\) & \(\frac{5\pi}{6}\) \\
% 					 \hline
% 					  Grad & $0^\circ$ & $30^\circ$ & $45^\circ$ & $60^\circ$ & $90^\circ$ & $ 120^\circ$ & $135^\circ$ & $150^\circ$\\
% 				  \hline
% 					\(\sin(\varphi)\) & $0$ & \(\frac{1}{2}\) & \(\frac{\sqrt{2}}{2}\) & \(\frac{\sqrt{3}}{2}\) & $1$ &  \(\frac{\sqrt{3}}{2}\) &  \(\frac{1}{\sqrt{2}}\) &  \(\frac{1}{2}\)   \\
% 					 \hline
% 					\(\cos(\varphi)\) &$1$ & \(\frac{\sqrt{3}}{2}\) & \(\frac{\sqrt{2}}{2}\) & \(\frac{1}{2}\) & $0$ & $-\dfrac{1}{2}$ & \(-\frac{1}{\sqrt{2}}\) &  \(-\frac{\sqrt{3}}{2}\)  \\
% 					 \hline
% 					 \(\tan(\varphi)\) &$ 0$ & \(\frac{1}{\sqrt{3}}\) &  $1$ & $\sqrt{3}$ & $\pm \infty$ &$-\sqrt{3}$ & $-1$ &  \(-\frac{1}{\sqrt{3}}\)\\ 
% 					 \hline
% 			\end{tabular}\\
			
% 			\begin{tabular}{|c|c|c|c|c|c|c|c|}
% 				\hline
% 					 \(\varphi \) &$\pi$ & \(\frac{7\pi}{6}\) & \(\frac{5\pi}{4}\) & \(\frac{4\pi}{3}\) &  \(\frac{3\pi}{2}\) &  \(\frac{5\pi}{3}\) &  \(\frac{7\pi}{4}\)  \\
% 					 \hline
% 					 Grad & $180^\circ$ & $210^\circ$ & $225^\circ$ & $240^\circ$ & $270^\circ$ & $ 300^\circ$ & $315^\circ$\\
% 				  \hline
% 					\(\sin(\varphi)\) & $0$ & \(-\frac{1}{2}\) & \(-\frac{\sqrt{2}}{2}\) & \(-\frac{\sqrt{3}}{2}\) & $-1$ &  \(-\frac{\sqrt{3}}{2}\) &  \(-\frac{1}{\sqrt{2}}\)  \\
% 					 \hline
% 					\(\cos(\varphi)\) &$-1$ & \(-\frac{\sqrt{3}}{2}\) & \(-\frac{\sqrt{2}}{2}\) & \(-\frac{1}{2}\) & $0$ & $\dfrac{1}{2}$ & \(\frac{1}{\sqrt{2}}\)  \\
% 					 \hline
% 					 \(\tan(\varphi)\) &$ 0$ & \(\frac{1}{\sqrt{3}}\) &  $1$ & $\sqrt{3}$ & $\pm \infty$ &$-\sqrt{3}$ & $-1$ \\ 
% 					 \hline
% 			\end{tabular}\\	
% 			\subsection{Trigonometrische Funktionen}
% 				\begin{itemize}
% 				\item \textbf{Definitionen}\\
% 					$ \sin(x) = \frac{1}{2i}(e^{ix}-e^{-ix}) = \smashoperator{\sum \limits_{n=0}^{\infty}} (-1)^n \frac{x^{2n+1}}{(2n+1)!}
% 					= \frac{x}{1!} - \frac{x^3}{3!} + \frac{x^5}{5!} \mp \dots $ \\
% 					$ \cos(x) = \frac{1}{2}(e^{ix}+e^{-ix}) = \smashoperator{\sum \limits_{n=0}^{\infty}} (-1)^n \frac{x^{2n}}{(2n)!}
% 					= \frac{x^0}{0!} - \frac{x^2}{2} + \frac{x^4}{4!} \mp \dots $ \\
% 					$ \tan(x) = \frac{\sin x}{\cos x} = \frac{1}{i} \cdot \frac{e^{ix}-e^{-ix}}{e^{ix}+e^{-ix}} = x+\frac13 x^3+\frac{2}{15}x^5+\frac{17}{315}x^7+\dotsb $ \\
% 					$ \cot(x) = \frac{1}{\tan x} = \frac{\cos x}{\sin x} $
				
% 				\item \textbf{Periodizität}\\
% 					$ \sin(\varphi) = -\cos(\varphi+\frac{\pi}{2}) $\hspace{1cm} $ \cos(\varphi) = \sin( \varphi + \frac{\pi}{2}) $\\
% 					$ \sin(\varphi) = \sin(\varphi+n \cdot 2\pi) $\hspace{1cm} $ \cos(\varphi) = \cos(\varphi+n \cdot 2\pi) $\\
% 					$ \tan(\varphi)=\tan(\varphi + n\cdot\pi) $
				
% 				\item \textbf{Symmetrien}\\
% 					$ \sin(-\varphi) = -\sin(\varphi) $, $ \cos(-\varphi) = \cos(\varphi) $, $ \tan(-\varphi)= -\tan(\varphi) $
					

% 				\item \textbf{Addtionstheoreme}\\
% 					$ \sin(x \pm y) = \sin(x)\cos(y) \pm \cos(x)\sin(y) $\\
% 					$ \cos(x \pm y) = \cos(x)\cos(y) \mp \sin(x)\sin(y) $\\
% 					$ \tan(x \pm y) = \frac{\tan(x) \pm \tan(y)}{1 \mp \tan(x) \tan(y)} = \frac{\sin(x \pm y)}{\cos(x \pm y)}$\\
% 					$ \cot(x \pm y) = \frac{\cot(x)\cot(y) \mp 1}{\cot(y) \pm \cot(x) } = \frac{\cos( x \pm y)}{\sin(x \pm y)} $\\
% 					$\sin{(a+b)}\sin{(a-b)}=\cos^2{b}-\cos^2{a}$\\
% 					$\cos{(a+b)}\cos{(a-b)}=\cos^2{b}-\sin^2{a}$\\
% 		 			$\sinh{(a+b)}=\sinh{(a)}\cosh{(b)}+\cosh{(a)}\sinh{(b)}$\\
% 					$\cosh{(a+b)}=\cosh{(a)}\cosh{(b)}+\sinh{(a)}\sinh{(b)}$\\
% 				\item \textbf{Phytagoras}\\
% 					$ \cos ^2 x + \sin^2 x = 1 $\\
% 					$\cosh^2{(\phi)}-\sinh^2{(\phi)}=1$
% 				\item\textbf{Sinussatz}\\
% 					$ \frac{a}{\sin(\alpha)} = \frac{b}{\sin(\beta)} = \frac{c}{\sin(\gamma)} = 2r $, $r$: Umkreisradius
% 				\item\textbf{Cosinussatz}\\
% 					$ c^2 = a^2 + b^2 - 2ab \cos(\gamma) $
% 				\item\textbf{Zusammenhänge}\\
% 					$2\sin{(\phi)}\cos{(\phi)}=\sin{(2\phi)}$\\
% 					$\cos^2{\phi}-\sin^2{\phi}=2\cos^2{\phi}-1=\cos{(2\phi)}$\\
% 					$\tan{(\frac{\phi}{2})}=\frac{\sin{\phi}}{\cos{\phi}+1}=\frac{1-\cos{(\phi)}}{\sin{(\phi)}}$\\
% 					$\sin{(3\phi)=3\sin{(\phi)}-4\sin^3{(\phi)}}$\\
% 					$\cos{(3\phi)=4\cos^3{(\phi)}-3\cos{(\phi)}}$\\
% 					$\tan{(2\phi)}=\frac{2\tan{(\phi)}}{1-\tan^2{(\phi)}}$\\
% 					$\tan{(3\phi)}=\frac{3\tan{(\phi)}-\tan^3{(\phi)}}{1-3\tan^2{(\phi)}}$\\
% 					$\tan^2{(\frac{\phi}{2})}=\frac{1-\cos{(\phi)}}{1+\cos{(\phi)}}$\\
% 					\eqbox{\sin^2{(\phi)}=\frac{1-\cos{(2\phi)}}{2}, \cos^2{(\phi)}=\frac{1+\cos{(2\phi)}}{2}}\\
% 					$\sin{(a)}\cos{(b)}=\frac{1}{2}(\sin{(a+b)}+\sin{(a-b)})$\\
% 					$\cos{(a)}\cos{(b)}=\frac{1}{2}(\cos{(a+b)}+\cos{(a-b)})$\\
% 					$\sin{(a)}\sin{(b)}=\frac{1}{2}(\sin{(a-b)}-\sin{(a+b)})$\\
% 					$\sin{a}+\sin{b}=2\sin{(\frac{a+b}{2})}\cos{(\frac{a-b}{2})}$\\
% 					$\sin{a}-\sin{b}=2\cos{(\frac{a+b}{2})}\sin{(\frac{a-b}{2})}$\\
% 					$\cos{a}+\cos{b}=2\cos{(\frac{a+b}{2})}\cos{(\frac{a-b}{2})}$\\
% 					$\cosh^2{a}+\sinh^2{a}=\cosh{(2a)}$\\
% 					$2\cdot \cosh^2{a}=1+\cosh{(2a)}$
% 				\item\textbf{Umkehrfunktionen}\\
% 				$\arsinh{(x)}=\ln{(x+\sqrt{x^2+1})}$\\
% 				$\arcosh{(x)}=\ln{(x+\sqrt{x^2-1})}$\\
% 				$\artanh{(x)}=\frac{1}{2}\ln{\left(\frac{1+x}{1-x}\right)}$\\
% 				$\arcoth{(x)}=\frac{1}{2}\ln{\left(\frac{x+1}{x-1}\right)}$\\
% 				\item\textbf{Verschachtelungen}\\
% 				$\sin(\arccos x)=\cos(\arcsin x)=\sqrt{1-x^2}$\\
% 				$\sin(\arctan x)=\cos(\arccot x)=\frac{x}{\sqrt{1+x^2}}$\\
% 				$\sin(\arccot x)=\cos(\arctan x)=\frac{1}{\sqrt{1+x^2}}$\\
% 				$\tan(\arcsin x)=\cot(\arccos x)=\frac{x}{\sqrt{1-x^2}}$\\
% 				$\tan(\arccos x)=\cot(\arcsin x)=\frac{\sqrt{1-x^2}}{x}$\\
% 				$\tan(\arccot x)=\cot(\arctan x)=\frac{1}{x}$\\
% 				$\arcsin(\cos x)\textit{ selber herleiten}$
% 				\end{itemize}				
				
% 			\subsection{Hyperbolische Funktionen}
% 				\begin{itemize}
% 					\item \textbf{Definitionen}\\						
% 					$ \sinh(x) = \frac{1}{2} \left( e^x - e^{-x} \right) = -i\sin(ix) = \smashoperator{\sum \limits_{n=0}^{\infty}} \frac{x^{2n+1}}{(2n+1)!}
% 					= x + \frac{x^3}{3!} + \frac{x^5}{5!} + \dots $ \hspace{1cm} $ \tanh(x) = \frac{\sinh(x)}{\cosh(x)} = \frac{e^x-e^{-x}}{e^x+e^{-x}} $\\
% 					$ \cosh(x) =  \frac{1}{2} \left( e^x + e^{-x} \right) = \cos(ix)= \smashoperator{\sum \limits_{n=0}^{\infty}} \frac{x^{2n}}{(2n)!}
% 					= 1 + \frac{x^2}{2!} + \frac{x^4}{4!} + \dots $ \\
% 					$ \arsinh(y) = \log(y + \sqrt{y^2+1}) $, $ \arcosh(y) = \log(y + \sqrt{y^2-1})$, $ \artanh(y) = \frac{1}{2} \log\frac{1+y}{1-y} $

					
% 					\item \textbf{Phytagoras}\\
% 						$ \cosh ^2 x - \sinh^2 x = 1 $
					
% 					\item \textbf{Additionstheoreme}\\
% 						$ \sinh(x \pm y) = \cosh(x) \sinh(y) \pm \sinh(x) \cosh(y) $\\
% 						$ \cosh(x \pm y) = \cosh(x) \cosh(y) \pm \sinh(x) \sinh(y) $
% 					\item \textbf{Zusammenhänge}\\
% 						$ \sinh(2x) = 2\sinh(x)\cosh(x) $\\
% 						$ \cosh(2x) = \cosh^2(x)+\sinh^2(x) $\\
% 						$ \cosh(\arsinh(t)) = \sqrt{1+t^2} $\\
% 						$ \sinh(\arcosh(t)) = \sqrt{t^2-1} $ für $t>0$
				
% 					Die Hyperbolischen Funktionen sind \emph{ungerade Funktionen}.
% 				\end{itemize}
				
% 	\subsection{Allgemeine Potenz}
% 				\textbf{Eigenschaften}\\
% 				\begin{tabular}{rclrcl}
% 					$ a^{\frac{m}{n}} $ & $ = $		&	$ \sqrt[n]{a^m} $	&
% 					$ a^x $ 		&	$ = $		&	$ \lim \limits_{m/n \to x} a^{\frac{m}{n}} $\\
% 					$ a^{x+y} $	&	$ = $		&	$ a^x \cdot a^y $	&
% 					$ (ab)^x $		&	$ = $		&	$ a^x \cdot b^x $	\\
% 					$ (a^x)^y $	&	$ = $		&	$ a^{xy} $			&	\\
% 					$ a^x $		&	$ < $		&	$ b^x $			& 	\multicolumn{3}{l}{für $ a < b $ und $ x > 0 $,}\\
% 					$ a^x $		&	$ < $		&	$ a^y $			& 	\multicolumn{3}{l}{für $ a > 1 $ und $ x < y $.}\\
% 				\end{tabular}			
				
				
% 			\subsection{Exponentialfunktion}
% 				$ e^z = \exp z= \smashoperator{\sum \limits_{k=0}^{\infty}} \frac{z^k}{k!} =
% 				1+z+\frac{z^2}{2}+\frac{z^3}{6}+\frac{z^4}{24}+\dots $ für $ z \in \mathbb{C} $\\
% 				Weiter $ e^x = \lim \limits_{n \to \infty} \left( 1 + \frac{x}{n} \right) ^n $ für $ x \in \mathbb{R}, n \in \mathbb{N}$\\
% 				$ e = \exp 1 = \lim \limits_{n \to \infty} \left( 1 + \frac{1}{n} \right) ^n = \lim \limits_{n \to \infty} \frac{n}{\sqrt[n]{n!}} =
% 				\smashoperator{\sum \limits_{k=0}^{\infty}} \frac{1}{k!} = 2.718\dots $\\
% 				\\
% 				Für alle $ z,w \in \mathbb{C} $ und $ n \in \mathbb{Z} $ gelten folgende \textbf{Umformungen}:\\
% 				$ \exp(z+w) = \exp(z) \cdot \exp(w) $ \hspace{20mm} $ \exp(nz) = \exp(z)^n $\\
% 				$ \exp(z) \cdot \exp(-z) = \exp(0) = 1 \implies \exp(z) \neq 0 $
				
% 				Für alle $ z \in \mathbb{C} $ gelten folgende \textbf{Umformungen}:\\
% 				\(\overline{e^z} = e^{\overline{z}}\) \hspace{40mm} \(\vert e^z \vert = e^{\text{Re}(z)} \)\\					
% 				Es gilt für alle \(t,x,y\in\mathbb{R}\)und \( z \in \mathbb{C} \) :\\
% 				\eqbox{e^{it} = \cos(t)+i\sin(t)} \quad und \quad \eqbox{e^{x+iy} = e^x \cos(y) + i e^x \sin(y)}\\
% 				desweiteren: \eqbox{e^{i\pi}+1=0} \quad \(e^{2\pi i} = 1\) \quad und \quad \(e^{z+2\pi i} = e^z\)
				
% 			\subsection{Logarithmus}
% 				\textbf{Umformungen}\\
% 				$ \log : \mathbb{R}^{>0} \to \mathbb{R} $\\
% 				Für $ x,y \in \mathbb{R}^{>0} $ und $ t \in \mathbb{R} $ und $ z \in \mathbb{C} $ gilt:\\
% 				\begin{tabular}{ll}
% 					$ \log e^t = t $										&		$ e^0 = 1 $\\
% 					$ e^{\log x} = x $									&		$ \log 1 = 0 $\\
% 					$ \log xy = \log x + \log y $ (da $ e^{a+b} =e^a e^b $)			&		$ \frac{1}{e^x} = e^{-x} $\\
% 					$ \log x^t = t \cdot \log x $								&		$ \log \frac{1}{x} = -\log x $\\
% 					$ x^t = e^{t \cdot \log x} $								&		$ x^z = e^{z \cdot \log x} $\\
% 				\end{tabular}\\
% 				\hspace*{1.1mm} $ a^x = y \iff x = \log_a y $ \hspace{22mm} $ e^x = y \iff x = \log y $
				
% 				\hspace*{1.1mm} $ e^z =  \lim \limits_{n \to \infty} \left( 1 - \frac{z}{n} \right) ^n $ für alle $ z \in \mathbb{C} $
				
% 				Für alle $ x \in \mathbb{R} $ mit $ \vert x \vert < 1 $ gilt\\
% 				$ \log(1+x) = \smashoperator{\sum \limits_{k=1}^{\infty}} (-1)^{k+1} \frac{x^k}{k} =
% 				x - \frac{x^2}{2} + \frac{x^3}{3} - \frac{x^4}{4} \pm \dots $
				
									
				
% 				\subsection{Komplexe Zahlen}
% \textbf{(Absoluter) Betrag} (Abstand des Punktes \((x,y)\) vom Ursprung):\\
% 				\(\vert z \vert = \sqrt{z \cdot \overline{z}} = \sqrt{x^2 + y^2}\) und somit auch \(\vert z \vert^2 = z\cdot \overline{z} = x^2+y^2\)\\
% 				\textbf{Komplex Konjugierte}\\
% 				$ \overline{z} = x-iy $\\
% 				\textbf{Polarkoordinaten} \hspace{0.5cm} \(z = r\cdot e^{i\varphi}\)\\
% 				sonst: \hspace*{1cm} (Bsp.: \(-8 \to r=8, \varphi = 180^\circ = \pi \) )\\
% 				\(r=\sqrt{x^2+y^2}\) und \( \varphi = \arctan \left(\frac{y}{x}\right) \)\\
% 				Polar \(\to\) Kartesisch: \eqbox{( x = r \cos \varphi \), \( y = r \sin \varphi )}\\
% 				\textbf{Eigenschaften}
% 				\begin{multicols*}{2}
% 					\begin{itemize}[noitemsep,topsep=0pt]
% 						\item $ x = \text{Re } z = \frac{z + \overline{z}}{2} $
% 						\item $ y = \text{Im } z = \frac{z - \overline{z}}{2i}$
% 						\item $ z \in \mathbb{R} \Longleftrightarrow z = \overline{z} $
% 						\item $ \overline{\overline{z}} = z $
% 						\item $ \overline{ \left( \frac{1}{z} \right)} = \frac{1}{(\overline{z})}$
% 						\item $ \overline{z_1 + z_2} = \overline{z_1} + \overline{z_2}$
% 						\item $ \overline{z_1 \cdot z_2} = \overline{z_1} \cdot \overline{z_2} $
% 						\item $ \frac{1}{z}= \frac{\overline{z}}{z \cdot \overline{z}} = \frac{\overline{z}}{\vert z \vert^2} $
% 					\end{itemize}
% 				\end{multicols*}\vspace{-\baselineskip}
% \eqbox{e^{it} = \cos(t)+i\sin(t)} \quad und \quad \eqbox{e^{x+iy} = e^x \cos(y) + i e^x \sin(y)}\\
% \\
% 				\textbf{Dreiecksungleichung}\\
% 				$ \vert z+z' \vert \leqslant \vert z \vert + \vert z' \vert $\\

% 			\subsection{\(n\)-te Wurzel einer Zahl berechnen}
% 				\emph{Vorgehen}
% 				\begin{enumerate}
% 					\item	Zahl in Polarform (\(r\cdot e^{i\varphi}\)) umwandeln (entweder direkt ablesen oder mit Umrechnungsformel)
% 					\item	\eqbox{\sqrt[n]{z} = \sqrt[n]{r} \cdot e^{i\cdot\frac{\varphi + 2\pi k}{n}}} berechnen mit \(k=0,\dots,n-1\)
% 					\item	Lösungen wenn verlangt wieder in Normalform bringen
% 				\end{enumerate}\vfill\null\columnbreak
% 				\includegraphics[width=0.66\textwidth]{images/DGL.pdf}
% 				\end{multicols*}		
% 				\begin{multicols*}{3}
% 	\section{Gew"ohnliche Differentialgleichungen (DGL)}
% 			\subsection{Klassifikation}
% 				\begin{itemize}
% 					\item \textbf{Ordnung}\\
% 					In einer DGL kommen Ausdrücke mit $x$, $y(x)$ und die Ableitungen $y'(x)$, $y''(x)$, $y^{(3)}(x)$, ... vor. Die Ordnung einer DGL ist die höchste vorkommende Ableitung.
% 					\item \textbf{Linear / nicht linear}\\
% 					Eine Gleichung ist linear, falls nur lineare Terme von y und deren Ableitungen vorkommen (also keine $\sin(y), e^{y'}, y^2, (y'')^3, ...$). Sonst heisst die Gleichung nicht linear.
% 					\item \textbf{Homogen / inhomogen}\\
% 					Eine Gleichung heisst homogen, falls es keine Terme gibt, welche nicht von y abhängen (also keine isolierten Konstanten und keine Terme die \underline{nur} von x abhängen). Sonst heisst die Gleichung inhomogen.
% 				\end{itemize}
% 			\subsection{Anfangswertproblem}
% 				Für die Eindeutigkeit einer Lösung brauchen wir Anfangsbedingungen, d.h. Werte $y(x_0), y'(x_0), ..., y^{(n-1)}(x_0)$. Für eine DGL n-ter Ordnung also n Bedingungen und einen fixen Punkt $x_0$ als ''Anfangszeit''.
% 			\subsection{Methoden für DGL 1. Ordnung}
% 				\subsubsection{Separierbare DGL (homogene Gleichungen)}
% 					\emph{Ausgangslage}:\\
% 					\dots ist eine DGL 1. Ordnung der Form:\\
% 					\eqbox{y'(x)=\frac{dy}{dx}=h(x)\cdot g(y)} mit $g(y)\neq0$
				
% 					\emph{L"osungsrezept}:
% 					\begin{enumerate}
% 						\item	Umformung\\
% 						$\dfrac{dy}{g(y)} = h(x) \cdot dx$
% 						\item	beide Seiten integrieren\\
% 						$\underbrace{\int \dfrac{1}{g(y)} dy}_{G(y) \text{ gesucht}} = \underbrace{\int h(x) dx}_{H(x) \text{ gesucht}} +\ C$\\
% 						($G(y)$ und $H(x)$ sind die Stammfunktionen von $\frac{1}{g(y)}$ und $h(x)$.)
% 						\item	Umformen nach $y(x) = \dots$\\
% 					\end{enumerate}
				
% 				\subsubsection{{Variation der Konstante} (inhomogene Gleichungen)}
% 					\emph{Ausgangslage}:\\
% 					\eqbox{y'(x)=\dfrac{dy}{dx} = h(x) \cdot g(y) + b(x)}
				
% 					\emph{L"osungsrezept}:
% 					\begin{enumerate}
% 						\item	Homogene Lösung\\
% 						\textit{Wir lösen die homogene Gleichung $y'(x)=h(x)\cdot g(y)$ und erhalten die homogene Lösung:}\\
% 						$y_h(x)=[Etwas\ mit\ Integrationskonstante\ C]$
% 						\item	Partikuläre Lösung\\
% 						\textit{Wir nehmen $y_h(x)$ und wandeln C in $C(x)$ um (Variation der Konstante). Die herausgekommene Lösung $y_p(x)=[Etwas\ mit\ C(x)]$ setzen wir nun in die Anfangs-DGL ein und lösen nach $C(x)$ auf.\\ \normalfont{\attention} Beim finden von $C(x)$ braucht man \underline{keine 2te Integ.Konstante} und es \underline{muss sich etwas wegkürzen}! (sonst falsch)}
% 						\item	Superpositionsprinzip: \eqbox{y(x)=y_h(x)+y_p(x)}
% 						\item Anfangswertproblem lösen
% 					\end{enumerate}
			
% 				\subsubsection{Substitution}
% 					Brauchen wir zum lösen von:
% 					$$y'=h \Big( \frac{y}{x}\Big),\hspace{10pt}y'=h(ax+by+c),$$ $$y'=h\Big(\frac{ax+by+c}{dx+ey+f}\Big),\hspace{10pt}y'=\frac{y}{x}\cdot h(xy)$$
% 					\begin{itemize}
% 						\item \textbf{Gleichungen der Form $y'=h(\frac{y}{x})$}\\
% 						Substitution: $z(x)=\frac{y(x)}{x}\Leftrightarrow y(x)=x\cdot z(x)$\\
% 						$\Rightarrow y'=z+x\cdot z'$
% 						\item \textbf{Gleichungen der Form $y'=h(ax+by+c)$}\\
% 						Substitution: $z(x)=ax+by(x)+c\Leftrightarrow y(x)=\frac{z-ax-c}{b}$\\
% 						$\Rightarrow y'=\frac{z'-a}{b}$
% 						\item \textbf{Gleichungen der Form $y'=h\big(\frac{ax+by+c}{dx+ey+f}\big)$}\\
% 						\begin{enumerate}
% 							\item Lösen des Gleichungssystem:\\
% 							$\begin{cases}ax+by+c=0\\ dx+ey+f=0\end{cases}$
% 							\item Determinante \underline{nicht} gleich 0:\\
% 							$det\left( \begin{matrix}
% 							a	&	b\\
% 							d	&	e	\end{matrix} \right)\neq0$
% 							\item neue Variable: $z=y-y_0 und\ t=x-x_0$
% 							\item Substitution: $y'\rightarrow z'$\\
% 							$\Rightarrow y'=\frac{dy}{dx}=\frac{d(z+y_0)}{d(t+x_0)}=\frac{dz}{dt}=z'$
% 						\end{enumerate}
% 						\item \textbf{Gleichungen der Form $y'=\frac{y}{x}\cdot h(xy)$}\\
% 						Substitution: $z(x)=x\cdot y(x)\Leftrightarrow y(x)=\frac{z(x)}{x}$\\
% 						$\Rightarrow y'=\frac{xz'-z}{x^2}$
% 					\end{itemize}
% 			\subsection{Methoden für DGL-Systeme}
% 				\subsubsection{Lineares, homogenes, n x n DGL-System}
% 					\emph{Ausgangslage}:\\
% 					\eqbox{\dot\vec y(t)=A\cdot\vec y(t)}\\
% 					\emph{L"osungsrezept}:
% 					\begin{enumerate}
% 						\item $\Rightarrow\vec y(t)=e^{(At)}\cdot\vec C\ \leftarrow gibt\ Vektor=e^{\lambda_1}\cdot C_1\cdot \vec{ev_1} + e^{\lambda_2}\cdot C_2\cdot \vec{ev_2} + \dotsm$
% 						\item Mit AWP: $\Rightarrow\vec y(t)=e^{(At)}\cdot\vec{y_0}$
% 					\end{enumerate}
% 			\subsection{Methoden für DGL n-ter Ordnung}
% 				\subsubsection{homogen, linear, konstante Koeffizienten}
% 					\emph{Ausgangslage}:\\
% 					\eqbox{y^{(n)}(x)+a_{n-1}y^{(n-1)}(x)+...+a_1y'-(x)+a_0y(x)=0}
			
% 					\emph{L"osungsrezept}:
% 					\begin{enumerate}
% 						\item	n viele Anfangswertbedingungen \& $a_0,\dots,a_{n-1}\in\mathbb{R}$
% 						\item Ansatz: \eqbox{y(x)=e^{\lambda x}\hspace{10pt}(Euler'scher\ Ansatz)}
% 						\item Charakteristische Gleichung lösen und $\lambda$'s finden:\\
% 						\eqbox{\lambda^n+a_{n-1}\lambda^{n-1}+\dots+a_1\lambda+a_0=0}\\[4pt]
% 						$\Rightarrow(\lambda-\lambda_1)^{m_1}\cdot(\lambda-\lambda_2)^{m_2}\cdot\dots\cdot(\lambda-\lambda_r)^{m_r}$
% 						\item \textbf{einfache Nullstellen:} (Faktor $(\lambda-\lambda_i)$ in char.Gl.)\\[2pt]
% 						geben die Lösung $c_i\cdot e^{\lambda_ix}$
% 						\item \textbf{k-fache Nullstellen:} (Faktor $(\lambda-\lambda_i)^k$ in char.Gl.)\\[2pt]
% 						geben die Lösung $\underbrace{c_{k1}\cdot e^{\lambda_ix}+c_{k2}\cdot x\cdot e^{\lambda_ix}+\dots+c_{kn}\cdot x^{k-1}\cdot e^{\lambda_ix}}_{k-Summanden}$
% 						\item \textbf{allgemeine Lösung:} Summe der einzelnen Lösungen\\
% 						\textit{Um $c_i$ zu finden, AWP lösen (LGS evtl. mit Lin Alg)}
% 					\end{enumerate}
% 					\attention\textbf{Reelle oder komplexe Lösungen}
% 					\begin{itemize}
% 						\item komplexe Lösung gesucht:\\
% 						\textit{es genügt, die beiden komplexen NST auszurechnen:\\
% 						$c_1\cdot e^{\lambda_1t}+c_2\cdot e^{\lambda_2t},\ \lambda_1=\overline\lambda_2$}
% 						\item reelle Lösung gesucht:\\
% 						\textit{Sei $\lambda$ eine der beiden NST, $\lambda=u+iv$ so ist $\overline\lambda$ die andere.}\\
% 						\textit{$\Rightarrow$ Lösung: }\eqbox{e^{ux}\cdot(A\cdot\cos{(vx)}+B\cdot\sin{(vx)})} 
% 					\end{itemize}
			
% 				\subsubsection{inhomogen, linear, konstante Koeffizienten}

% 					\emph{Ausgangslage}:\\
% 					\eqbox{y^{(n)}(x)+a_{n-1}y^{(n-1)}(x)+...+a_1y'-(x)+a_0y(x)=\underbrace{b(x)}_{Stoerung}}
			
% 					\emph{L"osungsrezept}:
% 					\begin{enumerate}
% 						\item Homogene Lösung $\rightarrow y_h(x)$
% 						\item Partikuläre Lösung\\
% 						\textit{Man muss den richtigen Ansatz für $y_p(x)$ finden (Tabelle)}
% 						\includegraphics[width=\columnwidth]{images/ansatz.png}
% 						\begin{itemize}
% 							\item \textit{Setze $y_p(x)$ in die Gleichung ein, und mache ein Koeffizientenvgl. um die Parameter A, B, $A_n$, $B_n$, $\dots$ in $y_p(x)$ zu finden. $\rightarrow y_p(x)$}
% 						\end{itemize}
% 						\attention \textit{Falls man zwei Störterme hat, also $b(x)=b_1(x)+ b_2(x)$, macht man zwei Ansätze und man bekommt \underline{zwei} part. Lösungen $y_{p_1}(x), y_{p_2}(x)$}\\
% 						\attention \textbf{\textit{Falls der Ansatz $y_p$ ein Term hat, welcher bereits in $y_h(x)$ vorkommt, muss der Ansatz mit x multipliziert werden.}}
% 						\item Superpositionsprinzip: $y(x)=y_h(x)+y_p(x)$
% 						\item Anfangswertproblem lösen
% 					\end{enumerate}
% 															% Semester 2: Ab Februar 2017
% 		\section{Einschub: Lineare Algebra}
% 			\subsection{Matrixmultiplikation}
% 				\includegraphics[width=0.5\columnwidth]{images/matrixmult1.png}
% 				\includegraphics[width=0.5\columnwidth]{images/matrixmult2.png}
% 			\subsection{Determinanten und Eigenwerte}
% 				\textbf{Determinante einer \(2 \times 2\) - Matrix}\\
% 				\(A = \MATR{a & b \\ c & d} \) \hspace{1cm} \(\text{det } A = ad-bc\)
				
% 				\textbf{Determinante einer \(3 \times 3\) - Matrix}\\
% 				\textbf{(Methode: Entwicklung nach einer Zeile oder Spalte)}\\
% 				\emph{Bsp.} 2. Zeile \hspace{2.5cm} Vorzeichen \textbf{so} w"ahlen:\\
% 				\(B = \MATR{a_{11} & a_{12} & a_{13} \\ a_{21} & a_{22} & a_{23} \\ a_{31} & a_{32} & a_{33}} \)
% 				\hspace{1.5cm} \(\MATR{+ & - & + \\ - & + & - \\ + & - & +}\)\\
% 				\(\text{det } B =\) \\
% 				\(-a_{21}\cdot \text{det}\MATR{a_{12} & a_{13} \\ a_{32} & a_{33}}
% 				+ a_{22}\cdot \text{det} \MATR{a_{11} & a_{13} \\ a_{31} & a_{33}}
% 				- a_{23} \cdot \text{det}\MATR{a_{11} & a_{12} \\ a_{31} & a_{32}}\) \\[4pt]
% 				\textbf{Regel von Sarrus}\\
% 				Determinante einer $3\times 3$-Matrix:\\
% 				\includegraphics[scale=0.35]{images/sarrus.png}\\
% 				$\det{\begin{pmatrix} a & b & c \\ d & e & f \\ g & h & i \end{pmatrix}}=aei+bfg+cdh-gec-hfa-idb$\\[4pt]
% 				\textbf{Eigenwerte}\\
% 				Um die Eigenwerte einer Matrix \(A\) zu finden folgende Gleichung nach allen m"oglichen \(\lambda\) aufl"osen:\\
% 				\( \text{det}(A - \lambda I) = 0 \) \hspace{1cm} (wobei \(I\) die Einheitsmatrix ist)\\
% 				\textbf{ACHTUNG} Es gilt: \(\text{det}(A-\lambda I) = 0 \Longleftrightarrow \text{det}(\lambda I - A) = 0 \) \\[4pt]
% 				\textbf{Diagonalisieren}\\
% 				\begin{enumerate}
% 					\item Eigenwerte berechnen
% 					\item$D=\MATR{\lambda_1&0&0\\0&\lambda_2&0\\0&0&\lambda_3},\ U=\MATR{\vec {ev_1} & \vec {ev_2} & {ev_3}}$\\
% 					Eigenvektoren: Kern von $A-D$
% 				\end{enumerate}
% 				\textbf{Invertierbarkeit}\\
% 				$\det{A}=0\Leftrightarrow$ Zeilen von $A$ sind linear Abhängig\\
% 				$\det{A}\neq 0\Leftrightarrow$ $A$ ist invertierbar\\[4pt]
% 				\textbf{Rechenregeln}
% 				\begin{itemize}
% 					\item Der Wert der Determinante ändert sich nicht, wenn man das Vielfache einer Spalte/Zeile zu einer anderen addiert/subtrahiert
% 					\item $\det{(A)}=\det{(A^T)}$
% 					\item Wenn zwei Zeilen/Spalten vertauscht werden, ändert sich das Vorzeichen der Determinanten
% 				\end{itemize}
% 				\textbf{Matrixnorm}:\ \ 
% 				\eqbox{||AB||\le||A||\cdot||B||}
% 			\subsection{Exponentialmatrix}
% 				\textbf{\textit{Definition:}}
% 				$$e^A=\exp(A)\coloneqq\sumni\frac{A^n}{n!},\ A^0=I_n=1\!\!1$$
% 				\textbf{\textit{Eigenschaften:}}
% 				\begin{enumerate}[label=(\roman*)]
% 					\item $\exp(0)=I_n=1\!\!1\ (Einheitsmatrix)$
% 					\item $\forall A\in\mathbb{C}^{n\times n},\ \forall\alpha,\beta\in\mathbb{C}:\ \exp((\alpha+\beta)A)=\exp(\alpha A)\cdot\exp(\beta A)$
% 					\item \hspace*{-5pt}$\forall A,B\in\mathbb{C}^{n\times n}\ mit\ AB=BA\ gilt:\exp(A+B)=\exp(A)\cdot\exp(B)$
% 					\item $(\exp(A))^{-1}=\exp(-A)$
% 				\end{enumerate}
% 				\textbf{\textit{Berechnung:}}
% 				\begin{itemize}
% 					\item Typ 1: D ist eine Diagonalmatrix	\(\MATR{x & 0 & 0 \\ 0 & x & 0 \\ 0 & 0 & x}\)\\
% 					$\Rightarrow\exp(D)=\MATR{e^x & 0 & 0 \\ 0 & e^x & 0 \\ 0 & 0 & e^x}$
% 					\item Typ 2: Diagonalisierbare Matrizen\\
% 					\textit{Sei A diag'bar mit $A=U\cdot D\cdot U^{-1}\Rightarrow\forall k\ge0:A^k=U\cdot D^k\cdot U^{-1}$}\\
% 					$\Rightarrow\exp(A)=\sumni\frac{UD^nU^{-1}}{n!}=U\cdot\Big(\sumni\frac{D^n}{n!}\Big)\cdot U^{-1}\\=U\cdot\exp(D)\cdot U^{-1}$
% 					\item Typ 3: obere Dreiecksmatrix\\
% 					$A=\MATR{x & \bigstar & \bigstar \\ 0 & x & \bigstar \\ 0 & 0 & x}=D+N,\\
% 					D\ eine\ Diagonalmatrix,\ N\ eine\ strikte\ obere\ Dreiecksmatrix\\
% 					\rightarrow Ueberpruefe\ DN=ND\Rightarrow\exp(A)=\exp(D)\cdot\exp(N)\\
% 					\rightarrow N\ ist\ nilpotent:\\ d.h.\ \exists k_0\in\mathbb{N}\ s.d.\ \forall k\ge k_0:\ N^k=0\Rightarrow N^0, N^1,\dots,\underbrace{N^5,\dots,N^k}_{=0}$
% 					\item Typ 4: Blockmatrizen (Blockdiagonalmatrizen)\\
% 					$A=\MATR{A_1 & 0 & 0\\ 0 & A_2 & 0\\ 0 & 0 & A_3}\Rightarrow\exp(A)=\MATR{e^{A_1} & 0 & 0\\ 0 & e^{A_2} & 0\\ 0 & 0 & e^{A_3}}$
% 					\item Typ 5: Bei einigen Matrizen kann sich ein Muster erkennen lassen, so dass z.B. $A^0 = A^3= A^6, A^1=A^4=A^7, A^2=A^5=A^8,\dots$
% 				\end{itemize}
% 			\subsection{Kern einer Matrix \textnormal{\textit{existiert wenn $\det=0$}}}
% 				z.B. $df(x_0)=\MATR{1 & 2 & 3 \\ 4 & 5 & 6 \\ 7 & 8 & 9}$, 
% 				$ker(df)\coloneqq\MATR{1 & 2 & 3 \\ 4 & 5 & 6 \\ 7 & 8 & 9}\cdot\MATR{v_1 \\ v_2 \\ v_3}=\MATR{0 \\ 0 \\ 0}$\\
% 				\begin{minipage}{0.27\columnwidth}$ker(df)=\MATR{1\\-2\\1}$, \end{minipage}
% 				\begin{minipage}{0.73\columnwidth}bei nur einer Zeile und drei Unbe: 3 Nullzeilen anfü.\\
% 				Jede Nullzeile wird eine ``Variable'',\\triviale Lösung gilt nicht\end{minipage}
% 			\subsection{Hurwitz-Kriterium}
% 				Seien $A_k\coloneqq k\times k-Matrizen\ oben\ links,\ (1\le k\le n)$
% 				\begin{itemize}
% 					\item A PD $\Leftrightarrow\forall 1\le k\le n:\ \text{det}\ A_k>0$
% 					\item A ND $\Leftrightarrow\forall 1\le k\le n:\ \begin{cases}k\ ungerade:\ \text{det}\ A_k<0\\k\ gerade:\ \text{det}\ A_k>0\end{cases}$
% 				\end{itemize}
% 			\subsection{Definitheit}
% 				\begin{tabular}{|c|c|}
% 					\hline
% 					positiv Def. & Alle EW $>$ 0 \\
% 					\hline
% 					positiv semi Def. & Alle EW $\ge$ 0 \\
% 					\hline
% 					negativ Def. & Alle EW $<$ 0 \\
% 					\hline
% 					negativ semi Def. & Alle EW $\le$ 0 \\
% 					\hline
% 					indefinit & Ein EW so, ein anderer so\\
% 					\hline
% 				\end{tabular}
% 			\subsection{Zeichnen von DGL / Phasenportrait}
% 				\begin{enumerate}
% 					\item Betrachte homogenes lineares DGL System's: $\MATR{\dot x \\ \dot y}=A\cdot\MATR{x \\ y}$
% 					\item Hat $A$ konjugiert komplexe Eigenwerte $\lambda, \bar\lambda$, so umrundet jede Trajektorie den einzigen stationären Punkt $(0,0)$ so:
% 					\begin{minipage}{0.5\columnwidth}
% 						\hspace*{-1cm}\includegraphics[width=1.33\columnwidth]{./images/strudelpunkte.png}
% 					\end{minipage}
% 					\begin{minipage}{0.45\columnwidth}
% 						\begin{itemize}
% 							\item$Re(\lambda)>0\Rightarrow$Polarradius wächst, $(0,0)$ ist abstossender Brenn-/Strudelpunkt
% 							\item$Re(\lambda)<0\Rightarrow$Polarradius fällt, $(0,0)$ ist anziehender Brenn-/Strudelpunkt
% 							\item$Re(\lambda)=0\Rightarrow$Polarradius konstant, $(0,0)$ ist Zentrum/Wirbelpunkt
% 						\end{itemize}
% 					\end{minipage}
% 					\item EW von $A$ reell und $\lambda_1\neq\lambda_2$: 
% 					\begin{itemize}
% 						\item grösserer EW: EV gibt Richtung für $t\to+\infty$
% 						\item kleinerer EW: EV gibt Richtung für $t\to-\infty$
% 					\end{itemize}
% 					\begin{enumerate}
% 						\item $\lambda_1, \lambda_2>0\Rightarrow(0,0)$ ein abstossender Knoten\\
% 						$t\to(0,0)$ kleinerer EW domin., $t\to+\infty$ grösserer EW domin.
% 						\item $\lambda_1, \lambda_2<0\Rightarrow(0,0)$ ein anziehender Knoten\\
% 						$t\to(0,0)$ grösserer EW domin., $t\to-\infty$ kleinerer EW domin.
% 						\item $\lambda_1<0, \lambda_2>0\ (o.B.d.A.)\Rightarrow(0,0)$ ein Sattelpunkt\\
% 						$t\to-\infty$ kleinerer EW domin., $t\to+\infty$ grösserer EW domin.
% 					\end{enumerate}
% 					\item EW von A reell und $\lambda_1=\lambda_2$:
% 					\begin{enumerate}
% 						\item Sternknoten, falls Eigenraum zweidimensional. Anziehend / Abstossend je nachdem ob $\lambda>0$ oder $\lambda<0$.
% 						\item Knoten, abstossend oder anziehend mit einer charakteristischen Richtung
% 					\end{enumerate}
% 					\item EW von A reell und $\lambda_1=0, \lambda_2\neq0$:
% 					Alle Punkte auf den EV von $\lambda_1$ singuläre Punkte, dann in Richtung EV von $\lambda_2$ abstossend / anziehend dazu.
% 					\item EW von A reell und $\lambda_1=\lambda_2=0$:
% 					Alle Punkte auf den EV von $\lambda_1$ singuläre Punkte, dann parallele Geraden dazu.
% 				\end{enumerate}
				
				
%   		\end{multicols*}
	
\setcounter{secnumdepth}{2}
\end{document}

\begin{comment}        
    	\subsection{Lineare DGL mit konstanten Koeffizienten}
       	\subsubsection{Homogen}
        	    \resizebox{\linewidth}{!}{
            	    \eqbox{a_ny^{(n)} + a_{n-1}y^{(n-1)} + \ldots + a_0y = 0 \qquad (a_0, \ldots, a_n \in \mathbb{R})}
        	    }
            	Verwende \textbf{Euler-Ansatz}: \\
            	\eqbox{y(x) = e^{\lambda x}
            		\qquad (\lambda \in
            	\mathbb{C})} \\
            	Löse \textbf{charakteristische Gleichung}: \\
            	\eqbox{a_n\lambda^n+a_{n-1}\lambda^{n-1}+ \ldots + \lambda_0 = 0} \\
            	\emph{Reelle einfache Eigenwerte}: Alle Eigenwerte sind reell und $\lambda_i \neq \lambda_j$ wenn $i \neq j$. Dann sind
            	\eqbox{y_i(t) =e^{\lambda_i t} \text{ und } y_j(t) =e^{\lambda_j t}}
            	Lösungen.\\
            	\emph{Komplexe Eigenwerte}: Treten immer nur komplex konjugiert auf. Für jeden komplex konjugierten Eigenwert
            	$\lambda = a \pm ib$ gibt es zwei linear unabhängige Lösungen mit reellen Koeffizienten:
            	\eqbox{y_1(t) = e^{at} \cdot \cos(bt) ~ , ~ y_2(t) = e^{at} \cdot \sin(bt)}
            	\emph{Mehrfache Eigenwerte}: Jede $m$-fache Nullstelle $\lambda$ hat $m$ linear unabhängige Lösungen:
            	\eqbox{e^{\lambda x}, xe^{\lambda x}, \ldots, x^{m-1}e^{\lambda x}}\\
            	\textbf{Superpositionsprinzip}: Sind die Funktionen $y_k$ linear unabhängige Lösungen, dann ist auch
            	$y(t) = c_1 y_1(t) + \ldots + c_ky_k(t)$
            	eine Lösung.\\
            	\attention Konstanten $c_i \in \mathbb{R}$ nicht vergessen!
            
    	\subsubsection{Inhomogen}
        	\emph{Prinzip}: Seien $Y_1(t)$, $Y_2(t)$, \ldots , $Y_n(t)$ Lösungen der DGL $Ly = 0$,
        	und sei $y_p(t)$ eine irgendwie gefundene Lösung der inhomogenen DGL $Ly = K(t)$. Dann ist 
        	\eqbox{y(t) = c_1 Y_1(t) + c_2 Y_2(t) + \ldots + c_k Y_k(t) + y_p(t)}
        	die allgemeine Lösung von $Ly = K(t)$.\\
        	\textbf{Superpositionsprinzip}: Sei $K(t) = K_1(t) + K_2(t)$. Falls $y_i$ eine Lösung der DGL $L_y = K_i(t)$ für $i = 1,2$ ist,
        	ist $y_1 + y_2$ eine Lösung der DGL $Ly = K$.\\
        	\emph{Ansatz mit unbestimmten Koeffizienten}: Die Anregung ist selber Lösung einer homogenen DGL mit konstanten Koeffizienten
        	oder eine Linearkombination von solchen.\\
        	\begin{quote}
        		\emph{"$y_p(x)$ hat dieselbe Form wie der inhomogene Term $b(x)$."}
        	\end{quote}
        	\begin{center}
        		\resizebox{\linewidth}{!}{
        			\begin{tabular}{| l | l | l |}
        				\hline
        				$b(x)$ & Spektralbedingung & Ansatz \\ \hline
        				\multirow{2}{*}{$e^{\lambda_0x}$} & $\lambda_0 \notin$ Spec L & $y_p(x) = ae^{\lambda_0x}$ \\ \cline{2-3}
        										       & $\lambda_0 \in$ Spec L ($m$-fach) & $y_p(x) = ax^me^{\lambda_0}$ \\ \hline
        				\multirow{2}{*}{$\cos \omega x$ oder $\sin \omega x$} & $\pm i\omega \notin$ Spec L &
        				$y_p(x) = a \cos \omega x + b \sin \omega x$ \\ \cline{2-3}
        				& $\pm i\omega \in$ Spec L ($m$-fach) & $y_p(x) = x^m (a \cos \omega x + b \sin \omega x)$ \\ \hline
        				\multirow{2}{*}{$x^r$} & $0 \notin$ Spec L & $y_p(x) = a_0 + a_1x + \ldots + a_rx^r$ \\ \cline{2-3}
        									    & $0 \in$ Spec L ($m$-fach) & $y_p(x) = x^m (a_0 + a_1x + \ldots + a_rx^r)$ \\ \hline
        				\multirow{2}{*}{$x^r e^{\lambda_0 x}$} & $\lambda_0 \notin$ Spec L &
        				$y_p(x)=(a_0 + a_1x + \ldots + a_rx^r) e^{\lambda_0 x}$ \\ \cline{2-3}
        				& $\lambda_0 \in$ Spec L & $y_p(x) = x^m (a_0 + a_1x + \ldots + a_rx^r) e^{\lambda_0 x}$ \\ \hline
        
        			\end{tabular}
        		}
        	\end{center}
        	\emph{Vorgehen}:
        	\begin{enumerate}
            	\item	Inhomogenen Teil aufteilen
            	\item	Für jeden inhomogenen Teil einen Ansatz aus der Tabelle wählen
            	\item	Den Ansatz in die (auf der rechten Seite reduzierte) DGL einsetzen, Koeffizientenvergleich machen
            	\item	So fortfahren für den Rest der inhomogenen Terme
            	\item	Zusammensetzen: $y(x) = c_1 \cdot y_{h_1}(x) + \ldots + c_m \cdot y_{h_m}(x) + y_{p_1}(x) + \ldots + y_{p_n}(x)$
            		(Für $m$ homogene, $n$ partikuläre Lsg.)
            	\item	Anfangsbedingungen einsetzen
        	\end{enumerate}
        
    	\subsection{Eulersche DGL}
        	Eine \textbf{Eulersche Differentialgleichung} ist eine \emph{lineare} homogene DGL der Form
        	\resizebox{\linewidth}{!}{
    	        \eqbox{c_n y^{(n)} + \frac{c_{n-1}}{x} y^{(n-1)} + \frac{c_{n-2}}{x^2} y^{(n-2)}
        	    + \ldots + \frac{c_1}{x^{n-1}} y' + \frac{c_0}{x^n} y = 0}
        	}
        	oder in alternativer Schreibweise
        	\resizebox{\linewidth}{!}{
        	    \eqbox{c_n x^n y^{(n)} + c_{n-1}x^{n-1} y^{(n-1)} + \ldots + c_1 x y' + c_0 y = 0}
        	}
        	\emph{Homogen:}
        	\begin{itemize}
            	\item		Bringe DGL auf obige Form ("alternative Schreibweise")
            	\item		Substituiere $y(x) = x^\alpha$, $x^\alpha$ kann direkt ausgeklammert werden
            	\item		Löse $inp(\alpha) := c_n \alpha^{(n)} + c_{n-1}\alpha^{(n-1)} + \ldots + c_1\alpha + c_0 = 0$
            	\item		Für $n$ verschiedene reelle Lösungen von $inp(\alpha)$ existieren $n$ linear unabhängige Lösungen:
            		\eqbox{Y_k(x) := x^{\alpha_k} \qquad (1 \leqslant k \leqslant n)}
            	\item		Die allgemeine Lösung ist dann
            		\eqbox{y(x) := c_1x^{\alpha_1} + c_2x^{\alpha_2} + \ldots + c_nx^{\alpha_n}}
            	\item		Eine \emph{doppelte Nullstelle} $\alpha_0$ hat die Lösungen
            		\eqbox{Y_1(x) := x^{\alpha_0}, \quad Y_2(x) := x^{\alpha_0} \log x}
        	\end{itemize}
        
        	\emph{Inhomogen:}
        	\begin{itemize}
            	\item		Substituiere $x = e^t$, $xy'(t) = \dot{y}$, $x^2y''(t) = \ddot{y} - \dot{y}$
            	\item		Löse inhomogene DGL mit konstanten Koeff. nach $y(t)$.
            	\item		Rücktransformation: $t = \log (x)$
        	\end{itemize}		

\end{comment}
