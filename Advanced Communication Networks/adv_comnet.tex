

% Advanced Topics in Communication Networks D-ITET
% ===========================================================================
% @Author: Noah Huetter
% @Date:   2019-09-24 17:26:28
% @Last Modified by:   noah
% @Last Modified time: 2019-10-08 10:59:46
% ---------------------------------------------------------------------------

\documentclass[a4paper, fontsize=8pt, DIV=1]{scrartcl}
\usepackage{lastpage}
\usepackage{hyperref}
% Include general settings and customized commands
%
% General packages and settings
% ===========================================================================
% Author:			Silvano Cortesi (cortesis@student.ethz.ch)
% Version:			1.2
% Last changed:		03.01.2018
%
% ---------------------------------------------------------------------------




\usepackage[german,british]{babel} %choose your language \usepackage[german]{babel}
%\usepackage[T1]{fontenc}
\usepackage[utf8]{inputenc}
\usepackage{fancyhdr}
%\usepackage{lastpage}
%\usepackage{lmodern}
%\usepackage{enumerate}
%\usepackage{float} % for positioning of figures
\usepackage[landscape, margin=1cm]{geometry}
\usepackage[dvipsnames]{xcolor}
\usepackage{pdfpages}


%% Math %%
\usepackage{amscd}
\usepackage{blindtext}
\usepackage{enumitem}
\usepackage{multicol}
\usepackage{parskip}
\usepackage{empheq}
\usepackage{amsmath}
\usepackage{amsfonts}
\usepackage{amssymb}
\usepackage{amsthm}
%\usepackage{dsfont}
%\usepackage{esint} % provides \oiint
\usepackage{mathrsfs}
%\usepackage{trfsigns}
%\numberwithin{equation}{subsection}
%\usepackage{numprint}

%% Graphics & Charts %%
\usepackage{graphicx}
%\usepackage{pdfpages}
%\usepackage{booktabs}
%\usepackage{array}
%\usepackage{paralist}
%\usepackage{framed}
%\usepackage{trfsigns}
\usepackage{tikz}
%\usepackage[lofdepth,lotdepth]{subfig}
%\usepackage{tikz}  %Graphen zeichnen
%\usetikzlibrary{decorations.pathmorphing}
%\usetikzlibrary{arrows.meta,arrows}
%\usepackage{pgfplots}
%% General Settings %%
%\setlength{\parindent}{0px}
%\setkomafont{captionlabel}{\normalfont\bfseries}

%\pagestyle{fancy}
%\lfoot{\tiny \today}
%\rfoot{\thepage\  / \pageref{LastPage}}
%\cfoot{}
%\renewcommand{\footrulewidth}{0.4pt}

%% provides command \uline{} for underlining words
%\usepackage{ulem}

%% colour headings
%\usepackage{color}
%\definecolor{bluen}{cmyk}{1,0.5,0,0}
%\definecolor{bloodorange}{cmyk}{0,.92,1,.2}
%\addtokomafont{section}{\color{bloodorange}}
%\addtokomafont{subsection}{\color{bloodorange}}
%\addtokomafont{subsubsection}{\color{bloodorange}}
%\addtokomafont{paragraph}{\small\color{bloodorange}}
%\addtokomafont{subparagraph}{\small\color{bloodorange}}

%% Signs & Special Formating %%
%\usepackage{ulem} %normalem: \emph{Text} is italic again.
%\usepackage{multicol,multirow}
%\usepackage{tabularx}
%\usepackage{stackrel}
%\usepackage{makeidx}
%\usepackage{mparhack} % bessere margiale bei seitenumbruch

% make document compact
\usepackage[compact]{titlesec}
\titlespacing{\section}{0pt}{*1}{*1}
\titlespacing{\subsection}{0pt}{*1}{*1}
\titlespacing{\subsubsection}{0pt}{*1}{*1}

\parindent 0pt
\pagestyle{empty}
\setlength{\unitlength}{1cm}
\setlist{leftmargin = *}

%include also newer PDF
\pdfminorversion=6

% Set the color of your style
% Avaiable are: Apricot, Aquamarine, Bittersweet, Black, Blue, blue, BlueGreen, BlueViolet, BrickRed, Brown, BurntOrange, CadetBlue, CarnationPink, Cerulean, CornflowerBlue, Cyan, Dandelion, DarkOrchid, Emerald, ForestGreen, Fuchsia, Goldenrod, Gray, Green, GreenYellow, JungleGreen, Lavender, ... (more at: http://en.wikibooks.org/wiki/LaTeX/Colors)
\def\StyleColor{BrickRed}

%
% General commands
% ===========================================================================
% Author:			Silvano Cortesi (cortesis@student.ethz.ch)
% Version:			1.2
% Last changed:		03.01.2018
%
% ---------------------------------------------------------------------------

%..ROEMISCHE_ZAHLEN
	\newcommand{\Roe}[1]{\uppercase\expandafter{\romannumeral #1 }}

%..ZAHLENMENGEN
	\newcommand{\N}{\mathbb{N}}
	\newcommand{\Z}{\mathbb{Z}}
	\newcommand{\Q}{\mathbb{Q}}
	\newcommand{\R}{\mathbb{R}}
	\newcommand{\real}{\R}
	\newcommand{\C}{\mathbb{C}}
	\newcommand{\complex}{\C}
	\newcommand{\0}{\mathbb{O}}
	\newcommand{\F}{\mathbb{F}}
	\newcommand{\K}{\mathbb{K}}
    \newcommand{\angstrom}{\textup{\AA}}
    
%..PFEILE
	\renewcommand{\leadsto}{\Longrightarrow}
	\newcommand{\leftrightleadsto}{\Longleftrightarrow}

%..VEKTOREN
	\newcommand{\Ul} {\underline}
	\newcommand{\vEx} {\vec{e}_x}
	\newcommand{\vEy} {\vec{e}_y}
	\newcommand{\vEz} {\vec{e}_z}
	\newcommand{\vEq} {\vec{e_1}}
	\newcommand{\vEw} {\vec{e_2}}
	\newcommand{\vEe} {\vec{e_3}}
	\newcommand{\transpose} {^{\text{T}}}
	\newcommand{\vect}[1]{\boldsymbol{#1}}
	
%..MATRIX
    \newcommand{\MATR}[1]{ \displaystyle \left( \begin{matrix} #1 \end{matrix} \right)}
    \newcommand{\MATRABS}[1]{ \displaystyle \left| \begin{matrix} #1 \end{matrix} \right|}


%..KOMPLEXE ZAHLEN
	\renewcommand{\Re}{\text{Re}\,}
	\renewcommand{\Im}{\text{Im}\,}

%..OPERATOREN
	\DeclareMathOperator{\grad}{grad}
	\renewcommand{\div}{\text{div}\,}
    	\DeclareMathOperator{\rot}{rot}
    	\DeclareMathOperator{\divg}{div}
    	\DeclareMathOperator{\Tr}{Tr}
    	\DeclareMathOperator{\const}{const}
	\DeclareMathOperator{\imag}{i}
	\newcommand{\Lapl}{\hbox{\footnotesize{$\Delta$}}}

%..DIFFERENTIALRECHNUNG
	\newcommand{\Dx} {\,\mathrm{d}}
	\newcommand{\abl}[1] {\frac{\mathrm{d}}{\mathrm{d}#1}}
	\newcommand{\Abl}[2] {\frac{\mathrm{d}#1}{\mathrm{d}#2}}
	\newcommand{\ablq}[1] {\frac{\mathrm{d^2}}{\mathrm{d}#1^2}}
	\newcommand{\Ablq}[2] {\frac{\mathrm{d^2}#1}{\mathrm{d}#2^2}}
	\newcommand{\pabl}[1] {\frac{\partial}{\partial#1}}
	\newcommand{\pablq}[1] {\frac{\partial^2}{\partial#1^2}}
	\newcommand{\Pabl}[2] {\frac{\partial#1}{\partial#2}}
	\newcommand{\Pablq}[2] {\frac{\partial^2#1}{\partial#2^2}}

%..INTEGRALRECHNUNG
	\newcommand{\dint}{\displaystyle{\int}}
	\newcommand{\intab}{\int^b_a}
	\newcommand{\intinf}{\int_{-\infty}^\infty}
	\newcommand{\dintab}{\displaystyle{\int^b_a}}
	\newcommand{\dintpi}{\displaystyle{\int^{\pi}_{-\pi}}}
	\newcommand{\dintzpi}{\displaystyle{\int^{2\pi}_{\mbox{-}2\pi}}}
	\newcommand{\dA}{\hspace{4pt}\mathrm{d}A}
	\newcommand{\dx}{\hspace{4pt}\mathrm{d}x}
	\newcommand{\dy}{\hspace{4pt}\mathrm{d}y}
	\newcommand{\dz}{\hspace{4pt}\mathrm{d}z}
	\newcommand{\dr}{\hspace{4pt}\mathrm{d}r}
	\newcommand{\ds}{\hspace{4pt}\mathrm{d}s}
	\newcommand{\dS}{\hspace{4pt}\mathrm{d}S}
	\newcommand{\dt}{\hspace{4pt}\mathrm{d}t}
	\newcommand{\dm}{\hspace{4pt}\mathrm{d}m}
	\newcommand{\dk}{\hspace{4pt}\mathrm{d}k}
	\newcommand{\dl}{\hspace{4pt}\mathrm{d}l}
	\newcommand{\du}{\hspace{4pt}\mathrm{d}u}
	\newcommand{\dv}{\hspace{4pt}\mathrm{d}v}
	\newcommand{\dV}{\hspace{4pt}\mathrm{d}V}
	\newcommand{\dphi}{\hspace{4pt}\mathrm{d}\varphi}
	\newcommand{\domega}{\hspace{4pt}\mathrm{d}\omega}
	\newcommand{\dvarsigma}{\hspace{4pt}\mathrm{d}\varsigma}
	\newcommand{\dtau}{\hspace{4pt}\mathrm{d}\tau}
	\newcommand{\dtheta}{\hspace{4pt}\mathrm{d}\vartheta}
	\newcommand{\dmu}{\hspace{4pt}\mathrm{d}\mu}
	\newcommand{\dxi}{\hspace{4pt}\mathrm{d}\xi}
	\newcommand{\deta}{\hspace{4pt}\mathrm{d}\eta}
	\newcommand{\dvecl}{\hspace{4pt}\mathrm{d}\vec{l}}
	\newcommand{\dvecS}{\hspace{4pt}\mathrm{d}\vec{S}}

%..LIMES
    \DeclareMathOperator{\limni}{\lim\limits_{n\to\infty}}
    \DeclareMathOperator{\limxi}{\lim\limits_{x\to\infty}}
    \DeclareMathOperator{\limho}{\lim\limits_{h\to0}}
    \newcommand{\limxai}[1]{\ensuremath{\lim\limits_{x\to #1}}}

%..SUMMEN
    \DeclareMathOperator{\sumni}{\sum_{n=0}^{\infty}}
    \newcommand{\sumnia}[1]{\ensuremath{\sum_{n=#1}^{\infty}}}


%..PARTIELLE ABLEITUNG
    \DeclareMathOperator{\partf}{\dfrac{\partial f}{\partial x}}
    \newcommand{\partfo}[1]{\ensuremath{\dfrac{\partial f}{\partial #1}}}
    \newcommand{\parto}[1]{\ensuremath{\dfrac{\partial }{\partial #1}}}
    \newcommand{\partt}[2]{\ensuremath{\dfrac{\partial^2 }{\partial #1\partial #2}}}
    \newcommand{\partq}[1]{\ensuremath{\dfrac{\partial^2 }{\partial #1^2}}}


%..ENUMERATION
    \newenvironment{abc}{\begin{enumerate}[(a)]}{\end{enumerate}}
    \newenvironment{cabc}{\begin{compactenum}[(a)]}{\end{compactenum}}
    \newenvironment{romanenum}{\begin{enumerate}[i.]}{\end{enumerate}}
    \newenvironment{cromanenum}{\begin{compactenum}[i.]}{\end{compactenum}}

%..FUNCTIONS
    \DeclareMathOperator{\arsinh}{arsinh}
    \DeclareMathOperator{\arcosh}{arcosh}
    \DeclareMathOperator{\artanh}{artanh}
    \DeclareMathOperator{\arcoth}{arcoth}
    \DeclareMathOperator{\arccot}{arccot}
    \DeclareMathOperator{\Arg}{Arg}
    \DeclareMathOperator{\Log}{Log}
    \newcommand{\dis}[1]{\hspace{#1cm}}
    \newcommand{\abs}[1]{\ensuremath{\left\vert#1\right\vert}}
    \newcommand{\attention}{\raisebox{-1pt}{{\makebox[1.6em][c]{\makebox[0pt][c]{\raisebox{.13em}{\small!}}\makebox[0pt][c]{\color{red}\Large$\bigtriangleup$}}}}}
    \DeclareMathOperator{\meq}{\stackrel{!}{=}}
    
    
% section color box
\setkomafont{section}{\mysection}
\newcommand{\mysection}[1]{%
    \Large\sffamily\bfseries%
    \setlength{\fboxsep}{0cm}%already boxed
    \colorbox{\StyleColor!40}{%
        \begin{minipage}{\linewidth}%
            \vspace*{2pt}%Space before
            #1
            \vspace*{-1pt}%Space after
        \end{minipage}%
    }}

%subsection color box
\setkomafont{subsection}{\mysubsection}
\newcommand{\mysubsection}[1]{%
    \normalsize \sffamily\bfseries%
    \setlength{\fboxsep}{0cm}%already boxed
    \colorbox{\StyleColor!20}{%
        \begin{minipage}{\linewidth}%
            \vspace*{2pt}%Space before
             #1
            \vspace*{-1pt}%Space after
        \end{minipage}%
    }}

%subsubsection color box
\setkomafont{subsubsection}{\mysubsubsection}
\newcommand{\mysubsubsection}[1]{%
	\normalsize \sffamily\bfseries%
	\setlength{\fboxsep}{0cm}%already boxed
	\colorbox{\StyleColor!10}{%
		\begin{minipage}{\linewidth}%
			\vspace*{2pt}%Space before
			#1
			\vspace*{-1pt}%Space after
		\end{minipage}%
}}

% highlighter
\newcommand{\hilight}[1]{\colorbox{\StyleColor}{#1}}
\newcommand{\highlighty}[1]{%
  \setlength{\fboxsep}{0pt}\colorbox{yellow!100}{\ensuremath{#1}}}

\newcommand{\highlightg}[1]{%
  \setlength{\fboxsep}{0pt}\colorbox{green!100}{\ensuremath{#1}}}

\newcommand{\highlightbg}[1]{%
   \colorbox{green!100}{$\displaystyle #1$}}  

% equation box        
\newcommand{\eqbox}[1]{\setlength{\fboxrule}{1mm}\fcolorbox{\StyleColor}{white}{\hspace{0.5em}$\displaystyle#1$\hspace{0.5em}}}

%center equationbox
\newcommand{\ceqbox}[1]{\vspace*{4pt} \begin{center}\eqbox{#1}\end{center}\vspace*{4pt}}


% This package makes formulas a bit more compact but less beautiful
% \usepackage{newtxtext,newtxmath}

% scala language description
\lstdefinelanguage{BNF}{%
    alsoletter={-},%
    sensitive,%
}[keywords,comments]%


% scala language description
\lstdefinelanguage{P4}{%
    sensitive,%
    backgroundcolor=\color{codeblue},%
%    morecomment=[l]//,%
%    morecomment=[s]{/*}{*/},%
}[keywords,comments]%

\definecolor{codeblue}{HTML}{DEF0FE}
\lstdefinestyle{P4style}{
    language=P4,%
    frame=none,%
    backgroundcolor=\color{codeblue},%
    keywords={action, action_function_declaration, action_profile, action_selector, algorithm, and, apply, attribute, attributes, bit, bytes, bytes_and_packets, calculated_field, control, counter, direct, dynamic_action_selection, else, extern, extern_type, extract, false, field_list, field_list_calculation, fields, header, header_type, hit, if, in, inout, input, instance_count, int, last, layout, mask, max, metadata, meter, method, min, min_width, miss, next, not, optional, or, output_width, packets, parse_error, parser, parser_drop, parser_exception, parser_value_set, primitive_action_declaration, range, register, result, return, saturating, select, selection_key, set_metadata, signed, static, table, true, update, valid, varbit, verify, width},%
    basicstyle=\ttfamily,%
    aboveskip=3mm,%
    belowskip=3mm,%
    fontadjust=true,%
    keepspaces=true,%
    keywordstyle=\bfseries,%
    captionpos=b,%
    framerule=0.3pt,%
    firstnumber=0,%
    numbersep=1.5mm,%
    numberstyle=\tiny,%
}
\lstset{%
    basicstyle=\ttfamily,%
%    language=P4,%
    aboveskip=3mm,%
    belowskip=3mm,%
    fontadjust=true,%
%    columns=[c]fixed,%
    keepspaces=true,%
%    commentstyle=\itshape,%
    keywordstyle=\bfseries,%
    captionpos=b,%
    framerule=0.3pt,%
    firstnumber=0,%
    numbersep=1.5mm,%
    numberstyle=\tiny,%
}

% \bibliography{semiconductordevices}
% \bibliographystyle{ieeetr}
\medmuskip=1mu

%change page style for header
\pagestyle{fancy}
\footskip 20pt

% Uncomment this line to make formulasheet ultra compact
% This removes
% - list of variables
% \newcommand{\makeultracompact}{irrelevant}
\let\makeultracompact\undefined

% Make stuff ultra compact if so desired
\ifdefined\makeultracompact
  \setlength{\parskip}{0pt}
  \setlength{\abovedisplayskip}{0pt}
  \setlength{\belowdisplayskip}{0pt}
  \setlength{\abovedisplayshortskip}{0pt}
  \setlength{\belowdisplayshortskip}{0pt}
\else
\fi
 
% -----------------------------------------------------------------------
\IfFileExists{../build/revision.tex}{
  \input{../build/revision.tex}
  \rhead{Compiled: \compiledate \hspace{1em} on: \hostname \hspace{1em} git-sha: \revision \hspace{1em} Noah Huetter}
}{\rhead{Noah Huetter}}

\ifdefined\makeultracompact
  \lhead{ETH Advanced Topics in Communication Networks 2019 \hspace{1em}compact version}
\else
  \lhead{ETH Advanced Topics in Communication Networks 2019}
\fi
\chead{\thepage}
\cfoot{}
\headheight 17pt \headsep 10pt
\title{ETH Advanced Topics in Communication Networks 2019}
\author{Noah Huetter}

\date{\today}
\begin{document}

\setcounter{page}{0}
\setcounter{secnumdepth}{2} %no enumeration of sections
\begin{multicols*}{2}
	\section*{Disclaimer}
	This summary is part of the lecture ``ETH Advanced Topics in Communication Networks'' (227-0575-00L) by Prof. Dr. Laurent Vanbever (FS19). It is based on the lecture. \\[6pt]
	Please report errors to \href{mailto:huettern@student.ethz.ch}{huettern@student.ethz.ch} such that others can benefit as well.\\[6pt]	
  The upstream repository can be found at \href{https://github.com/noah95/formulasheets}{https://github.com/noah95/formulasheets}
	\vfill\null
  \columnbreak
  %%%%%%%%%%%%%%%%%%%%%%%%%%%%%
  \tableofcontents
  \vfill\null
  %\columnbreak
  %%%%%%%%%%%%%%%%%%%%%%%%%%%%%
	\pagebreak
  \maketitle 
  \setcounter{page}{1}
  \thispagestyle{fancy}

  % ---------------------------------------------------------------------------
  \section{P4 language}
  % ---------------------------------------------------------------------------
  \subsection{Data types}

  \subsubsection{Base types}

  \begin{tabularx}{\linewidth}{ l X}
    \lstinline[style=P4style]!bool! &
    Boolean value\\
    \lstinline[style=P4style]!bit<W>! &
    Bit-string of width \texttt{W}\\
    \lstinline[style=P4style]!int<W>! &
    Signed integer of width \texttt{W}\\
    \lstinline[style=P4style]!varbit<W>! &
    Bit-string of dynamiv length \texttt{<=W}\\
    \lstinline[style=P4style]!match_kind! &
    describes ways to match table keys\\
    \lstinline[style=P4style]!error! &
    used to signal errors\\
    \lstinline[style=P4style]!void! &
    no values, used in few restricted circumstances\\
  \end{tabularx}

  \subsubsection{Header}
  Parsing a packet using \texttt{extract()} fills in the fields of the header.
  A successful \texttt{extract()} sets to true the validity bit of the extraced header.

  \begin{lstlisting}[style=P4style]
action:
header Ethernet_h {
  bit<48> dstAddr;
  bit<48> srcAddr;
  bit<16> etherType;
}
Ethernet_h ethernetHeader;
\end{lstlisting}

  \subsubsection{Header Stack}
  Array of up to \texttt{N} headers of type \texttt{Mpls\_h}.

\begin{lstlisting}[style=P4style]
header Mpls_h {
  bit<20> label;
  bit<3>  tc;
  bit     bos;
  bit<8>  ttl;
}
Mpls_h[10] mpls;
\end{lstlisting}

  \subsubsection{Header union}
Either \texttt{IPv4} or \texttt{IPv6} header is present.

\begin{lstlisting}[style=P4style]
header_union IP_h {
 IPv4_h v4;
 IPv6_h v6;
}
\end{lstlisting}


  \subsubsection{Struct}
  Unordered collection of named members.

\begin{lstlisting}[style=P4style]
struct standard_metadata_t {
 bit<9> ingress_port;
 bit<9> egress_spec;
 bit<9> egress_port;
 ...
}
\end{lstlisting}

  \subsubsection{Tuple}
  Unordered collection of unnamed members.

\begin{lstlisting}[style=P4style]
tuple<bit<32>, bool> x;
x = { 10, false }
\end{lstlisting}

  \subsubsection{Other}
  \begin{tabularx}{\linewidth}{ l X}
  \lstinline[style=P4style]!enum Prio \{High, Low\}! &
  Enumeration \\
  \lstinline[style=P4style]!typedef bit<48> macAddr_t! &
  Create subtypes \\
  \lstinline[style=P4style]!extern! &
  {}  \\
  \lstinline[style=P4style]!parser! &
  {}  \\
  \lstinline[style=P4style]!control! &
  {}  \\
  \lstinline[style=P4style]!package! &
  {}  \\
  \end{tabularx}

  \subsection{Data structures}

  \begin{lstlisting}[style=P4style]
struct standard_metadata_t {
  bit<9>  ingress_port;
  bit<9>  egress_spec;
  bit<9>  egress_port;
  bit<32> instance_type;
  bit<32> packet_length;
  bit<32> enq_timestamp;
  bit<19> enq_qdepth;
  bit<32> deq_timedelta;
  bit<19> deq_qdepth;

  // intrinsic to metadata v1model
  bit<48> ingress_global_timestamp;
  bit<48> egress_global_timestamp;
  bit<16> mcast_grp;
  bit<16> egress_rid;
  bit<1>  checksum_error;
  error parser_error;
  bit<3> priority;
}\end{lstlisting}

\begin{tabularx}{\linewidth}{ l X }
  \lstinline[style=P4style]!ingress_port! &
  Port the packet was received \\
  \lstinline[style=P4style]!egress_spec! &
  Port where the packet will be transmitted \\
\end{tabularx}

  \subsection{Examples}

  \subsubsection{Multicast group ID}
  For a L2-switch to work, it has to multicast packets. Simple switch has a multicast features that
  allows setting a metadata field so that packets are sent to multiple ports.

  First, on the switch, create a multicast group, node and assign these.
  \begin{lstlisting}[language=bash]
mc_mgrp_create <mcast_grp_id>
mc_node_create <replication_id> <port_number> [port_number]
mc_node_associate <mcast_grp_id> <node_handle_id>\end{lstlisting}
  \texttt{mc\_node\_create} returns the \texttt{node\_handle\_id}. For every \texttt{replication\_id} the \texttt{mc\_node\_create} is incremented.

  Example: Mulcicast \texttt{mcast\_grp\_id=1} with \texttt{replication\_id=0} on ports \texttt{1,2,3,4} 
  \begin{lstlisting}[language=bash]
mc_mgrp_create 1
mc_node_create 0 1 2 3 4
mc_node_associate 1 0\end{lstlisting}

  To send a packet to this group, in P4 set the \texttt{standard\_metadata.mcast\_grp} to \texttt{mcast\_grp\_id} during the ingress pipeline.

  \subsubsection{Cloning Packets}
  Is a feature of simple switch.

  Configure the switch with a mirroring session \texttt{session} mapped to an \texttt{output\_port}.
  \begin{lstlisting}[style=P4style]
mirroring_add <session> <output_port>\end{lstlisting}

  During ingress pipeline, a packet can be cloned using \texttt{clone}. The packet will be cloned to the egress pipeline and port configured by \texttt{mirroring\_add}. When using \texttt{clone3} you can add as a third parameter a metadata \texttt{struct}.
  \begin{lstlisting}[style=P4style]
clone(CloneType.I2E, <session>)
// Generally
clone(in CloneType type, in bit<32> session)
clone3<T>(in CloneType type, in bit<32> session, in T data)\end{lstlisting}

  To identify a cloned packet in the egress pipeline, check the \texttt{standard\_metadata.instance\_type==1} field.

  When a packet is cloned all its metadata fields
  are reset to the default value (usually 0).


  \subsubsection{Digest packets}
  A feature of simple switch to send packets to the controller without having to clone it.

  \begin{lstlisting}[style=P4style]
// Define a struct that is sent to the controller
struct learn_t {
    macAddr_t srcAddr;
    inPort_t  inPort;
}
// Add the struct to the metadata
struct metadata {
    /* empty */
    //TODO 3: delcare one learn_t variable
    learn_t learn;
}
// fill in the meatdata fields
meta.learn.srcAddr = hdr.ethernet.srcAddr;
meta.learn.inPort = (inPort_t)standard_metadata.ingress_port;
// send to the controller
// nothing else has to be done
digest(1, meta.learn); // first argument is always 1\end{lstlisting}

  
\end{multicols*}

\setcounter{secnumdepth}{2}
\end{document}
